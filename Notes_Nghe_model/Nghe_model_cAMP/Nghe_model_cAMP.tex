%% Based on a TeXnicCenter-Template by Tino Weinkauf.
%%%%%%%%%%%%%%%%%%%%%%%%%%%%%%%%%%%%%%%%%%%%%%%%%%%%%%%%%%%%%

%%%%%%%%%%%%%%%%%%%%%%%%%%%%%%%%%%%%%%%%%%%%%%%%%%%%%%%%%%%%%
%% HEADER
%%%%%%%%%%%%%%%%%%%%%%%%%%%%%%%%%%%%%%%%%%%%%%%%%%%%%%%%%%%%%
\documentclass[a4paper,twoside,10pt]{report}
% Alternative Options:
%	Paper Size: a4paper / a5paper / b5paper / letterpaper / legalpaper / executivepaper
% Duplex: oneside / twoside
% Base Font Size: 10pt / 11pt / 12pt


%% Language %%%%%%%%%%%%%%%%%%%%%%%%%%%%%%%%%%%%%%%%%%%%%%%%%
\usepackage[USenglish]{babel} %francais, polish, spanish, ...
\usepackage[T1]{fontenc}
\usepackage[ansinew]{inputenc}

\usepackage{lmodern} %Type1-font for non-english texts and characters


%% Packages for Graphics & Figures %%%%%%%%%%%%%%%%%%%%%%%%%%
\usepackage{graphicx} %%For loading graphic files
\usepackage{subfig} %%Subfigures inside a figure
%\usepackage{pst-all} %%PSTricks - not useable with pdfLaTeX

%% Please note:
%% Images can be included using \includegraphics{Dateiname}
%% resp. using the dialog in the Insert menu.
%% 
%% The mode "LaTeX => PDF" allows the following formats:
%%   .jpg  .png  .pdf  .mps
%% 
%% The modes "LaTeX => DVI", "LaTeX => PS" und "LaTeX => PS => PDF"
%% allow the following formats:
%%   .eps  .ps  .bmp  .pict  .pntg


%% Math Packages %%%%%%%%%%%%%%%%%%%%%%%%%%%%%%%%%%%%%%%%%%%%
\usepackage{amsmath}
\usepackage{amsthm}
\usepackage{amsfonts}


%% Line Spacing %%%%%%%%%%%%%%%%%%%%%%%%%%%%%%%%%%%%%%%%%%%%%
%\usepackage{setspace}
%\singlespacing        %% 1-spacing (default)
%\onehalfspacing       %% 1,5-spacing
%\doublespacing        %% 2-spacing


%% Other Packages %%%%%%%%%%%%%%%%%%%%%%%%%%%%%%%%%%%%%%%%%%%
%\usepackage{a4wide} %%Smaller margins = more text per page.
%\usepackage{fancyhdr} %%Fancy headings
%\usepackage{longtable} %%For tables, that exceed one page


%%%%%%%%%%%%%%%%%%%%%%%%%%%%%%%%%%%%%%%%%%%%%%%%%%%%%%%%%%%%%
%% Remarks
%%%%%%%%%%%%%%%%%%%%%%%%%%%%%%%%%%%%%%%%%%%%%%%%%%%%%%%%%%%%%
%
% TODO:
% 1. Edit the used packages and their options (see above).
% 2. If you want, add a BibTeX-File to the project
%    (e.g., 'literature.bib').
% 3. Happy TeXing!
%
%%%%%%%%%%%%%%%%%%%%%%%%%%%%%%%%%%%%%%%%%%%%%%%%%%%%%%%%%%%%%

%%%%%%%%%%%%%%%%%%%%%%%%%%%%%%%%%%%%%%%%%%%%%%%%%%%%%%%%%%%%%
%% Options / Modifications
%%%%%%%%%%%%%%%%%%%%%%%%%%%%%%%%%%%%%%%%%%%%%%%%%%%%%%%%%%%%%

%\input{options} %You need a file 'options.tex' for this
%% ==> TeXnicCenter supplies some possible option files
%% ==> with its templates (File | New from Template...).



%%%%%%%%%%%%%%%%%%%%%%%%%%%%%%%%%%%%%%%%%%%%%%%%%%%%%%%%%%%%%
%% DOCUMENT
%%%%%%%%%%%%%%%%%%%%%%%%%%%%%%%%%%%%%%%%%%%%%%%%%%%%%%%%%%%%%
\begin{document}

\pagestyle{empty} %No headings for the first pages.


%% Title Page %%%%%%%%%%%%%%%%%%%%%%%%%%%%%%%%%%%%%%%%%%%%%%%
%% ==> Write your text here or include other files.

%% The simple version:
\title{Notes on modeling of growth-regulation interactions in the cAMP system}
\author{Martijn Wehrens}
%\date{} %%If commented, the current date is used.
\maketitle

%% The nice version:
%\input{titlepage} %%You need a file 'titlepage.tex' for this.
%% ==> TeXnicCenter supplies a possible titlepage file
%% ==> with its templates (File | New from Template...).


%% Inhaltsverzeichnis %%%%%%%%%%%%%%%%%%%%%%%%%%%%%%%%%%%%%%%
%\tableofcontents %Table of contents
\cleardoublepage %The first chapter should start on an odd page.

\pagestyle{plain} %Now display headings: headings / fancy / ...



%% Chapters %%%%%%%%%%%%%%%%%%%%%%%%%%%%%%%%%%%%%%%%%%%%%%%%%
%% ==> Write your text here or include other files.

%\input{intro} %You need a file 'intro.tex' for this.


%%%%%%%%%%%%%%%%%%%%%%%%%%%%%%%%%%%%%%%%%%%%%%%%%%%%%%%%%%%%%
%% ==> Some hints are following:

%\chapter{Some small hints}\label{hints}

%\section{German Umlauts and other Language Specific Characters}\label{umlauts}

\chapter{Modeling the noise transmission}

\section{Ornstein-Uhlenbeck fluctuations model}

%\subsection{The model assumptions}

%\begin{figure}
    %    \label{fig:modeldrawing}
%    \centering
%    \includegraphics[width=0.9\textwidth]{drawings_png\model_nghe.png}
%    \caption{Model as proposed by Kiviet 2014 et al.}
%\end{figure}

\subsection{Implicit noise equations}

Initially, I used the following model were ODEs are numerically solved:

\begin{align}
\label{myfirstequation}
\dot{M} = & - \frac{(M-M_0)}{\tau}  \nonumber \\ 
          & + c_M \cdot \Gamma_M  \nonumber \\ % (1-T_{E\rightarrow M})
          & + T_{E\rightarrow M} \cdot c_M \cdot (\frac{E}{E_0} - 1)  
\end{align}
% deltaMetabolism = -(metabolismValues(end)-parameters.metabolism0) * 1/parameters.dampingTimeMetabolism + ... % damping; 1 is the equilibrium value
% (1-parameters.transmissionEnzymeMetabolism) * ((rand()-.5)*2) * parameters.noiseSizeMetabolism + ...                              % white noise component
% parameters.transmissionEnzymeMetabolism * (enzymeValues(end)/parameters.enzymeTarget-.5) * parameters.noiseSizeMetabolism;             % noise component due enzyme


\begin{align}
	\dot{\lambda} = & -\frac{(\lambda - \lambda_0 )}{\tau_\lambda} \nonumber \\ 
 			& + c_\lambda \cdot \Gamma_\lambda \nonumber \\  %  (1-T_{M\rightarrow\lambda}) \cdot
			& +    T_{M\rightarrow\lambda} \cdot c_\lambda \cdot (\frac{M}{M_0}-1) 
\end{align}

\begin{align}
\label{mythirdequation}
\dot{P} = & - \frac{(P-P_0)}{\tau_P} \nonumber \\ 
		 & + c_\lambda \cdot \Gamma_\lambda \nonumber \\ % (1-T_{M\rightarrow P}) \cdot
         & + T_{M\rightarrow P} \cdot c_P \cdot (\frac{M}{M_0}-1)  \nonumber \\ 
         & + R_{M\rightarrow P} \cdot c_P \cdot (\frac{M}{M_0}-1)
\end{align}

\begin{align}
\label{mylastequation}
\dot{E} = P - \lambda E
\end{align}

Where $M$ describes the state of the metabolism, $\lambda$ is the growth rate, $P$ is the production rate, $E$ is the amount of enzyme, $\tau$ is a dampening term ($X_0$ is the equilibrium value), $T_{X \rightarrow Y}$ is the noise transmission constant from $X$ towards $Y$, $c_X$ is a constant that sets the size of the fluctuations, $\Gamma_X$ is a white noise source. %The terms $(1-T_{X \rightarrow Y}$  )
$R_{X \rightarrow Y}$ indicates a regulatory interaction, but this notation is currently just cosmetical, as $T_\text{effective}=T+R$.

This is similar to the model that Philipe Nghe had in Kiviet et al. \cite{Kiviet2014}, which was inspired by \cite{Dunlop2008} (see supplement).
A major difference between my model and the Nghe/Dunlop model is that in the Nghe/Dunlup model dampening occurs on the noise only, whilst in my model, dampening occurs on the parameters of interest themselves.
(In the Dunlop paper itself, where the model is intended to describe something else, there are two dampening terms, dampening both the noise and the other parameters respectively; see below.)

\section{Separate noise equations}

A model with separate noise sources looks as follows:

The noise term is described by:

\begin{align}
\label{generalgillespienoise}
\dot{N}_X = \sqrt{c_X} \cdot \Gamma_X - N_X/\tau
\end{align}

Where I mainly used Daniel Gillespie's notation \cite{Gillespie1996}.
With for our case $X$ equaling $\lambda$, $M$ or $P$. Note that $\tau^{-1}=\beta$ ($\beta$ is used in Nghe/Dunlop).
Now, in the Dunlop model, which models a completely different process than the one described here \cite{Dunlop2008}, the \textit{solutions} of the ODEs describing the noise are plugged into the ODEs describing the protein dynamics. This leads to an additional memory effect.
%
That is:

\begin{align}
\label{dunlopgeneralequation}
\dot{X} = & N_X  + F(X) + X/\tau
,
\end{align}

with $F(X)$ some arbitrary function of $X$. 
Note that the $N_X$ function also contains a $\tau$ term, which is effectively integrated, thus leading to effects of the fluctuations much longer timescales than $\tau$. 
This effect is partially countered by the third term in Eq. \ref{dunlopgeneralequation}, which also contains the $\tau$ term.

\section{Nghe/Dunlop model}

The Nghe/Dunlop model does not seem to have this integration of noise, but instead defines:


\begin{align}
\dot{E} = P - \lambda E
\end{align}

(which is the same in earlier mentioned models), and:

\begin{align}
\mu = T_{\mu \leftarrow E} E + T_{\mu \leftarrow G} \Gamma_G + \Gamma_\mu
,
\end{align}

\begin{align}
\label{productionNghe}
p = T_{E \leftarrow E} E + T_{E \leftarrow G} \Gamma_G + \Gamma_E
,
\end{align}

where I here left out the normalization constants, $X_0$, and the fact that parameters were rewritten as deviations from their means ($X=\bar{X}+\delta{X}$).
I am currently not sure how these equations relate to my own equations, as subscripts in Eq. \ref{productionNghe} seem inconsistent with the fact that transmission should be towards $P$ (e.g. what is $T_{E\leftarrow E}$?).

In any case, given that noise and other parameters are related in terms of parameters (not derivatives), the following formulae are probably underlying the Nghe/Dunlop model:

\begin{align}
\dot{M} = & \dot{N}_M  \nonumber \\ 
& + T_{E\rightarrow M} \cdot c_M \cdot (\frac{E}{E_0} - 1)  
\end{align}

\begin{align}
\dot{\lambda} = & \dot{N}_\lambda \nonumber \\ 
& +    T_{M\rightarrow\lambda} \cdot c_\lambda \cdot (\frac{M}{M_0}-1) 
\end{align}

\begin{align}
\dot{P} = & \dot{N}_P \nonumber \\ 
& + T_{M\rightarrow P} \cdot c_P \cdot (\frac{M}{M_0}-1)  \nonumber \\ 
& + R_{M\rightarrow P} \cdot c_P \cdot (\frac{M}{M_0}-1)
\end{align}

Where the normalizations by $X_0$ were left out again.
Note however, that this results in the absence of dampening terms on the parameters $X$ themselves, which leads to non-steady state behavior of the parameters.
The term $X/\tau$ could be added to each of the equations to resolve this issue; the term $-N_X/\tau$ from the noise ODEs could then be dropped.



%%%%%%%%%%%%%%

\section{Obtaining the cross correlations}

\subsection{Obtaining solutions in Fourier space}

How do we get the cross correlations? A way to define correlations is\footnote{From Wikipedia, https://en.wikipedia.org/wiki/Cross-correlation and Weisstein, Eric W. "Cross-Correlation Theorem." From MathWorld--A Wolfram Web Resource. http://mathworld.wolfram.com/Cross-CorrelationTheorem.html}:

\begin{align}
R_{f,g}(\tau) = f \star g = \int_{\tau=\infty}^{\infty} {\bar f(\tau) g(t+\tau) \delta \tau}
\end{align}

With the $\bar{f}$ denoting the complex conjugate. This is equal to the convolution of $f^*(-t)$ and $g(t)$, 

\begin{align}
f \star g = f*(-t) * g
.
\end{align}

It can be derived that:

\begin{align}
\mathcal{F} (f \star g) = \overline{\mathcal{F} (f)} \mathcal{F}(g)
.
\end{align}

This property can be exploited to find the cross correlation that we are interested in:

\begin{align}
R_{\mu,E}(\tau) = \mathcal{F}^{-1} \left( \overline{\mathcal{F} (\mu)} \mathcal{F}(E) \right)
.
\end{align}

First we find the solutions to the ODEs in Fourier space:

\begin{align}
\label{ODEEFourier}
\mathcal{F} \left( \dot{E} \right) = & \mathcal{F} \left( P - \lambda E \right) \nonumber \\
%
i\omega \tilde{E} = & \tilde{P} - \tilde{\lambda} \tilde{E}
\end{align}

\begin{align}
 \tilde{E} = & \frac{1}{i\omega+\tilde{\lambda}} \tilde{P} 
\end{align}

For Eq. \ref{generalgillespienoise} the solution in Fourier space is:

\begin{align}
\tilde{N}_X = \frac{\sqrt{c_x}}{i\omega - 1/\tau} \tilde{\Gamma}_x
\end{align}

which however will not be directly used to solve Eq. \ref{myfirstequation}-\ref{mylastequation}. With regard to  Eq. \ref{myfirstequation}-\ref{mythirdequation}, the solutions in general terms can be found from taking the Fourier transform:

\begin{align}
i \omega \tilde{X} = - \frac{\tilde{X}-X_0}{\tau_X} + c_X \tilde{\Gamma}_X + T_{Y \rightarrow X} c_X (\frac{\tilde{Y}}{X_0} - 1)
\end{align}

which, solving for $X$ gives:

\begin{align}
\label{solutionXFourier}
\tilde{X} = \left( i \omega + \frac{1}{\tau_X} \right)^{-1} \left( \frac{X_0}{\tau_X} + c_X \tilde{\Gamma}_X + T_{Y \rightarrow X} c_X (\frac{\tilde{Y}}{Y_0} - 1) \right)
,
\end{align}

$X$ being either $\lambda$, $M$ or $P$.
%
The solution for $E$ in Fourier space (i.e. to Eq. \ref{mylastequation}) without $P$ or $\lambda$ terms is very involved. This can be seen by plugging Eq. \ref{solutionXFourier} into Eq. \ref{ODEEFourier}, which results in:

\begin{align}
\label{ODEEFourierExpanded}
i\omega \tilde{E} = & 
%P
\left( i \omega + \frac{1}{\tau_P} \right)^{-1} \left( \frac{P_0}{\tau_P} + c_P \tilde{\Gamma}_P + T_{M \rightarrow P} c_P (\frac{\tilde{M}}{M_0} - 1) \right)
     \nonumber \\
    & -
    % lambda
    \left( i \omega + \frac{1}{\tau_\lambda} \right)^{-1} \left( \frac{\lambda_0}{\tau_\lambda} + c_\lambda \tilde{\Gamma}_\lambda + T_{M \rightarrow \lambda} c_\lambda (\frac{\tilde{M}}{M_0} - 1) \right)
     \tilde{E}
\end{align}
Where M is defined as:

\begin{align}
\label{solutionForM}
\tilde{M} = \left( i \omega + \frac{1}{\tau_M} \right)^{-1} \left( \frac{M_0}{\tau_M} + c_M \tilde{\Gamma}_M + T_{E \rightarrow M} c_M (\frac{\tilde{E}}{E_0} - 1) \right)
\end{align}

Plugging Eq. \ref{solutionForM} into Eq. \ref{ODEEFourierExpanded} results in an equation where the only time-dependent parameter is E.
Solving that formula for E gives a very involved formula, which is not displayed here.

\subsection{Actually getting the cross-correlations}

As Dunlop et al. point out, we are now looking at the expectation values of the correlation functions (since we're talking about noise), so in general terms:

\begin{align}
\left< R_{X,Y}(\tau)\right > = 
\left< \mathcal{F}^{-1} \left[ \overline{\mathcal{F} (X)} \mathcal{F}(Y) \right] \right>
\end{align}



\section{Notes}

Note that Philippe Nghe's scripts can be found at:

\begin{verbatim}
    \\storage01\data\AMOLF\groups\tans-group\Former-Users\
    Nghe\Noise_Growth\matlab3
\end{verbatim}


%% <== End of hints
%%%%%%%%%%%%%%%%%%%%%%%%%%%%%%%%%%%%%%%%%%%%%%%%%%%%%%%%%%%%%



%%%%%%%%%%%%%%%%%%%%%%%%%%%%%%%%%%%%%%%%%%%%%%%%%%%%%%%%%%%%%
%% BIBLIOGRAPHY AND OTHER LISTS
%%%%%%%%%%%%%%%%%%%%%%%%%%%%%%%%%%%%%%%%%%%%%%%%%%%%%%%%%%%%%
%% A small distance to the other stuff in the table of contents (toc)
\addtocontents{toc}{\protect\vspace*{\baselineskip}}

%% The Bibliography
%% ==> You need a file 'literature.bib' for this.
%% ==> You need to run BibTeX for this (Project | Properties... | Uses BibTeX)
%\addcontentsline{toc}{chapter}{Bibliography} %'Bibliography' into toc
%\nocite{*} %Even non-cited BibTeX-Entries will be shown.
%\bibliographystyle{alpha} %Style of Bibliography: plain / apalike / amsalpha / ...
\bibliography{./../../library} %You need a file 'literature.bib' for this.
\bibliographystyle{plain}

%%% The List of Figures
%\clearpage
%\addcontentsline{toc}{chapter}{List of Figures}
%\listoffigures
%
%%% The List of Tables
%\clearpage
%\addcontentsline{toc}{chapter}{List of Tables}
%\listoftables
%%}

%%%%%%%%%%%%%%%%%%%%%%%%%%%%%%%%%%%%%%%%%%%%%%%%%%%%%%%%%%%%%
%% APPENDICES
%%%%%%%%%%%%%%%%%%%%%%%%%%%%%%%%%%%%%%%%%%%%%%%%%%%%%%%%%%%%%
\appendix
%% ==> Write your text here or include other files.

%\input{FileName} %You need a file 'FileName.tex' for this.


\end{document}

