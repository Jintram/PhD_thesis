



%\chapter{The Ribosomal City}
%\chapter{The Ribosome, a puzzling protein-rna complex}
\chapter{Ribosomal dynamics, a puzzling affair}
\label{chapter:ribosomes}


%%%%%%%%%%%%%%%%%%%%%%%%%%%%%%%%%%%%%%%%%%%%%%%%%%%%%%%%%%%%%%%%%%%%%%%%%%%%%%%%%%%%%%%%%%%%%%%%%%%%%%%%%%%%%%%%%%%%%%%%%%%%%%%%%%%%%%%%%%%%%%%%
% Picture of the ribosome
%%%%%%%%%%%%%%%%%%%%%%%%%%%%%%%%%%%%%%%%%%%%%%%%%%%%%%%%%%%%%%%%%%%%%%%%%%%%%%%%%%%%%%%%%%%%%%%%%%%%%%%%%%%%%%%%%%%%%%%%%%%%%%%%%%%%%%%%%%%%%%%%

\begin{figure}
    \centering    
    %\includegraphics[width=1.0\textwidth]{pdf_2016-02-17_pos2_L31-mCerulean_clouds.pdf}
    \includegraphics[width=0.8\textwidth]{ribosomes_allproteinscolored.png}
    \caption{ 
        \textbf{Picture of the ribosome.}
        The ribosome consist of a small and a large subunit. These two come together on messenger RNA templates to perform protein synthesis.
        The complete ribosome is made of 3 ribosomal RNA molecules and more than 50 proteins, shown in grey and color in the picture above, respectively \cite{Chen2013}.
        The 16S rRNA and 23S rRNA molecules function as backbone. % for the small and large subunits, respectively.
        Proteins labeled S1 up to S22 
        % (S22 being sub-stoichiometric) 
        bind to the 16S rRNA to form the small subunit, and proteins labeled L1-L36 and the 5S rRNA bind to the 23S rRNA to form the large subunit \cite{Keseler2017}.
        The Ecocyc database lists 58 ribsomal proteins, though also different numbers of ribosomal proteins are listed (ref. \cite{Chen2013} e.g. talks about 54 ribosomal proteins).
        % The 5S rRNA also binds to the large subunit.
        Interestingly, the ribosomal RNA constitutes 73-80\% of the total RNA found in an \textit{E. coli} cell (mRNA constitutes 3-4.5\% and tRNA 15-20\%) \cite{Norris1972}.
        Assembly of the ribosome also requires many co-factors \cite{Chen2013}. % note also DnaK is required.
        %
        This image is based on x-ray crystallography (PDB ID: 4v4q, \cite{Schuwirth2005}). Pdb files downloaded from \texttt{www.rcsb.org} \cite{Berman2000} 
        visualized with UCSF Chimera (version 1.11.2, build 41376, \cite{pettersen2004}).
        % Previously, I seemed to have used 4v4a, which is from 2003 http://www.rcsb.org/pdb/explore.do?structureId=4v4a.        
        %
        %        
    }
    \label{fig:ribo:pictureofribo}
\end{figure}

%%%%%%%%%%%%%%%%%%%%%%%%%%%%%%%%%%%%%%%%%%%%%%%%%%%%%%%%%%%%%%%%%%%%%%%%%%%%%%%%%%%%%%%%%%%%%%%%%%%%%%%%%%%%%%%%%%%%%%%%%%%%%%%%%%%%%%%%%%%%%%%%

\section{Introduction}

\subsection{The central role of the ribosome}

Ribosomes are life.
%
In 1958, Francis Crick, who discovered the structure of DNA with James Watson and Rosalind Franklin,
formulated what is called the central dogma of molecular biology.
This states 
% declares 
that DNA holds the amino acid sequence information required to build a protein, and that this sequence information is first translated to RNA, which is then transcribed into a protein \cite{Crick1958}.
% 
This dogma relates immediately to some of the features that are said to define life: 
the capability to store information and 
the regeneration of components from scratch \cite{Lawrence2005, Koshland2002}.
%
Also around that time, 
Slightly earlier, in 1955, "new cytoplasmic components" were already discovered under an electron microscope \cite{Palade1955}.
These components turned out to be the components in the cell that perform the last step of the central dogma: 
the synthesis of proteins using mRNA molecules as templates; also known as transcription.
The components became known as ribosomes.
%
% Given the central place of the ribosome, in the circle of life, and 
Given the idea that early life consisted only of RNA molecules that could catalyze other reactions (the "RNA world" \cite{Campbell2002}), it is not surprising that the ribosomes 
% these components of the cell
consist of both catalytic RNA and proteins,
as shown in figure \ref{fig:ribo:pictureofribo}.
%
%As such, it is not only an embodiment of the central dogma, but also the exception to the rule.
As such, we can say ribosomes are a key element of life.

% life
% Lawrence: information storage, catalysts, energy relation environment, grow and reproduce, respond stimuli
% Koshland: program, improvisation (MW: long term adaptation), compartmentalization, energy, regeneration, adaptabilty (short term adaptation), seclusion (being able to separate different processes, e.g. enzyme's specifity).

\subsection{Understanding the role of ribosomes in protein fluctuations}

In this thesis, 
we have been trying to further 
% When we want to understand the cell, ribosomes are a good starting point.
%It is our aim to better 
understand % cellular dynamics.
temporal fluctuations in the concentrations of cellular components.%, which occur due to the intrinsic stochastic nature of chemical processes that underlay cellular 
%
As mentioned in earlier chapters (\ref{chapter:literaturereview} and  \ref{chapter:CRP}),
these fluctuations are a result of the intrinsic stochastic nature of chemical processes that occur in the cell.
% (see also previous chapters \ref{chapter:literaturereview} and  \ref{chapter:CRP}).
%
Given the central role of the ribosome in the cell that was just discussed, ribosomes might play a big role in such fluctuations.
%Since ribosomes are key components in the cell, 
Indeed, it is often suggested that fluctuations in ribosomal concentration could result in cell-wide protein concentration fluctuations \cite{Davidson2008, Raj2008, Chalancon2012, Bruggeman2018}.
%
Such concerted concentration fluctuations are also referred to as extrinsic noise (as opposed to intrinsic noise, fluctuations that only occur in one cellular species) \cite{Elowitz2002}.

\subsection{Quantifying ribosomal dynamics}
\label{section:ribos:walker}

In a previous thesis from the Tans lab, Noreen Walker 
% In previous work from our lab, Walker et al. 
quantified the dynamic relationship between single cell ribosomal expression fluctuations and growth in steady state conditions \cite{Walker2016t}.
%
The experiments involved presented many challenges, 
of which a number is described in her thesis.
%
% The results from these experiments were rather inconclusive.
%
Given previously described central role of the ribosome, 
it was hypothesized that the stochastically fluctuating concentration of ribosomes might be limiting, 
meaning that ribosomal fluctuations might result in growth rate fluctuations.
%
In single cell time lapse experiments, no clear indications were found to support this hypothesis.
%
Instead, 
ribosomal proteins L19 and L31 that were labeled with mCherry (L31-R and L19-R)
% mCherry-labeled L31 and L19 subunits (L31-R and L19-R) 
showed expression-growth cross-correlations (CCs) indicative of dilution-scenario behaviour in minimal medium, or none at all in rich medium (see chapters \ref{chapter:literaturereview} and \ref{chapter:CRP} for discussions about the use of cross-correlations to interpret dynamics). 
mCerulean labeled L31 ribosomal protein (L31-C) showed expression-growth CCs with small correlations around zero delay 
% at positive delays 
%between expression and growth 
in minimal medium.
This difference between the L31-C and L31-R experiments was unexpected, as only the label was different.
%
In rich medium, the L31-C cross-correlation was consistent with the L31-R cross-correlation, and showed almost no correlation.
%
Since the ribosome mainly consists of RNA, 
a GFP reporter under the control of one of the ribosomal RNA promoters was also used (rrna-G).
Like the L31-C reporter, the dynamics of this rrna-G reporter showed positive correlations in minimal medium and only very small correlations in rich medium.
% an additional reporter was used which consisted of the ribosomal RNA promoter 
%
Also growth-expression scatter plots were used, to investigate possible interesting shapes of growth-expression relationships, but this yielded no noteworthy shapes.

In an attempt to force the cell in a scenario where ribosomes are limiting, experiments were conducted where cells were grown in the presence of sub-inhibitory concentrations of tetracycline, an antibiotic.
%
Though also interesting in itself, the outcome of such an experiment could serve as a reference to interpret other experiments.
%
In rich medium, both for the L31-R and the rrn-G reporter, the antibiotic did not result in more limiting behaviour, i.e. correlations did not become more positive.
%
Given the disparate observations, both between reporters and conditions, 
and the absence of a point of reference,
the nature of the ribosomal dynamics remained fairly elusive; it was concluded that ribosomal fluctuations perhaps do not have a pivotal role in steady state cellular growth dynamics.

\subsection{New work}

In this chapter, we will describe a few additional experiments that were aimed at gaining further insights in the ribosomal dynamics.
Specifically, we tried to understand whether fluctuations in ribosomal concentration could have cell-wide implications.
%
To answer this question, we tried both new experiments as well increase the throughput of existing experiments.
%
Since this is an ongoing project, this chapter will be more succinct than previous ones.

\section{Results}
% > looked at dual reporters with pn25 reporter to assess effect from ribosomal concentration > protein production
    % ^> (outlook) Note that also R_CR-pq would be interesting in future with minor technical adjustment
% > looked at shift experiment to look at a (dynamic) environment where we knew cells with more ribosomes have an advantage
% > Additional semi-steady state measurements w/ and w/o antibiotics (incl. the pn25 reporter)
% > (outlook) we also looked into adding a plasmid that would introduce a burden unto the ribosomes 
%	% ^> Note that cells will adjust, so in a sense this might not increase the effect of fluctuations (perhaps put in sketch of two graphs with two optimums)
% > (outlook) we also looked into rrna reporter in delta(rna) strains
% > (outlook) maybe r-proteins are not as stochastically equivalent as sometimes assumed, ratio Y_m9:Y_ab and R_m9:R_ab not equal (2.4 and 2.3)
    % ^> what about production though, that's ratio_Y 3.18 vs. 2.15, should these ratios not be equal? --> not to C-ratio as the growth rates differ.
% > (outlook) ppgpp titration
% > (outlook) perhpas the fluctuations in total ribosome activity are a difficult function of fluctuations in all of the separate ribosomal components, which all might have their own (partially) independent contributions, and therefor it is hard to see single correlations. (the C

\subsection{Additional strains}

\begin{table}[h]
    \begin{tabularx}{\textwidth}{llXl}
        %
        \textbf{ASC number}	& \textbf{Shorthand} & \textbf{Description} & \textbf{Source}		\\
        \hline
        ASC976  &	Prrna-C, pn25-Y	&	Δphp::pn25-mVenus-cmR, Δche::Prrsa-mCerulean-kanR. (Kanamycin and chloramphenicol resistance.)  & VS \\
        %     
        ASC968	& L19-C, pn25-Y	& L19-gc-mCerulean-kanR (GC linker), Δ(…)::pn25-mVenus-cmR.	(Kanamycin and chloramphenicol resistance.)	& VS \\
        %
        ASC1058	& L9-R, S2-Y	& Also known as JE202. rplI-mCherry-KanR (L9), rpsB-venus-CmR (S2). (Kanamycin and chloramphenicol resistance.) Gift from Johan Elf lab. & \cite{Wallden2016} \\
        \hline
        %
        % ASC631 & pn25-R, lacA-G & $\Delta$lacA::gfp-cat, $\Delta$php::pn25-mCherry-kanR & \cite{Kiviet2014} \\
        ASC666 & L31-R, gltA-G  & L31::mCherry-kanR, gltA::gfpA206K-cat & \cite{Kiviet2014} \\
        ASC810	& L31-C & L31-gc-mCerulean-kanR (GC linker). (Kanamycin resistant.) & NW, VS \\
        \hline
    \end{tabularx}
    \caption{\textbf{Strains used in this chapter.} Strain ASC1058 was a kind gift from the Johan Elf lab. VS indicates these strains were created by Vanda Sunderlikova, technician in the Tans lab. ASC stands for AMOLF strain collection. Note that Y, C, G and R indicate yellow, cyan, green and red fluorescent reporters, respectively. The first two strains in this table were specifically created for experiments described in this chapter, and the third strain was requested specifically for work described in this chapter. The last two strains in the table were created earlier.}
    \label{table:ribostrains1}
\end{table}

%To be able to further explore the interaction of ribosomal expression with cellular dynamics, 
To be able to further explore the implications of ribosomal fluctuations, 
we produced additional strains.
%
Importantly, we were interested in the effect fluctuations in ribosomal expression would have on 
the single cell capacity to produce proteins.
%single cell protein production.
%
We therefore created two dual-label strains that both carried ribosomal reporter constructs, as well as constitutively expressed fluorescent reporters.
%
In the first strain, we fused the mCerulean fluorescent reporter sequence to the ribosomal RNA promoter (rrsa), and additionally inserted an mVenus sequence to the constitutive pn25 promoter. Both were chromosomally inserted.
%
The pn25-mVenus reporter served as a readout for fluctuations in protein production that likely affect all proteins that are produced in the cell. 
% a proxy for protein production in general.
%
In the second strain, we introduced an mCerulean sequence behind the chromosomal L19 ribosomal protein sequence, thus creating a strain that produces L19 ribosomal proteins that are fused to mCerulean fluorescent labels.
We also introduced the constitutive pn25-mVenus reporter to this strain.
%
These strains, the Prrna-C, pn25-Y strain and the L19-C, pn25-Y strain are listed in table \ref{table:ribostrains1}.

Furthermore, because the ribosome consists of so many ribosomal proteins, we wanted to be able to confirm 
potentially observed dynamics for one labeled ribosomal protein, also for other labeled ribosomal proteins.
%
% potential observations in a strain that had other ribosomal proteins labeled.
For this purpose, we requested a strain from the Elf lab which had both the L9 and S2 ribosomal proteins labeled by the mCherry and Venus fluorescent proteins, respectively \cite{Wallden2016}.
This strain was kindly supplied by the Elf lab, and is also listed in table \ref{table:ribostrains1} with the shorthand notation L9-R, S2-Y.

Thus, we expanded our ability to track ribosomal dynamics by 
expanding our collection of ribosomal reporters, 
%obtaining more strains with ribosomal reporters, and additionally 
and introduced a way to probe the effect of ribosomal fluctuations on protein expression by the introduction of the pn25 reporter.

%\red{
%pn25 both L and rna, refer to table;
%elf strain}

%\subsection{An attempt to create a limiting situation}
\subsection{Antibiotic shift experiments to create limiting situations}

%\subsubsection{The shift experiment}

To understand the dynamics of a process, it is often convenient to grasp what happens in extreme cases.
%
We therefore attempted to device an experiment which would lead to a situation where single cells that expressed more ribosomes than the population average would have a growth advantage, thus forcing a "limiting" situation.
% 
We did this by growing cells in minimal medium in our microfluidic device 1 for a few hours, and then switching to minimal medium supplemented with antibiotics (see chapter \ref{chapter:methods} and \ref{chapter:filarecovery} for a description of the microfluidics device).
%We did this by growing cells in minimal medium in our microfluidic device 1 (see chapters \ref{chapter:methods} and \ref{chapter:filarecovery} for a description of the microfluidics device).
%
We hypothesized that right after the medium has switched, cells in the population --- which at that point are calibrated for growth in minimal medium --- that happen to express more ribosomes have a growth advantage.
%
To analyse whether this was indeed the case, we made growth-concentration scatter plots for multiple points in time during the experiment.
%
We further analysed these measurements by calculating the slope and correlation of these scatter plots for each point in time.

Some data involving an antibiotic switch was already gathered
%One experiment like this was already conducted 
by Tans lab alumni Sarah Boulinea for the L31-R, gltA-G strain (see table \ref{table:ribostrains1}); 
we also ran our analysis on this dataset.
Also, two additional experiments were conducted, one with the L31-C strain where one microcolony was analyzed, 
and one with the rrsa-C, pn25-Y strain, where three microcolonies were analyzed.
%
Before looking at the scatter plots, we first look at the 
trends in fluor concentration for all these five datasets, which are shown in figure \ref{fig:ribo:scatter1}.
This figure shows that for the Prrna-C strain datasets, the signal goes up after the switch to medium with antibiotics, as is expected.
However, for both for the L31-R and L31-C datasets, the signal appears to be going down after the shift (though pre-shift fluctuations on the population average level of the L31-R data make it unclear what is exactly going on).
It is unclear why this happens, since ribosome inhibition should increase relative ribosomal demand \cite{You2013}.
%
We now turn to the scatter plots.
%As mentioned, we created growth-expression scatter plots for each time point within the dataset, for all of these datasets, and f
For brevity we only show the representative series of scatter plots that relates to the L31-C dataset, see figure \ref{fig:ribo:scatter1}. 
This figure shows that there is no clear change in correlation visible by eye after the switch to medium with antibiotics.
%
As mentioned however, we further quantified the data by calculating both the slope (least square fit by Matlab's \texttt{polyfit} function) and correlation coefficient for each of the scatter plots.
%
This is shown in figures \ref{fig:ribo:switch1}-\ref{fig:ribo:switch5}.
%
The L31-R dataset (figure \ref{fig:ribo:switch1}) arguably shows a small increase in correlation, but the correlation also decreases swiftly after that.
The L31-Y dataset (figure \ref{fig:ribo:switch2}) is difficult to interpret, 
likely the positive correlations before the switch are caused by chance as there are only a few cells in the microcolony at those points in time.
%
The first Prrsa-Y dataset (figure \ref{fig:ribo:switch3}) shows a change in signal right after the switch, but this is not seen in the two other Prrs-Y datasets (\ref{fig:ribo:switch4}-\ref{fig:ribo:switch5}).
%
%These figures show that there is no clear change in correlation observed after the switch.
In conclusion, it is difficult to interpret these datasets, as the different datasets show different trends.


%%%%%%%%%%%%%%%%%%%%%%%%%%%%%%%%%%%%%%%%%%%%%%%%%%%%%%%%%%%%%%%%%%%%%%%%%%%%%%%%%%%%%%%%%%%%%%%%%%%%%%%%%%%%%%%%%%%%%%%%%%%%%%%%%%%%%%%%%%%%%%%%
%%%%%%%%%%%%%%%%%%%%%%%%%%%%%%%%%%%%%%%%%%%%%%%%%%%%%%%%%%%%%%%%%%%%%%%%%%%%%%%%%%%%%%%%%%%%%%%%%%%%%%%%%%%%%%%%%%%%%%%%%%%%%%%%%%%%%%%%%%%%%%%%
% Switch experiment figures
%%%%%%%%%%%%%%%%%%%%%%%%%%%%%%%%%%%%%%%%%%%%%%%%%%%%%%%%%%%%%%%%%%%%%%%%%%%%%%%%%%%%%%%%%%%%%%%%%%%%%%%%%%%%%%%%%%%%%%%%%%%%%%%%%%%%%%%%%%%%%%%%

%%%%%%%%%%%%%%%%%%%% General trend of signals
\begin{figure}
    \centering    
    \includegraphics[width=0.3\textwidth]{pdf_2012-11-15_pos5_L31-mCherry_fluorTrend.pdf}
    \includegraphics[width=0.3\textwidth]{pdf_2016-02-17_pos2_L31-mCerulean_fluorTrend.pdf} \\
    \includegraphics[width=0.3\textwidth]{pdf_2016-09-20_pos1_prrsa-mCerulean_pn25-yfp_fluorTrend} % added
    \includegraphics[width=0.3\textwidth]{pdf_2016-09-20_pos2_prrsa-mCerulean_pn25-yfp_fluorTrend}
    \includegraphics[width=0.3\textwidth]{pdf_2016-09-20_pos3_prrsa-mCerulean_pn25-yfp_fluorTrend}            
    \caption{ 
        \textbf{Fluorescence intensity for switch from clean medium to medium supplemental with antibiotics.}
        Red dots show single cell obsvervations, the black line indicates the colony average.
        (Top left) L31-R strain. 
        (Top right) L31-C strain.
        (Bottom three) All are rrsa-C strains.      
    }
    \label{fig:ribo:fluorsignals}
\end{figure}


%%%%%%%%%%%%%%%%%%%% Scatter plots (only one shown for brevity).
\begin{figure}
    \centering    
    \includegraphics[width=1.0\textwidth]{pdf_2016-02-17_pos2_L31-mCerulean_clouds.pdf}
    %\includegraphics[width=0.8\textwidth]{pdf_2012-11-15_pos5_L31-mCherry_summaryPlot.pdf}
    \caption{ 
        \textbf{Scatter plots for switch from clean medium to medium supplemental with antibiotics.}
        %
        %        
    }
    \label{fig:ribo:scatter1}
\end{figure}

%%%%%%%%%%%%%%%%%%%% Summary plots for shift experiments
\begin{figure}
    \centering    
    %\includegraphics[width=1.0\textwidth]{pdf_2016-02-17_pos2_L31-mCerulean_clouds.pdf}
    \includegraphics[width=0.8\textwidth]{pdf_2012-11-15_pos5_L31-mCherry_summaryPlot.pdf}
    \caption{ 
        \textbf{Scatter plots for switch from clean medium to medium supplemental with antibiotics.}
        %
        %        
    }
    \label{fig:ribo:switch1}
\end{figure}
\begin{figure}
    \centering    
    %\includegraphics[width=1.0\textwidth]{pdf_2016-02-17_pos2_L31-mCerulean_clouds.pdf}
    \includegraphics[width=0.8\textwidth]{pdf_2016-02-17_pos2_L31-mCerulean_summaryPlot.pdf}
    \caption{ 
        \textbf{Scatter plots for switch from clean medium to medium supplemental with antibiotics.}
        %
        %        
    }
    \label{fig:ribo:switch2}
\end{figure}
\begin{figure}
    \centering    
    %\includegraphics[width=1.0\textwidth]{pdf_2016-02-17_pos2_L31-mCerulean_clouds.pdf}
    \includegraphics[width=0.8\textwidth]{pdf_2016-09-20_pos1_prrsa-mCerulean_pn25-yfp_summaryPlot.pdf}
    \caption{ 
        \textbf{Scatter plots for switch from clean medium to medium supplemental with antibiotics.}
        %
        %        
    }
    \label{fig:ribo:switch3}
\end{figure}
\begin{figure}
    \centering    
    %\includegraphics[width=1.0\textwidth]{pdf_2016-02-17_pos2_L31-mCerulean_clouds.pdf}
    \includegraphics[width=0.8\textwidth]{pdf_2016-09-20_pos2_prrsa-mCerulean_pn25-yfp_summaryPlot.pdf}
    \caption{ 
        \textbf{Scatter plots for switch from clean medium to medium supplemental with antibiotics.}
        %
        %        
    }
    \label{fig:ribo:switch4}
\end{figure}
\begin{figure}
    \centering    
    %\includegraphics[width=1.0\textwidth]{pdf_2016-02-17_pos2_L31-mCerulean_clouds.pdf}
    \includegraphics[width=0.8\textwidth]{pdf_2016-09-20_pos3_prrsa-mCerulean_pn25-yfp_summaryPlot.pdf}
    \caption{ 
        \textbf{Scatter plots for switch from clean medium to medium supplemental with antibiotics.}
        %
        %        
    }
    \label{fig:ribo:switch5}
\end{figure}

%%%%%%%%%%%%%%%%%%%%%%%%%%%%%%%%%%%%%%%%%%%%%%%%%%%%%%%%%%%%%%%%%%%%%%%%%%%%%%%%%%%%%%%%%%%%%%%%%%%%%%%%%%%%%%%%%%%%%%%%%%%%

\subsection{Higher throughput and additional cross-correlations}

\subsubsection{Many experiments in one}

One reason that it is hard to analyse the data from the antibiotic switch experiments using microfluidic device 1, 
is
that the size of the microcolony starts out small, which leads to a 
high variability of the observables during the first part of the experiment.
%
To produce a dataset with a constant large amount of cells, we performed additional antibiotic switch experiments in microfluidic device 2.
%
Conveniently, microfluidic device 2 allows for multi-day experiments, which allowed us 
%to increase the throughput of the experiments, and allowed us 
to also measure additional steady state expression-growth cross-correlation curves. % by prolonged multi-hour experiments.
%
Importantly, we performed these additional experiments with the new Prrna-C, pn25-Y and L19-C, pn25-Y strains, to investigate the effect of ribosomal fluctuations on protein production rates with the pn25-mVenus reporter.
%
%For example, table \ref{table:ribosomes:typicaldevice2experiment} shows the setup of a typical experiment.
An example of a typical 
%For example, the 
experimental design is given by the measurements we performed on the L19-R pn25-Y strain, which involved the following sequence of supplied medium: \textit{6 hours TY, 6 hours TY + TET, 2 hours TY, 2 hours TY + TET, 2 hours TY, 2 hours TY + TET, 2 hours TY, 6 hours M9, 6 hours M9 + TET, 2 hours M9, 2 hours M9 + TET, 2 hours M9, 2 hours M9 + TET}. (M9 indicates M9 minimal medium here, supplemented with lactose, uracil and tween20; see chapter \ref{chapter:filarecovery} for further description of used media). 
%
%By analyzing the last two hours of the 6hr sequences, we could investigate the steady state behavior of the population.
%The many switches would allow us to investigate the behavior in the case of switches. XXXXXX

Thus, to recapitulate, each experiment with microfluidic device 2 can potentially give (1) data on the ribosome-growth dynamics during antibiotic shifts, (2) data on ribosome-protein expression dynamics during antibiotic shifts, (3) additional CCs regarding ribosome-growth dynamics to characterize steady state relationships, and (4) CCs of previously uncharacterised ribosome-protein expression dynamics. 
Additionally, we could also analyse the response growth medium shifts and the response to medium shifts that occur at high frequency, but we will not focus on this here.


\subsubsection{Results of the experiments}

The large amount of data generated by microfluidic device 2 is both an advantage and a disadvantage.
%
We will not go into technical details here, but both the amount of data and also the nature of the growth in the wells present the computer analysis with new challenges.
%
We therefore only analysed parts of the data that was produced by these experiments at this moment.
%
We chose to first analyse the data where we assumed cells had reached steady state; and where we can perform cross-correlation analyses (i.e. analyses mentioned in points 3 and 4 in the previous paragraph).
Additional data recorded at the times of the switches exists but has not been fully analysed yet.
%This selection of the data did not include the parts were switches were performed, 
%but only the data were we assumed cells had reached steady state growth.
%We therefore here only present cross-correlation analysis.
%
We also note that 
additional manual corrections 
% (on top of those already performed) 
could perhaps further refine the analyses that we show here. 
%
For example, some of the tracked lineages show unrealistic fluctuations which might be due to artefacts in the computer analysis and might be removed or corrected (see supplemental figures \ref{fig:ribo:branchesXXXX}), and also the analysis now only takes into account a selection of the total amount of imaged cells, this selection could be extended to obtain more data.

%\subsubsection{\red{Expression-growth CCs of constitutive reporter are consistent with dilution mode}}
%\subsubsection{Results from ribosome and constitutive dual reporter strains}
\subsubsection{Results from strains with ribosome and constitutive reporters}

At any rate, figures \ref{fig:ribo:CCsEmuYpn25}-\ref{fig:ribo:CCsPmuYpn25Ribo} show data obtained from the Prrna-C, pn25-Y and L19-C, pn25-Y strains.
%
Most of these CCs show large error bars, and often fall inside the range of the negative control.
This indicates that more data is required to draw definitive conclusions.
Furthermore, figure \ref{fig:ribo:meansPn25R} shows that the concentration of ribosomal RNA reporters is quite low in comparison to other reporters. This might be due a weak ribosomal binding site, 
and further emphasizes caution is required when interpreting this data.
%
Nevertheless, figures \ref{fig:ribo:CCsEmuYpn25} and \ref{fig:ribo:CCsPmuYpn25} show the CCs calculated for the pn25 constitutive expression-growth relationship for growth in minimal medium, TY medium, and TY medium supplemented with sub-inhibitory concentration of the antibiotic tetracycline (0.5 $\upmu$M).
%
Most of the correlation-growth curves show negative correlation values for negative delays.
The production rate-growth curves do not show clear trends.
We saw earlier (see chapter \ref{chapter:CRP}) that constitutive reporters often show dilution mode dynamics.
The CCs we observe here could be consistent with that transmission mode.
%
Furthermore, figures \ref{fig:ribo:CCsEmuYRibo} and \ref{fig:ribo:CCsPmuYRibo} show data from the same strains, but show the CCs for the ribosomal reporters (rrna and L19).
The CC in panel A does not show a clear trend.
The CCs in panels B-D however, which show the behaviour of the rrna reporter for different conditions, all show behaviour that also might be consistent with dilution mode transmission of noise; negative concentration-growth at negative delays correlations are seen both for M9 and TY medium, and also for the condition where antibiotic was added to the medium.
%
It is striking that there are no noticeable differences in the rrna expression growth dynamics between different conditions.
%
One explanation might be that the rrna reporter might not capture all ribosomal RNA fluctuations, since it is by its nature not a translational fusion reporter such as the other ribosomal reporters.



The dual reporter strains with both ribosomal and pn25 constitutive reporters were constructed to allow us to not only study ribosome-growth dynamics, but also allow us to study the impact of ribosomal fluctuations and protein expression.
%
%We now turn to the question we can address with the ribosomal and constitutive reporter strains:
To understand whether ribosomes indeed have an effect on protein production, we look at the CCs between ribosomal concentration and constitutive reporter concentration, shown in figure \ref{fig:ribo:CCsEERiboPn25}, and CCs between ribosomal production rate and constitutive production rate, as shown in figure \ref{fig:ribo:CCsPPRiboPn25}.
%
The L19-C, pn25-Y concentration-concentration CC shows a very unclear pattern, 
but the other concentration-concentration CCs show a clear positive peak around $\tau = 0$ delay, indicating concerted fluctuations.
%
This is expected, since concerted fluctuations were observed earlier for a pair of two constitutive reporters \cite{Elowitz2002}, for groups of proteins \cite{Stewart-Ornstein2012} and in general is expected for any two proteins because of the existence of extrinsic noise \cite{Chalancon2012}.
%Thus, a positive correlations between expression is expected to be found for any two proteins in the cells, and also for our ribosomal and constitutive protein reporter pair.
%
However, here we look at two proteins of which one reports for ribosomal concentration, which might have an impact on protein production itself.
This might change the dynamics.
%
Additional features in the CC on top of the positive peak at 0 delay might tell us something about the ribosome-protein expression dynamics.
For example, more pronounced correlations at positive delays could indicate that transmission of fluctuations occurs from ribosomes to protein expression.
%
However, The Prrna-C, pn25-Y concentration correlation for M9 medium (figure \ref{fig:ribo:CCsEERiboPn25}.B) seems to show the opposite: there is a positive correlation at negative delays.
This implies that ribosome concentration fluctuations correlates with past protein concentration fluctuations.
%
The two reporters might have different maturation times, which could lead to artificial correlations negative delay.
%
Maturation times are however in the order of tens of minutes \cite{Iizuka2011, Walker2016t}, and a discrepancy between the two is thus not expected to cause such a big effect as observed here.
%
Hypothetically, % when ribosomes are not limiting to protein production,
fluctuations in protein abundance might positively regulate ribosomal production,
which might help the cell anticipate ribosomal demand;
though very speculative, 
this might explain the correlation at negative delays.

In TY medium (figure \ref{fig:ribo:CCsEERiboPn25}.C), the CC does not show a stronger correlation at negative delays.
%
One could speculate this is because ribosomes become more limiting here, thus shifting the balance from negative delays in M9 medium to positive delays in TY medium.
%
The CC in TY medium with antibiotics (figure \ref{fig:ribo:CCsEERiboPn25}.D) shows a strong background signal.
The second peak in correlation at negative delays might be an artefact of insufficient data.
We do not have further interpretations of this shape.

The production-production CCs (figure \ref{fig:ribo:CCsPPRiboPn25}) all appear to show peaks at 0 delay, that are rather symmetrical around the y-axis, and appear to not show additional features.
This is consistent with aforementioned concerted protein expression fluctuations.

%%%%%%%%%%%%%%%%%%%%%%%%%%%%%%%%%%%%%%%%%%%%%%%%%%%%%%%%%%%%%%%%%%%%%%%%%%%%%%%%%%%%%%%%%%%%%%%%%%%%%%%%%
% Dataset w/ pn25 reporters
%%%%%%%%%%%%%%%%%%%%%%%%%%%%%%%%%%%%%%%%%%%%%%%%%%%%%%%%%%%%%%%%%%%%%%%%%%%%%%%%%%%%%%%%%%%%%%%%%%%%%%%%%
%%%%%%%%%%%%%%%%%%%%%%%%%%%%%%%%%%%%%%%%%%%%%%%%%%%%%%%%%%%%%%%%%%%%%%%%%%%%%%%%%%%%%%%%%%%%%%%%%%%%%%%%%

%%%%%%%%%%%%%%%%%%%%%%%%%%%%%%%%%%%%%%%%%%%%%%%%%%%%%%%%%%%%%%%%%%%%%%%%%%%%%
% Overview figures of branches and means for all datasets
%%%%%%%%%%%%%%%%%%%%%%%%%%%%%%%%%%%%%%%%%%%%%%%%%%%%%%%%%%%%%%%%%%%%%%%%%%%%%

% Means for ASC968andASC976 %%%%%%%%%%%%%%%%%%%%%%%%%%%%%%%%%%%%%%%%%%%%%%%%%%%%%%%%%%%%%%%%%%%%%%%%%%%%%
\begin{figure}
    \centering
    \includegraphics[width=0.8\textwidth]{pdf_ASC968andASC976_overview_means.pdf}
    \caption{ 
        \textbf{Population mean values of different parameters measured in different strains and conditions.}
        %        (A) L9-R, S2-Y strain. Grown in in M9 minimal medium.
        %        (B) L9-R, S2-Y strain. Grown in in M9 minimal medium supplemented with antibiotics. 
        %        (A) L19-C, pn25-Y strain. Grown in M9 minimal medium.
        %        (B) Prrna-C, pn25-Y strain. Grown in M9 minimal medium.
        %        (C) Prrna-C, pn25-Y strain. Grown in TY medium.
        %        (D) Prrna-C, pn25-Y strain. Grown in TY medium supplemented with antibiotics.
        %
        %       A: giu_asc968_M9_steady, dY5_divAreaPx_cycCor_muP9_fitNew_atdY5_cycCor
        %        B: giu_asc1058_M9_steady, dY5_divAreaPx_cycCor_muP9_fitNew_atdY5_cycCor
        %        C: giu_asc1058_M9_steady_antibiotics, dY5_divAreaPx_cycCor_muP9_fitNew_atdY5_cycCor
        %        D: giu_asc976_TY_steady, dY5_divAreaPx_cycCor_muP9_fitNew_atdY5_cycCor
        %        E: giu_asc976_TY_steady_antibiotics, dY5_divAreaPx_cycCor_muP9_fitNew_atdY5_cycCor
        %        F: giu_asc976_M9_steadymw_asc976_M9_steady, dY5_divAreaPx_cycCor_muP9_fitNew_atdY5_cycCor
        %        
    }
    \label{fig:ribo:meansPn25R}
\end{figure}


%%%%%%%%%%%%%%%%%%%%%%%%%%%%%%%%%%%%%%%%%%%%%%%%%%%%%%%%%%%%%%%%%%%%%%%%%%%%%
% Constitutive reporters 
%%%%%%%%%%%%%%%%%%%%%%%%%%%%%%%%%%%%%%%%%%%%%%%%%%%%%%%%%%%%%%%%%%%%%%%%%%%%%

\begin{figure}
    \centering
    \includegraphics[width=0.8\textwidth]{pdf_ASC968andASC976_CCs_Y6_mean_cycCor_muP9_fitNew_cycCor.pdf}
    \caption{ 
        \textbf{Cross-correlations between concentration of constitutive reporter and growth.}
        (A) L19-C, pn25-Y strain. Grown in M9 minimal medium.
        (B) Prrna-C, pn25-Y strain. Grown in M9 minimal medium.
        (C) Prrna-C, pn25-Y strain. Grown in TY medium.
        (D) Prrna-C, pn25-Y strain. Grown in TY medium supplemented with antibiotics.
        %
        %        (A) giu_asc968_M9_steady						L19-mCerulean, 	 pn25-mVenus
        %        (B) giu_asc976_M9_steady mw_asc976_M9_steady	Prrna-mCerulean, pn25-mVenus
        %        (C) giu_asc976_TY_steady						Prrna-mCerulean, pn25-mVenus
        %        (D) giu_asc976_TY_steady_antibiotics			Prrna-mCerulean, pn25-mVenus
        %        
    }
    \label{fig:ribo:CCsEmuYpn25}
\end{figure}

\begin{figure}
    \centering
    \includegraphics[width=0.8\textwidth]{pdf_ASC968andASC976_CCs_dY5_divAreaPx_cycCor_muP9_fitNew_atdY5_cycCor.pdf}
    \caption{ 
        \textbf{Cross-correlations between production rate of constitutive reporter and growth.}
        (A) L19-C, pn25-Y strain. Grown in M9 minimal medium.
        (B) Prrna-C, pn25-Y strain. Grown in M9 minimal medium.
        (C) Prrna-C, pn25-Y strain. Grown in TY medium.
        (D) Prrna-C, pn25-Y strain. Grown in TY medium supplemented with antibiotics.
        %
        %        (A) giu_asc968_M9_steady						L19-mCerulean, 	 pn25-mVenus
        %        (B) giu_asc976_M9_steady mw_asc976_M9_steady	Prrna-mCerulean, pn25-mVenus
        %        (C) giu_asc976_TY_steady						Prrna-mCerulean, pn25-mVenus
        %        (D) giu_asc976_TY_steady_antibiotics			Prrna-mCerulean, pn25-mVenus
        %        
    }
    \label{fig:ribo:CCsPmuYpn25}
\end{figure}

%%%%%%%%%%%%%%%%%%%%%%%%%%%%%%%%%%%%%%%%%%%%%%%%%%%%%%%%%%%%%%%%%%%%%%%%%%%%%

%%%%%%%%%%%%%%%%%%%%%%%%%%%%%%%%%%%%%%%%%%%%%%%%%%%%%%%%%%%%%%%%%%%%%%%%%%%%%
% Ribosomal reporters
%%%%%%%%%%%%%%%%%%%%%%%%%%%%%%%%%%%%%%%%%%%%%%%%%%%%%%%%%%%%%%%%%%%%%%%%%%%%%

\begin{figure}
    \centering
    \includegraphics[width=0.8\textwidth]{pdf_ASC968andASC976_CCs_C6_mean_cycCor_muP9_fitNew_cycCor.pdf}
    \caption{ 
        \textbf{Cross-correlations between concentration of ribosomal reporter and growth.}
        (A) L19-C, pn25-Y strain. Grown in M9 minimal medium.
        (B) Prrna-C, pn25-Y strain. Grown in M9 minimal medium.
        (C) Prrna-C, pn25-Y strain. Grown in TY medium.
        (D) Prrna-C, pn25-Y strain. Grown in TY medium supplemented with antibiotics.
        %
        %        (A) giu_asc968_M9_steady						L19-mCerulean, 	 pn25-mVenus
        %        (B) giu_asc976_M9_steady mw_asc976_M9_steady	Prrna-mCerulean, pn25-mVenus
        %        (C) giu_asc976_TY_steady						Prrna-mCerulean, pn25-mVenus
        %        (D) giu_asc976_TY_steady_antibiotics			Prrna-mCerulean, pn25-mVenus
        %        
    }
    \label{fig:ribo:CCsEmuYRibo}
\end{figure}

\begin{figure}
    \centering
    \includegraphics[width=0.8\textwidth]{pdf_ASC968andASC976_CCs_dC5_divAreaPx_cycCor_muP9_fitNew_atdC5_cycCor.pdf}
    \caption{ 
        \textbf{Cross-correlations between production rate of ribosomal reporter and growth.}
        (A) L19-C, pn25-Y strain. Grown in M9 minimal medium.
        (B) Prrna-C, pn25-Y strain. Grown in M9 minimal medium.
        (C) Prrna-C, pn25-Y strain. Grown in TY medium.
        (D) Prrna-C, pn25-Y strain. Grown in TY medium supplemented with antibiotics.
        %
        %        (A) giu_asc968_M9_steady						L19-mCerulean, 	 pn25-mVenus
        %        (B) giu_asc976_M9_steady mw_asc976_M9_steady	Prrna-mCerulean, pn25-mVenus
        %        (C) giu_asc976_TY_steady						Prrna-mCerulean, pn25-mVenus
        %        (D) giu_asc976_TY_steady_antibiotics			Prrna-mCerulean, pn25-mVenus
        %        
    }
    \label{fig:ribo:CCsPmuYRibo}
\end{figure}


%%%%%%%%%%%%%%%%%%%%%%%%%%%%%%%%%%%%%%%%%%%%%%%%%%%%%%%%%%%%%%%%%%%%%%%%%%%%%
% Correlations between ribosomal expression and pn25 expression
%%%%%%%%%%%%%%%%%%%%%%%%%%%%%%%%%%%%%%%%%%%%%%%%%%%%%%%%%%%%%%%%%%%%%%%%%%%%%

\begin{figure}
    \centering
    \includegraphics[width=0.8\textwidth]{pdf_ASC968andASC976_CCs_C6_mean_cycCor_Y6_mean_cycCor.pdf}
    \caption{ 
        \textbf{Cross-correlations between concentration of ribosomal reporter and concentration of pn25 reporter.}
        (A) L19-C, pn25-Y strain. Grown in M9 minimal medium.
        (B) Prrna-C, pn25-Y strain. Grown in M9 minimal medium.
        (C) Prrna-C, pn25-Y strain. Grown in TY medium.
        (D) Prrna-C, pn25-Y strain. Grown in TY medium supplemented with antibiotics.
        %
        %        (A) giu_asc968_M9_steady						L19-mCerulean, 	 pn25-mVenus
        %        (B) giu_asc976_M9_steady mw_asc976_M9_steady	Prrna-mCerulean, pn25-mVenus
        %        (C) giu_asc976_TY_steady						Prrna-mCerulean, pn25-mVenus
        %        (D) giu_asc976_TY_steady_antibiotics			Prrna-mCerulean, pn25-mVenus
        %        
    }
    \label{fig:ribo:CCsEERiboPn25}
\end{figure}

\begin{figure}
    \centering
    \includegraphics[width=0.8\textwidth]{pdf_ASC968andASC976_CCs_dC5_divAreaPx_cycCor_dY5_divAreaPx_cycCor.pdf}
    \caption{ 
        \textbf{Cross-correlations between production rate of ribosomal reporter and production rate of pn25 reporter.}
        (A) L19-C, pn25-Y strain. Grown in M9 minimal medium.
        (B) Prrna-C, pn25-Y strain. Grown in M9 minimal medium.
        (C) Prrna-C, pn25-Y strain. Grown in TY medium.
        (D) Prrna-C, pn25-Y strain. Grown in TY medium supplemented with antibiotics.
        %
        %        (A) giu_asc968_M9_steady						L19-mCerulean, 	 pn25-mVenus
        %        (B) giu_asc976_M9_steady mw_asc976_M9_steady	Prrna-mCerulean, pn25-mVenus
        %        (C) giu_asc976_TY_steady						Prrna-mCerulean, pn25-mVenus
        %        (D) giu_asc976_TY_steady_antibiotics			Prrna-mCerulean, pn25-mVenus
        %        
    }
    \label{fig:ribo:CCsPPRiboPn25}
\end{figure}


%%%%%%%%%%%%%%%%%%%%%%%%%%%%%%%%%%%%%%%%%%%%%%%%%%%%%%%%%%%%%%%%%%%%%%%%%%%%%

%%%%%%%%%%%%%%%%%%%%%%%%%%%%%%%%%%%%%%%%%%%%%%%%%%%%%%%%%%%%%%%%%%%%%%%%%%%%%%%%%%%%%%%%%%%%%%%%%%%%%%%%%
% Dataset with Elf dual reporter
%%%%%%%%%%%%%%%%%%%%%%%%%%%%%%%%%%%%%%%%%%%%%%%%%%%%%%%%%%%%%%%%%%%%%%%%%%%%%%%%%%%%%%%%%%%%%%%%%%%%%%%%%
%%%%%%%%%%%%%%%%%%%%%%%%%%%%%%%%%%%%%%%%%%%%%%%%%%%%%%%%%%%%%%%%%%%%%%%%%%%%%%%%%%%%%%%%%%%%%%%%%%%%%%%%%

\subsubsection{Results from strains with two ribosomal reporters}

As mentioned, to get a full picture of ribosomal dynamics, we employed a strain which has labels on additional ribosomal proteins. This train has a red fluorescent reporter on the L9 ribosomal protein, and a yellow reporter on the S2 ribosomal protein.
%
We performed measurements on this strain that provide additional information on the single cell dynamics between ribosomal expression and growth.
Additionally, the presence of two ribosomal reporters in one strain allows us to investigate the interaction between them.
%
We analysed strain L9-R, S2-Y 
% We show analyses from experiments with this L9-R, S2-Y strain 
growing in steady state both in M9 minimal medium with and without sub-inhibitory doses of tetracycline (\red{0.5 $\upmu$M}).
%
To see whether these reporters behaved as expected, we first looked at the response of the reporter concentration to the presence of antibiotics.
Figure \ref{fig:ribo:meansRR} shows that due to the presence of antibiotics, the population average growth rate decreased by 12\%, whilst the concentration of ribosomal reporter more than doubles.
%
Such large increases of ribosomal concentration are consistent with earlier observations of growth in the presence of antibiotics, although in earlier observations this large increase was accompanied by a larger decrease in growth rate \cite{Hui2015}.
%
Figures \ref{fig:ribo:CCsEmuS2} and \ref{fig:ribo:CCsPmuS2} show CCs from the S2 r-protein expression-growth dynamics, and 
figures \ref{fig:ribo:CCsEmuL9} and \ref{fig:ribo:CCsPmuL9} show CCs from the L9 r-protein expression-growth dynamics.
%
Like the strains discussed earlier, the CCs showed high correlations in the negative control correlations.
This indicates that more experiments are required to make claims about this data.




Nevertheless, we can try to interpret the current data.
%
The left panel in figure \ref{fig:ribo:CCsEmuS2}.A shows the S2 reporter concentration-growth CC for cells growing in M9 minimal medium, which shows negative correlations consistent with dilution.
We ignore the right panel, given the pattern observed in the right panel of figure \ref{fig:ribo:CCsEERiboribo}.A, which will be discussed later.
%
Interestingly, figure \ref{fig:ribo:CCsEmuS2}.B, which shows S2 reporter concentration-growth CC for cells growing in the same medium supplemented with antibiotics, does not show these negative delays. 
This could be indicative of a situation where ribosomes are more limiting due to the antibiotic stress, and fluctuations transmit from ribosomal fluctuations to the growth rate of the cell.
%
Production rate-growth CCs of the S2 reporter, shown in figure \ref{fig:ribo:CCsPmuS2} are harder to interpret, but not inconsistent with the earlier observations and interpretation.


The negative concentration-growth CCs of the L9 reporter in M9 medium --- where we again only consider the the left panel in figure \ref{fig:ribo:CCsEmuL9}.A --- are consistent with the negative concentration growth CCs of the S2 reporter.  
The dynamics in M9 medium plus antibiotics again seem different, as negative correlations now manifest as positive delays (instead of at negative delays).
The negative correlations are however different from the positive correlations we saw for the S2 reporter in the same condition.
%
This indicates that either repetitions of these experiments might yield different results, or one of the labels does not appropriately reflect ribosomal concentration, or different ribosomal proteins might have their own dynamics.
%
Production-growth CCs of the L9 reporter (figure \ref{fig:ribo:CCsPmuL9}) are again hard to interpret, although the condition with antibiotic stress shows a distinct peak around zero delay.

Aside from expression-growth dynamics, we can additionally look at the interaction between expression of the two ribosomal proteins. 
% 
Firstly, the right panel in figure \ref{fig:ribo:CCsEERiboribo}.A shows the S2-L9 concentration-concentration CC; which shows a shape that we have not observed before and is inconsistent with the left panel in the same figure. Correlations seem to increase for longer delays. 
Taken together, we think data from this experiment should be disregarded; as indicated earlier for this condition.
(This implies not taking into account all the right panels in figures \ref{fig:ribo:CCsEmuS2}.A, \ref{fig:ribo:CCsPmuS2}.A, \ref{fig:ribo:CCsEmuL9}.A and \ref{fig:ribo:CCsPmuL9}.A.)
%
In any case, for other panels, since we expect both reporters to represent the ribosomal concentration we expect the CCs to show positive correlations that are symmetrical around zero delay.
%
The left panel in \ref{fig:ribo:CCsEERiboribo}.A shows the S2-L9 concentration-concentration CC for growth in M9 minimal medium, which is --- against expectations --- not symmetric.
The S2-L9 concentration-concentration CC for growth in M9 minimal medium supplemented with antibiotics is different, but also not symmetric (figure \ref{fig:ribo:CCsEERiboribo}.B).
%
This asymmetry is consistent with the earlier differences in expression growth CCs between the two labels.
Again, it might be that 
repetitions of these experiments might yield different results, or one of the labels does not appropriately reflect ribosomal concentration, or different ribosomal proteins might have their own dynamics.



Furthermore, the CCs between the production rates of the S2 and L9 labels (figure \ref{fig:ribo:CCsEERiboribo}) do seem symmetric in both the M9 minimal medium and M9 minimal medium with antibiotics, which is consistent with aforementioned expectations.
We do not observe further noteworthy features in these CCs.



%%%%%%%%%%%%%%%%%%%%%%%%%%%%%%%%%%%%%%%%%%%%%%%%%%%%%%%%%%%%%%%%%%%%%%%%%%%%%%%%%%%%%%%%%%%%%%%%
% Elf strain mean overview figure
%%%%%%%%%%%%%%%%%%%%%%%%%%%%%%%%%%%%%%%%%%%%%%%%%%%%%%%%%%%%%%%%%%%%%%%%%%%%%%%%%%%%%%%%%%%%%%%%

\begin{figure}
    \centering
    \includegraphics[width=0.8\textwidth]{pdf_riboASC1058_overview_means.pdf}
    \caption{ 
        \textbf{Population mean values of different parameters measured in different strains and conditions.}
        %        (A) L9-R, S2-Y strain. Grown in in M9 minimal medium.
        %        (B) L9-R, S2-Y strain. Grown in in M9 minimal medium supplemented with antibiotics. 
        %        (A) L19-C, pn25-Y strain. Grown in M9 minimal medium.
        %        (B) Prrna-C, pn25-Y strain. Grown in M9 minimal medium.
        %        (C) Prrna-C, pn25-Y strain. Grown in TY medium.
        %        (D) Prrna-C, pn25-Y strain. Grown in TY medium supplemented with antibiotics.
        %
        %       A: giu_asc968_M9_steady, dY5_divAreaPx_cycCor_muP9_fitNew_atdY5_cycCor
        %        B: giu_asc1058_M9_steady, dY5_divAreaPx_cycCor_muP9_fitNew_atdY5_cycCor
        %        C: giu_asc1058_M9_steady_antibiotics, dY5_divAreaPx_cycCor_muP9_fitNew_atdY5_cycCor
        %        D: giu_asc976_TY_steady, dY5_divAreaPx_cycCor_muP9_fitNew_atdY5_cycCor
        %        E: giu_asc976_TY_steady_antibiotics, dY5_divAreaPx_cycCor_muP9_fitNew_atdY5_cycCor
        %        F: giu_asc976_M9_steadymw_asc976_M9_steady, dY5_divAreaPx_cycCor_muP9_fitNew_atdY5_cycCor
        %        
    }
    \label{fig:ribo:meansRR}
\end{figure}

%%%%%%%%%%%%%%%%%%%%%%%%%%%%%%%%%%%%%%%%%%%%%%%%%%%%%%%%%%%%%%%%%%%%%%%%%%%%%
% S2-Y ribosomal reporter
%%%%%%%%%%%%%%%%%%%%%%%%%%%%%%%%%%%%%%%%%%%%%%%%%%%%%%%%%%%%%%%%%%%%%%%%%%%%%

\begin{figure}
    \centering
    \includegraphics[width=0.8\textwidth]{pdf_riboASC1058_CCs_Y6_mean_cycCor_muP9_fitNew_cycCor.pdf}
    \caption{ 
        \textbf{Cross-correlations between the concentration of the S2 ribosomal protein and growth.}
        (A) L9-R, S2-Y strain. Grown in in M9 minimal medium.
        (B) L9-R, S2-Y strain. Grown in in M9 minimal medium supplemented with antibiotics. 
        %
%        (A) A: giu_asc1058_M9_steady, dR5_divAreaPx_cycCor_muP9_fitNew_atdR5_cycCor
%        (B) B: giu_asc1058_M9_steady_antibiotics, dR5_divAreaPx_cycCor_muP9_fitNew_atdR5_cycCor
    }
    \label{fig:ribo:CCsEmuS2}
\end{figure}

\begin{figure}
    \centering
    \includegraphics[width=0.8\textwidth]{pdf_riboASC1058_CCs_dY5_divAreaPx_cycCor_muP9_fitNew_atdY5_cycCor.pdf}
    \caption{ 
        \textbf{Cross-correlations between the production rate of the S2 ribosomal protein and growth.}
        (A) L9-R, S2-Y strain. Grown in in M9 minimal medium.
        (B) L9-R, S2-Y strain. Grown in in M9 minimal medium supplemented with antibiotics. 
%
%        (A) A: giu_asc1058_M9_steady, dR5_divAreaPx_cycCor_muP9_fitNew_atdR5_cycCor
%        (B) B: giu_asc1058_M9_steady_antibiotics, dR5_divAreaPx_cycCor_muP9_fitNew_atdR5_cycCor
        %        
    }
    \label{fig:ribo:CCsPmuS2}
\end{figure}

%%%%%%%%%%%%%%%%%%%%%%%%%%%%%%%%%%%%%%%%%%%%%%%%%%%%%%%%%%%%%%%%%%%%%%%%%%%%%



%%%%%%%%%%%%%%%%%%%%%%%%%%%%%%%%%%%%%%%%%%%%%%%%%%%%%%%%%%%%%%%%%%%%%%%%%%%%%
% L9-R ribosomal reporter
%%%%%%%%%%%%%%%%%%%%%%%%%%%%%%%%%%%%%%%%%%%%%%%%%%%%%%%%%%%%%%%%%%%%%%%%%%%%%

\begin{figure}
    \centering
    \includegraphics[width=0.8\textwidth]{pdf_riboASC1058_CCs_R6_mean_cycCor_muP9_fitNew_cycCor.pdf}
    \caption{ 
        \textbf{Cross-correlations between the concentration of the L9 ribosomal protein and growth.}
        (A) L9-R, S2-Y strain. Grown in in M9 minimal medium.
        (B) L9-R, S2-Y strain. Grown in in M9 minimal medium supplemented with antibiotics. 
%
%        (A) A: giu_asc1058_M9_steady, dR5_divAreaPx_cycCor_muP9_fitNew_atdR5_cycCor
%        (B) B: giu_asc1058_M9_steady_antibiotics, dR5_divAreaPx_cycCor_muP9_fitNew_atdR5_cycCor
        %        
    }
    \label{fig:ribo:CCsEmuL9}
\end{figure}

\begin{figure}
    \centering
    \includegraphics[width=0.8\textwidth]{pdf_riboASC1058_CCs_dR5_divAreaPx_cycCor_muP9_fitNew_atdR5_cycCor.pdf}
    \caption{ 
        \textbf{Cross-correlations between the production rate of the L9 ribosomal protein and growth.}
        (A) L9-R, S2-Y strain. Grown in in M9 minimal medium.
        (B) L9-R, S2-Y strain. Grown in in M9 minimal medium supplemented with antibiotics. 
%
%        (A) A: giu_asc1058_M9_steady, dR5_divAreaPx_cycCor_muP9_fitNew_atdR5_cycCor
%        (B) B: giu_asc1058_M9_steady_antibiotics, dR5_divAreaPx_cycCor_muP9_fitNew_atdR5_cycCor
        %        
    }
    \label{fig:ribo:CCsPmuL9}
\end{figure}

%%%%%%%%%%%%%%%%%%%%%%%%%%%%%%%%%%%%%%%%%%%%%%%%%%%%%%%%%%%%%%%%%%%%%%%%%%%%%

%%%%%%%%%%%%%%%%%%%%%%%%%%%%%%%%%%%%%%%%%%%%%%%%%%%%%%%%%%%%%%%%%%%%%%%%%%%%%
% Correlations between ribosomal expression r-protein1 and r-protein2
%%%%%%%%%%%%%%%%%%%%%%%%%%%%%%%%%%%%%%%%%%%%%%%%%%%%%%%%%%%%%%%%%%%%%%%%%%%%%

\begin{figure}
    \centering
    \includegraphics[width=0.8\textwidth]{pdf_riboASC1058_CCs_Y6_mean_cycCor_R6_mean_cycCor.pdf}
    \caption{ 
        \textbf{Cross-correlations between concentrations of two ribosomal proteins.}
        (A) L9-R, S2-Y strain. Grown in in M9 minimal medium.
        (B) L9-R, S2-Y strain. Grown in in M9 minimal medium supplemented with antibiotics.
        %        
    }
    \label{fig:ribo:CCsEERiboribo}
\end{figure}

\begin{figure}
    \centering
    \includegraphics[width=0.8\textwidth]{pdf_riboASC1058_CCs_dY5_divAreaPx_cycCor_dR5_divAreaPx_cycCor.pdf}
    \caption{ 
        \textbf{Cross-correlations between production rates of two ribosomal proteins.}
        (A) L9-R, S2-Y strain. Grown in in M9 minimal medium.
        (B) L9-R, S2-Y strain. Grown in in M9 minimal medium supplemented with antibiotics.
        %        
    }
    \label{fig:ribo:CCsPPRiboribo}
\end{figure}


%%%%%%%%%%%%%%%%%%%%%%%%%%%%%%%%%%%%%%%%%%%%%%%%%%%%%%%%%%%%%%%%%%%%%%%%%%%%%






    






			

\begin{table}                       
\begin{tabular}{ l l l l }
    \centering
    r-protein & Gene name & Operon & Fluorescent label \\ 
    \hline
    L31 & rpmE & none & Cerulean, mCherry \\ 
    L19 & rplS & rpsP-rimM-trmD-rplS & Cerulean, mCherry \\ 
    L9 & rplI & rpsF-priB-rpsR-rplI & mCherry \\ 
    S2 & rpsB & ttf-rpsB-tsf & Venus\\ 

\end{tabular}
    \caption{Labeld ribosomal proteins in the Tans lab. 
    %    Ribosomal proteins that we have labeled.
    } \label{tab:ribolabeledprots}
\end{table}



\section{Discussion and conclusion}

% PART 1
%(-) summarize conclusions
%(-) compare conclusions between different ribosomal reporters
%(.) we need C-p correlations to check effect ribo flucs on production of proteins..

\subsection{Antibiotic shift experiments on device 1}

In conclusion, we have used different single cell approaches in an attempt to further understand the ribosome.
%
We first performed an experiment using our microfluidic device 1 to subject cells to a switch from growth in minimal medium to growth in minimal medium with antibiotics.
Based on extensive previous work, we expected that cells needed a higher concentration of ribosomes in the second condition (see e.g. ref. \cite{You2013}), and thus that single cells that expressed more ribosomes relative to the population average right before the switch would benefit shortly after the switch.
%
If we would be able to observe single cell growth advantages, this would imply limiting behaviour of ribosomes in this scenario and serve as positive control and point of reference for other experiments. 
%
The results, to our surprise, showed that some of our reporters (specifically, for the L31 ribosomal protein) did not displayed the expected increased concentration during exposure to antibiotics.
%
This was also observed before \cite{Walker2016t}, and it
raises the question how representative the L31 reporter is for the ribosomal concentration.
%
A second reporter we used, a ribosomal RNA promoter fused to the cyan fluorescent reporter protein, did show the expected increase of ribosomal concentration after the switch to medium supplemented with antibiotics.
%
The results from these experiments regarding single cell growth advantage however were inconclusive, they neither showed a clear absence nor a clear presence of an increased correlation after the switch.

\subsection{Single cell ribosome expression and growth}

To follow up on these experiments, we performed experiments using microfluidic device 2, which allows for measurements that are longer, thus allow for more switches, and enable tracking more cells. 
%
This set of experiments would have provided additional information on the ribosomal behavior during switches from medium that requires less ribosomal expression to medium that requires more ribosomal expression, such as medium with sub-inhibitory concentrations antibiotics.
Additionally, the long time lapse capabilities of the device would allow us to record data in steady state.
On top of this, we used newly constructed strains in these experiments that allowed us to measure the effect of ribosomal fluctuations on single cell protein expression, as the strains also contained a constitutive pn25 reporter construct.
%
However, The analysis of data in this device presented additional challenges for the analysis.
%The large amount of data, different cellular movements during growth in the wells and image features of cells in this device are slightly different in this device, presenting additional challenges during the analysis.
%
We therefore only analysed a subselection of the steady state data that was acquired during the experiments performed with microfluidic device 2.

On top of this, the data that we did analyse showed a very high background signal, which indicated that more experiments are needed to draw definitive conclusions.
%
Nevertheless, we made several observations regarding the interaction between single cell ribosomal fluctuations and single cell growth rate fluctuations.
%
Firstly, in experiments involving the ribosomal RNA reporter, we saw that cross-correlation analysis in most conditions could be consistent with dilution mode transmission of noise.
This would be consistent with the idea that ribosomal RNA does not affect growth rate, but growth rate fluctuations instead result in ribosomal fluctuations.
% > REMEMBER TO LATER POINT OUT THAT RRNA MIGHT NOT BE REPORTED FOR CORRECTLY
%
Secondly, in experiments in M9 minimal medium involving labeled ribosomal proteins (L9 and S2), we also saw cross-correlations that might be consistent with dilution mode dynamics.
However, when this strain grew on M9 minimal medium supplemented with antibiotics, cross-correlations were less consistent with dilutions mode dynamics.
Hence, contrary to what the r-RNA data indicated, this data suggested that ribosomal fluctuations could become more consequential for the cell due to translational stress.

Similar experiments were described earlier by Noreen Walker \cite{Walker2016t}, 
we briefly summarized her experimental observations in section \ref{section:ribos:walker}.
%
Observations by Walker on a strain with a ribosomal RNA reporter in medium M9 medium supplemented with acetate (our lactose-supplemented M9 sustains a higher growth rate) or defined rich medium did not show dilution mode behaviour, which could mean ribosomal RNA shows different dynamics in different media.
%
Some of the experiments involving L19 and L31 reporters that she used showed dilution mode behavior in M9 minimal medium, similar to our observations on the labeled S2 and L9 ribosomal proteins in minimal medium, but some of her experiments did not. (This depended on the reporter involved, mCherry or mCerulean, respectively.)
%
Walker also performed CC analyses of steady state growth in rich medium supplemented in antibiotics, which resulted only in subtle changes in the dynamics for an labelled L31 ribosomal protein and a ribosomal RNA promoter reporter construct.
It is unclear how these results in different medium connect to our results.

\subsection{Single cell ribosome expression and protein expression}

In this work, we performed experiments on strains that carried both a reporter for ribosomal expression, and a constitutive reporter.
This latter reporter was intended to probe the effect of ribosomal expression fluctuations on the cells ability to produce proteins.
%
We observed positive correlations in the CCs between the concentrations of these two reporters. 
%
The positive correlations were however hard to interpret since 
%
%The positive correlations in the CCs between the concentrations of these two reporters were hard to interpret since 
they are expected between the expression of any two proteins, 
and thus do not necessarily imply that there is an effect of ribosomal expression fluctuations on protein production in the cell.
%
We did observe additional features in the CCs, 
these depended on the medium in which the cells were growing.
%
In most media, these features were not consistent with an effect of ribosomal fluctuations on protein production,
except in TY.
%
In TY medium we observed additional correlations between constitutive protein expression and past ribosome expression,
which is consistent with an effect of ribosomal fluctuations on protein production rates.
%
TY medium sustains the fastest bacterial growth rate, 
which could explain an increased sensitivity to ribosomal fluctuations.


To shed further light on these matters, one could look at the cross-correlation between
the concentration of the ribosomal reporter and the production of the constitutive reporter.
%
This would require a rather straightforward extension of our analysis scripts.


\subsection{The ribosome is a complex structure}

% PART 2
% measuring the ribosome is not simple ... 
%(a) ribosomes are big and it is difficult to pick exactly what to label
%(b) it is difficult to label RNA properly
%(z) mention the order in which ribosomes are formed \cite{Chen2013} 
% 		<---> perhaps not required for my story to mention this? Chen2013 was cited anyways, which should be the case

We have discussed many experiments, which employed different ribosomal labels and probed dynamics in different conditions.
%
Different conditions sometimes showed different results.
%
Often however, results also depended on which ribosomal protein was tracked.
%
One explanation for that observation is that ribosomal subunits have their own dynamics.
%
Though the ribosome is often viewed as a complex that has a fixed composition, 
recent discoveries in eukaryotes show that this might actually not be the case \cite{Preiss2016, Slavov2015}.
%
Purification and mass-spectrometry of ribosomes in yeast and mouse embryonic stem cells showed that 
pools of ribosomes with different ribosomal protein composition exist, which could also be linked to functional properties \cite{Slavov2015}.
%
This raised the suggestion that the cell can modulate expression of ribosomal proteins separately, 
to achieve specific regulatory goals.
%
It is not infeasible this is also the case in \ecoli, and this would explain why experiments tracking different ribosomal proteins show different dynamics.
%
This potential added layer of regulation makes it more difficult to understand the effect of ribosomal concentration fluctuations.
Coincidentally, for eukaryotes, it has been observed that especially ribosomal surface proteins vary -- which are also convenient targets for labelling.
In fact, the L9, S2 and L19 proteins that we  labelled are also on the surface of the ribosome, see figure \ref{fig:ribo:labelsPicGiulia}.
Also the L31 protein that we labelled is only loosely attached to the ribosome \cite{Walker2016t}.
%
Consistent with the view that the ribosomal proteins are not expressed equally for functional reasons, 
not all operons from which the ribosomal proteins are expressed are regulated by exactly the same regulators \cite{Keseler2017}.
%
Additionally, some ribosomal subunits are observed to be essential, whilst others are not (i.e. some respective null mutants are viable, whilst others are not).
This further illustrates that different ribosomal proteins might have different functional roles.
%
From the proteins that we labelled, L9 and L31 are non-essential, and S2 and L19 are essential.
%
This suggests that it could be that 
single cell fluctuations in different ribosomal proteins might thus have a different effects on growth rates.
%
This is consisent with our observations that the shape of CCs sometimes depended on which ribosomal protein was labelled.
%
Understanding ribosomal dynamics based on labelling ribosomal proteins might thus prove a challenging task.

\begin{figure}
    \centering
    \includegraphics[width=0.70\textwidth]{riboLabelsGiuliaCombined.png}
    % \includegraphics[width=0.49\textwidth]{ribosomes_L19.png}
    \caption{ 
        \textbf{Some of the ribosomal proteins that were labeled for experiments in this chapter.}
        The structure of the ribosome is displayed twice in grey, with on the left the L19 ribosomal protein highlighted in red, and on the right the L9 and S2 ribosomal subunits highlighted in red and yellow, respectively.
        This image is based on x-ray crystallography (PDB ID: 4v4q, \cite{Schuwirth2005}). Pdb files downloaded from \texttt{www.rcsb.org} \cite{Berman2000} 
        visualized with UCSF Chimera (version 1.11.2, build 41376, \cite{pettersen2004}). These two visualizations were made by Giulia Bergamaschi.
        %        Giulia's text with this figure:
        %         Label location on the ribosomes; the fluorescent reporters were attached to the colored
        %         proteins. Left: labelling in strain asc1058: the L9 (red) protein has been tagged with mCherry in the
        %         large subunit, while S2 (yellow) has been labelled with mVenus in the small one. Right: labelled
        %         protein L19 (mCerulean) in strain asc968 in blue (created with UCSF Chimera visualisation software,
        %         www.cgl.ucsf.edu/chimera).
        %        
    }
    \label{fig:ribo:labelsPicGiulia}
\end{figure}


\subsection{Making ribosomes limiting}

% PART 3
%(c) when we "stress" the ribosomes a new equilibrium sets in
%    (c) point out that ribosome might adapt itself to circumstances by regulation, and so it is logical that dynamic fluctuations have similar effects in different conditions
%(d) would be nice to be able to titrate the ribosomes as we did with lac --> introduce new constructs
%    (-) either key protein
%    (-) or rip out whole r-RNA parts and fill up again

%> making limiting is desired
%> just stressing is probably not useful
%  ^>  new equilibrium
%> instead perhaps rna plasmid construct

As mentioned in the results section, it can sometimes be very useful to create an extreme situation to better understand the dynamics of a system.
%
Thus, creating a situation where ribosomes are limiting in single cells, i.e. a situation in which single cell growth rates correlate strongly with ribosomal expression, might allow us to better understand ribosomal dynamics.
%
This topic is also discussed in the previously mentioned thesis by Walker \cite{Walker2016t}. 
%
We think it is not straightforward to create such a situation.
%
For example, whilst it might be a sensible idea that adding sub-inhibitory concentrations of antibiotics to the cellular growth medium might cause translational stress and 
thus more limiting behaviour of ribosomes,
the addition of antibiotics to growth medium was observed to have limited consequences in experiments.
%
% This could be explained when 
One explanation for this 
might be that a population of cells reacts to an antibiotic disturbance by altering its ribosomal expression level 
in such a way that local deviations from the newly established level have similar effects as deviations from the original expression level.
%This might be explained by a scenario where the population reacts to an antibiotic disturbance by altering its ribosomal expression level to cope with the translational stress,
%but where deviations from that new expression level show similar dynamics as deviations from the original situation with the lower ribosomal expression level
% in the original situation.
How this might work is depicted in figure \ref{fig:ribo:optimumCartoon}.
%
Different stressors instead of antibiotics might be used to put the cell under translational duress.
%
To create an alternative situation where ribosomal capacity is challenged, 
we constructed strain ASC1088 (see table \ref{table:ribo:strains2}) which carries a plasmid that expresses the mCherry protein at a high expression level.
%
Similar to the antibiotic treatment, protein over-expression might also have put a large burden on ribosomal capacity, thus increasing the effect ribosomal fluctuations have on growth rate (see Walker's thesis for a longer discussion on this topic \cite{Walker2016t}).
%
However, 
as we suspected that experiments with over-expression might yield similar results as the antibiotic experiments, 
we abandoned experiments with this strain.

% Xx future experiments could overcome xx.
%There might be ideas for future experiments to force ribosomes to become limiting.
Future experiments might be devised in which ribosomes are forced to become limiting.
%
As also mentioned by Walker \cite{Walker2016t}, previously constructed \textit{rrs} null mutants \cite{Condon1993, Condon1995, Quan2015, Bollenbach2009} might provide a venue towards this aim.
%
Given the role of the ribosomal RNA as backbone for the ribosome, shortages in ribosomal RNA content might have profound effects on the cellular growth rate.
%
Ribosomal RNA is expressed from multiple partially redundant operons, as shown in figure \ref{fig:ribo:tabledeltarrn}. 
%
% The null mutants previsouly created by 
Bollenbach et al. \cite{Bollenbach2009} created several null mutants which had more and more of these operons removed, as also shown in figure \ref{fig:ribo:tabledeltarrn}. 
%

\subsection{TODO}

> just explain idea briefly
> move details to small supplement (use word doc)

COMPLICATIONS:
Maeda2015 shows different expression strengths of operons
Quan2015 shows especially tRNA might be limiting, and also shows growth rate defect is limited
Quan2013 already published a plasmid that expresses the rrnB operon
Kaczanowska2007 provides a detailed map of the rrnB operon

\begin{figure}
    \centering
    \includegraphics[width=1.0\textwidth]{optimum_figure.pdf}
    % \includegraphics[width=0.49\textwidth]{ribosomes_L19.png}
    \caption{ 
        \textbf{Cartoon of how ribosomal expression might affect growth rate.}
        Ribosomal expression in single cell deviates from the population average. 
        Both the average and single cell expression levels might not lie exactly at the optimum.
        Low expression might lead to slower growth because ribosomes become limiting.
        High expression might decrease growth rate because resources to produce these ribosomes are drawn from other essential cellular processes.
        Panel (a) shows how the expression-growth relationships in single cells might relate between cells not exposed to antibiotics (black line) and cells exposed to antibiotics (red line). Panel (b) shows that these relationships might be similar if we consider only the deviations from the population averages (which are here assumed to coincide with optimal expression levels). $R$ and $\mu$ respectively indicate ribosomal expression and growth rate, $\Delta{R}$ and $\Delta{\mu}$ indicate mean-subtracted respective values.%these respective values after the average value has been subtracted.
    }
    \label{fig:ribo:optimumCartoon}
\end{figure}

\begin{figure}
    \centering
    \includegraphics[width=1.0\textwidth]{tableRrna.png}
    \caption{ 
        \textbf{Table showing the \textit{rrn} operons in \ecoli with cartoon overlay indicating different \textit{rrn} null mutant strains created by Bollenbach et al..}
        %Information on operon structure was obtained from ref. \cite{Keseler2017}, \textit{rrn} null mutants where published in ref. \cite{Bollenbach2009}.
        Ribosomal RNA in \ecoli is expressed from multiple partially overlapping operons. 
        Operons contain different ribosomal parts (and tRNA molecules). 
        \textit{rrs} encodes 16S rRNA, \textit{rrl} encodes 23S rRNA, and \textit{rrf} encodes 5S rRNA.
        Other abbreviations indicate respective gene encodes a tRNA molecule.
        This is also indicated in the second line of the "ribosomal parts" column. 
        Symbols indicate equivalence between the operons.
        The last column indicates by regulators of the operons, symbols in brackets indicate evidence is not definitive.
        (This information was obtained from the online database Ecocyc \cite{Keseler2017}).
        The overlayed crosses and circles indicate the null mutants that were constructed by Bollenbach et al. \cite{Bollenbach2009}.
        Each column of crosses corresponds to one strain, a cross indicates that the respective strain lacks the operon indicated. A circle indicates additional tRNA was expressed from a plasmid.
        These strains were also referred to as $\Delta$1-6, where the number corresponds to the number of operons that were removed.
    }
    \label{fig:ribo:tabledeltarrn}
\end{figure}



\subsection{Structure}
These observations emphasize the complexity of the ribosomal dynamics.
%
Given the central role of the ribosome in the cell, and the extensive biochemical structure with many proteins and RNA molecules involved, 
this complexity is perhaps not surprising.
%




% DISCARDED conclusions
% (d) take the gel pad as reference and point out that it would be nice to lower negative controls to the same level
% (b) would be nice to re-run some of noreen's analyses with the script that creates the controls.




\section{Acknowledgements}

%I am indebted to Noreen Walker and Giulia Bergamaschi regarding this chapter.
%
I am thankful to Noreen Walker, who fostered the ribosome project for 4 years, for her collaboration and input regarding these additional experiments.
%
Much of the data shown in this chapter were taken by Giulia Bergamaschi, master student in the Tans lab at that time. I am thankful for 
the mountains of work she performed in only three months time.
%her hard work. 

Molecular graphics and analyses 
% addition MW
of crystal structures
%
were performed with the UCSF Chimera package (production version 1.11.2, build 41376). Chimera is developed by the Resource for Biocomputing, Visualization, and Informatics at the University of California, San Francisco (supported by NIGMS P41-GM103311). 




%%%%%%%%%%%%%%%%%%%%%%%%%%%%%%%%%%%%%%%%%%%%%%%%%%%%%%%%%%%%%%%%%%%%%%%%%%%%%%%%%%%%%%%%%%%%%%%%%%
% Supplementary stuff
%%%%%%%%%%%%%%%%%%%%%%%%%%%%%%%%%%%%%%%%%%%%%%%%%%%%%%%%%%%%%%%%%%%%%%%%%%%%%%%%%%%%%%%%%%%%%%%%%%

\FloatBarrier
\clearpage

\section{Supplemental figures and tables}

\begin{table}[h]
    \begin{tabularx}{\textwidth}{llXl}
        %
        \textbf{ASC number}	& \textbf{Shorthand} & \textbf{Description} & \textbf{Source}		\\
        \hline
        %      									
        ASC656	& L31-R & L31-mCherry-kanR (no linker).	(Kanamycin resistant.) & NW, VS \\
        ASC657	& L19-R & L19-mCherry-kanR (no linker).	(Kanamycin resistant.) & NW, VS \\
        ASC680	& L31-R, Prrn-G & L31-mCherry-kanR (no linker), Δ(cheZ)::Prrn-GFP-catR. (Kanamycin and chloramphenicol resistant.)	& NW, VS \\
        ASC779	& Prrn-G& Δ(cheZ)::Prrn-GFP, rrsa promoter. (No resistance.) &	NW, VS \\
        
        \hline										
        & L19-C & 	L19-mCerulean L19-gc-mCerulean-kanR (GC linker)	(Kanamycin resistance.)	& VS \\
        & Prrn-C & 	Δ(cheZ)::Prrn-mCerulean-kanR	(Kanamycin resistance)	& VS \\
        \hline
        ASC1088 & mCherry+
        % L19-C, pn25-Y, p-pn25-R 
        &	ASC968 + high copy plasmid w. pn25-mCherry (Resistances: kanamycin, chloramphenicol, ampicilin.) & VS \\
        \hline
    \end{tabularx}
    \caption{\textbf{Strains used in previous work and miscellaneous strains.} NW indicates these strains were used in Noreen Walker's thesis \cite{Walker2016t}. VS indicates these strains are created by Vanda Sunderlikova.}
    \label{table:ribo:strains2}
\end{table}



% Branches for ASC968andASC976 %%%%%%%%%%%%%%%%%%%%%%%%%%%%%%%%%%%%%%%%%%%%%%%%%%%%%%%%%%%%%%%%%%%%%%%%%%%%%

\begin{figure}
    \centering
    \includegraphics[width=0.8\textwidth]{pdf_ASC968andASC976_branches_muP9_fitNew_cycCor.pdf}
    \caption{ 
        \textbf{Growth of single cells of the different strain populations in the different conditions.}
        (A) L9-R, S2-Y strain. Grown in in M9 minimal medium.
        (B) L9-R, S2-Y strain. Grown in in M9 minimal medium supplemented with antibiotics. 
        (A) L19-C, pn25-Y strain. Grown in M9 minimal medium.
        (B) Prrna-C, pn25-Y strain. Grown in M9 minimal medium.
        (C) Prrna-C, pn25-Y strain. Grown in TY medium.
        (D) Prrna-C, pn25-Y strain. Grown in TY medium supplemented with antibiotics.
        %
        %       A: giu_asc968_M9_steady, dY5_divAreaPx_cycCor_muP9_fitNew_atdY5_cycCor
        %        B: giu_asc1058_M9_steady, dY5_divAreaPx_cycCor_muP9_fitNew_atdY5_cycCor
        %        C: giu_asc1058_M9_steady_antibiotics, dY5_divAreaPx_cycCor_muP9_fitNew_atdY5_cycCor
        %        D: giu_asc976_TY_steady, dY5_divAreaPx_cycCor_muP9_fitNew_atdY5_cycCor
        %        E: giu_asc976_TY_steady_antibiotics, dY5_divAreaPx_cycCor_muP9_fitNew_atdY5_cycCor
        %        F: giu_asc976_M9_steadymw_asc976_M9_steady, dY5_divAreaPx_cycCor_muP9_fitNew_atdY5_cycCor
        %        
    }
    \label{fig:ribo:branchesASC968andASC976}
\end{figure}

% Branches for ASC1058 %%%%%%%%%%%%%%%%%%%%%%%%%%%%%%%%%%%%%%%%%%%%%%%%%%%%%%%%%%%%%%%%%%%%%%%%%%%%%


\begin{figure}
    \centering
    \includegraphics[width=0.8\textwidth]{pdf_riboASC1058_branches_muP9_fitNew_cycCor.pdf}
    \caption{ 
        \textbf{Growth of single cells of the different strain populations in the different conditions.}
        (A) L9-R, S2-Y strain. Grown in in M9 minimal medium.
        (B) L9-R, S2-Y strain. Grown in in M9 minimal medium supplemented with antibiotics. 
        (A) L19-C, pn25-Y strain. Grown in M9 minimal medium.
        (B) Prrna-C, pn25-Y strain. Grown in M9 minimal medium.
        (C) Prrna-C, pn25-Y strain. Grown in TY medium.
        (D) Prrna-C, pn25-Y strain. Grown in TY medium supplemented with antibiotics.
        %
        %       A: giu_asc968_M9_steady, dY5_divAreaPx_cycCor_muP9_fitNew_atdY5_cycCor
        %        B: giu_asc1058_M9_steady, dY5_divAreaPx_cycCor_muP9_fitNew_atdY5_cycCor
        %        C: giu_asc1058_M9_steady_antibiotics, dY5_divAreaPx_cycCor_muP9_fitNew_atdY5_cycCor
        %        D: giu_asc976_TY_steady, dY5_divAreaPx_cycCor_muP9_fitNew_atdY5_cycCor
        %        E: giu_asc976_TY_steady_antibiotics, dY5_divAreaPx_cycCor_muP9_fitNew_atdY5_cycCor
        %        F: giu_asc976_M9_steadymw_asc976_M9_steady, dY5_divAreaPx_cycCor_muP9_fitNew_atdY5_cycCor
        %        
    }
    \label{fig:ribo:branchesASC1058}
\end{figure}


%%%%%%%%%%%%%%%%%%%%%%%%%%%%%%%%%%%%%%%%%%%%%%%%%%%%%%%%%%%%%%%%%%%%%%%%%%%%%%%%%%%%%%%%%%%%%%%%%%%%%%%%%%%%%%%%%%%%%%%%%%%%%%
% Notes to self
%%%%%%%%%%%%%%%%%%%%%%%%%%%%%%%%%%%%%%%%%%%%%%%%%%%%%%%%%%%%%%%%%%%%%%%%%%%%%%%%%%%%%%%%%%%%%%%%%%%%%%%%%%%%%%%%%%%%%%%%%%%%%%

\clearpage

{\color{red}
    \section{Misc. todo's etc}
    
    References that might be convenient:
    \begin{itemize}
        \item Bosdriesz E, Molenaar, D., Teusink, B., and Bruggeman, F.J. CHAPTER 6 How fast-growing bacteria robustly tune their ribosome concentration to approximate growth rate maximisation. Manuscr. Accept. Publ. FEBS J., 119–153.
        \cite{BosdrieszE}
        \item Maeda, M., Shimada, T., and Ishihama, A. (2015). Strength and Regulation of Seven rRNA Promoters in Escherichia coli. PLoS One 10, 1–19. \cite{Maeda2015}
        \item Gyorfy, Z., Draskovits, G., Vernyik, V., Blattner, F.F., Gaal, T., and Posfai, G. (2015). Engineered ribosomal RNA operon copy-number variants of E. coli reveal the evolutionary trade-offs shaping rRNA operon number. Nucleic Acids Res. 43, 1783–1794. \cite{Gyorfy2015}
        \item \textit{(The following two articles talk about the ribosomal RNA deletion strains, also surprising interactions with antibiotics. Of main importance is the observation that mainly tRNA seems to be limiting. See also unpublished \cite{Quan2013}.)} Bollenbach, T., Quan, S., Chait, R., and Kishony, R. (2009). Nonoptimal Microbial Response to Antibiotics Underlies Suppressive Drug Interactions. Cell 139, 707–718. \cite{Bollenbach2009} AND 1. Quan, S., Skovgaard, O., McLaughlin, R.E., Buurman, E.T., and Squires, C.L. (2015). Markerless Escherichia coli rrn Deletion Strains for Genetic Determination of Ribosomal Binding Sites. G3 Genes|Genomes|Genetics 5, 2555–2557. \cite{Quan2015} \textit{Quan et al. also refer to Orelle et al. 2013; Polikanov et al. 2014b; Orelle et al. 2015.}
        \item \textit{(This article details the rRNA regulation.)} Schneider, D.A., Ross, W., and Gourse, R.L. (2003). Control of rRNA expression in Escherichia coli. Curr. Opin. Microbiol. 6, 151–156. \cite{Schneider2003} 
        \item \textit{(These two article with description of the structures generated from rRNA genes, also mention the cleavage of the RNA genes.)}  Kaczanowska, M., and Ryden-Aulin, M. (2007). Ribosome Biogenesis and the Translation Process in Escherichia coli. Microbiol. Mol. Biol. Rev. 71, 477–494. \cite{Kaczanowska2007} AND Shajani, Z., Sykes, M.T., and Williamson, J.R. (2011). Assembly of bacterial ribosomes. Annu. Rev. Biochem. 80, 501–26. \cite{Shajani2011}
        NOTE that \cite{Chen2013} has also a picture of biogenesis and assembly.
        \item \textit{(Following article measures burden on ribosome and effect on growth in a dynamic environment.)} Shachrai, I., Zaslaver, A., Alon, U., and Dekel, E. (2010). Cost of Unneeded Proteins in E. coli Is Reduced after Several Generations in Exponential Growth. Mol. Cell 38, 758–767.        
        \cite{Shachrai2010}
        \item Maaloe1979 and Scott2010 (see Hui2015) which claims that ribosomes operate at "saturated translational capacity".
    \end{itemize} 
    
\clearpage    
}
















