\begin{table}[h]
    \begin{tabularx}{\textwidth}{llXl}
        %
        \textbf{ASC number}	& \textbf{Shorthand} & \textbf{Description} & \textbf{Source}		\\
        \hline
        ASC976  &	Prrna-C, pn25-Y	&	Δphp::pn25-mVenus-cmR, Δche::Prrsa-mCerulean-kanR. (Kanamycin and chloramphenicol resistance.)  & VS \\
        %     
        ASC968	& L19-C, pn25-Y	& L19-gc-mCerulean-kanR (GC linker), Δ(…)::pn25-mVenus-cmR.	(Kanamycin and chloramphenicol resistance.)	& VS \\
        %
        ASC1058	& L9-R, S2-Y	& Also known as JE202. rplI-mCherry-KanR (L9), rpsB-venus-CmR (S2). (Kanamycin and chloramphenicol resistance.) Gift from Johan Elf lab. & \cite{Wallden2016} \\
        \hline
        %
        % ASC631 & pn25-R, lacA-G & $\Delta$lacA::gfp-cat, $\Delta$php::pn25-mCherry-kanR & \cite{Kiviet2014} \\
        ASC666 & L31-R, gltA-G  & L31::mCherry-kanR, gltA::gfpA206K-cat & \cite{Kiviet2014} \\
        ASC810	& L31-C & L31-gc-mCerulean-kanR (GC linker). (Kanamycin resistant.) & NW, VS \\
        \hline
    \end{tabularx}
    \caption{\textbf{Strains used in this work.} Strain ASC1058 was a kind gift from the Johan Elf lab. VS indicates these strains were created by Vanda Sunderlikova, technician in the Tans lab. ASC stands for AMOLF strain collection. Note that Y, C, G and R indicate yellow, cyan, green and red fluorescent reporters, respectively.}
    \label{table:ribostrains1}
\end{table}