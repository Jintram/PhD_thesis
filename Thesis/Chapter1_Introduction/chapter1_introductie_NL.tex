
% Texstudio Spellchecker language
% !TeX spellcheck = nl_NL

\chapter*{Introductie (NL)}
\addcontentsline{toc}{chapter}{Introductie (NL)}
\setheader{Introductie (NL)}

% SWITCH LANGUAGE TO DUTCH FOR HYPHENATION
\selectlanguage{dutch}


% Who should understand this? Rob, Karin, my mother to some extend...

\section*{De kracht van getallen}

De wereld om ons heen beschrijven met behulp van getallen is niet slechts een hobby voor getallenfetisjisten.
Kwantitatieve beschrijvingen van fenomenen kunnen inzichten geven waar men anders niet toe gekomen was.
%
Ook in de biologie zijn hier voorbeelden van.
%
Zoals het Lotka-Volterra model uit de vroege 20e eeuw \cite{Lotka1920,Volterra1928}, dat fluctuaties in de hoeveelheden roofdieren en prooien verklaart met behulp van differentiaalvergelijkingen.
%
Of het wiskundige reactie-diffusiemodel van Alan Turing,
dat verklaart hoe ruimtelijke heterogeniteit spontaan kan ontstaan uit moleculen die 
interageren volgens simpele regels \cite{Turing1952}.
%
Deze principes zijn bijvoorbeeld extreem belangrijk tijdens de embryogenese, wanneer cellen moeten bepalen tot welk deel van het lichaam zij zich zullen ontwikkelen.
%
Echter,
wellicht deels vanwege een terughoudendheid bij biologen  \cite{Lazebnik2003} en deels door een gebrek aan de juiste onderzoeksmiddelen \cite{Kitano2002, Wollman2018}, 
zijn dit soort kwantitatieve methoden nooit breed toegepast in de biologie.
%
Maar sinds het nieuwe millennium heeft kwantitatieve biologie
een enorme vlucht genomen.
%
Ik hoop dat het werk dat beschreven wordt in deze thesis ook spannende voorbeelden geeft van nieuwe inzichten die verkregen zijn met deze nieuwe golf van kwantitatieve onderzoeken.




\section{Prying in the lives of single cells}

Voordat \textit{high troughput} methoden werden uitgevonden, 
bestudeerden bacteriële onderzoekers doorgaans geen individuele exemplaren van hun onderzoeksobjecten. 
%
Metingen werden gedaan aan reageerbuizen met miljoenen bacteriële cellen.
%
Zulke metingen geven een goed beeld van het gemiddelde gedrag van een bacterie. 
Hele studieboeken zijn gevuld met gedetailleerde informatie die zo verkregen is.
%
Deze manier van onderzoeken geeft echter geen volledig beeld van hoe een bacterie werkt.
%
Nieuwe technieken stellen ons in staat het leven van de individuele bacterie beter in kaart te brengen.
%
Een eeuw geleden moest men 
bacteriële kolonies nog met het blote oog door de microscoop observeren en 
handmatig 
natekenen
om delingstijden (een maat voor de groeisnelheid) van individuele bacteriën te bepalen \cite{Kelly1932}.
%
Tegenwoordig
kunnen we hetzelfde hetzelfde doen met een computer, 
en het leven van duizenden bacterien in kaart brengen in slechts een paar dagen werk.
Zie ook hoofdstuk \ref{chapter:methods} van deze thesis.
%
Figuur \ref{fig:intro:bacs}.A laat een voorbeeld zien van een microscoop-opname van een microkolonie van de staafvormige \textit{Escherichia coli} bacterie,
verkregen met een van onze microscopen.
%
Een dergelijke geautomatiseerde aanpak 
om individuele cellen (\text{single cells}) te bestuderen 
geeft nieuwe inzichten,  
%
zoals bijvoorbeeld beschreven in hoofdstuk \ref{chapter:filarecovery}.
%
Hierin beschrijven we dat 
toen we de effecten van antibiotica op individuele cellen wilden onderzoeken, 
ons iets eigenaardigs opviel \cite{RozendaalVerslagXXX}.
%viel ons iets eigenaardigs op.
%
De bacteriën stopten met delen door de antibiotica, 
maar bleven wel groeien,
waardoor zogeheten gefilamenteerde bacterien ontstonden.
%
Deze bacteriën hebben een sterk verlengde morfologie,
zie ook figuur \ref{fig:intro:bacs}.B.
%
Wanneer de antibiotica verwijderd werd, en de \ecoli met de verlengde morfologieën weer startten met delen, 
zagen we dat ze altijd deelden op zeer bepaalde relatieve posities.
%
Voor deze observaties dacht men dat gefilamenteerde bacteriën zich gedroegen als een ketting van samengevoegde bacteriën.
%
Maar in plaats van te delen volgens regels zoals je die op basis van deze aanname zou verwachten,
volgden de bacteriën een heel andere set regels.
%
Dit vereiste een onverwachte en niet eerder geziene mobiliteit van de bacteriële delingsstructuren.
%
Het liet ook zien dat bacteriële regulatiesystemen die de delingsstructuren plaatsen een functionaliteit hebben in de filamenteuze bacteriën, die eerder niet onderkend werd. 
%
Daarnaast impliceren de resultaten dat bacteriën hun grootte nauwkeurig reguleren, zelfs als ze een geprolongeerde morfologie hebben. 
%
De filamenteuze morfologie wordt niet vaak in het lab bestudeerd, maar is wel relevant in de praktijk, waar bacteriën bijvoorbeeld antibiotica kuren of koude condities (in de koelkast) overleven door een gefilamenteerde vorm aan te nemen.


\begin{figure}
    \begin{minipage}[c]{0.5\textwidth}
        %\centering    
        %\includegraphics[width=1.0\textwidth]{pdf_2016-02-17_pos2_L31-mCerulean_clouds.pdf}
        \includegraphics[width=0.99\textwidth]{figure_bacterial_colonies.pdf}
    \end{minipage}\hfill
    \begin{minipage}[c]{0.5\textwidth}
        \caption{ 
            \textbf{Bacterial colonies.}
            These pictures were taken with a microscope in the Tans lab, and show a growing microcolony of bacteria in favorable conditions (A) and (part of) a colony of bacteria that are exposed to a sub-lethal dosis of the antibiotic tetracycline (B). This colony in panel B shows a so-called filamentous morphology, because under the influence of certain stresses bacteria can stop dividing but continue elongating. 
            % 1uM TET, note that information on this dataset can be found in the script 
            % ershovwehrensallfigures
            %
            %        
        }
        \label{fig:intro:bacs}
    \end{minipage}
\end{figure}


\section*{Individuele cellen zijn anders}

\textit{Single cell} experimenten helpen ons ook verschillen tussen individuele bacteriën beter te begrijpen.
%
In 1976 lieten Spudich en Koshland in een baanbrekende studie zien dat genetisch identieke bacteriën toch ander gedrag laten zien wanneer zij naar voedsel zoeken \cite{Spudich1976}.
%
Sommige bacteriën zwommen liever lange rechte stukken, terwijl anderen liever vaker van richting veranderden tijdens het zwemmen, waardoor ze een kronkeliger zoekpad aflegden.
%
Dit verschil in gedrag werd toegeschreven aan kans-gebeurtenissen in hun interne processen.
%Dit verschil in gedrag werd toegeschreven aan gebeurtenissen met een willekeurige component in hun interne processen.
%
Het is nu duidelijk dat in het algemeen geldt dat biochemische processen die leiden tot het maken van bacteriële beslissingen niet altijd dezelfde uitkomst hebben. 
%
Er wordt gedacht dat de bron van deze stochasticiteit ligt in reacties tussen moleculen die slechts in kleine aantallen aanwezig zijn, maar plaatsvinden in relatief grote volumes.
%
Omdat de reactanten elkaar moeten vinden door diffusie, introduceert dat een kans component.
%
Deze stochasticiteit beïnvloedt ook de expressie van genen.  
%
Zelfs wanneer een gen op een constant, gelijk niveau wordt geactiveerd door regulatiemoleculen, zal de expressie van dat gen fluctueren over de tijd en verschillend zijn tussen individuen \cite{Elowitz2002}. 
%
Dit impliceert dat eiwitten concentraties in cellen constant fluctueren, wat alle cellulaire processen zal beïnvloeden.
%
Inderdaad is aangetoond dat zelfs fluctuaties in de concentraties van één enzym kan leiden tot fluctuaties in de groeisnelheid van individuele cellen \cite{Kiviet2014}.
%
Hoofdstuk \ref{chapter:literaturereview} is een literatuurstudie naar de consequenties van stochasticiteit voor het bacteriele metabolisme, en daaruit volgende effecten op de groei van individuele cellen en populatiedynamica.

\section{Regulatie en stochasticiteit}

Een interessante vraag is in hoe verre stochastische fluctuaties cellulaire regulatienetwerken verstoren. 
%
In hoofdstuk \ref{chapter:CRP} laten we zien dat een belangrijke metabool regulatie-eiwit niet alleen reageert op signalen uit cellulaire omgeving, 
maar ook reageert op stochastische fluctuaties die vanuit de cel zelf komen.
%
Dit doet vermoeden dat een stochastische fluctuatie in een eiwit concentratie die ergens in de cel ontstaat (bijvoorbeeld in de concentratie van een metabool enzym) gevolgd kan worden door een fluctuatie van een specifieke metaboliet, wat op zijn beurt weer leidt tot een regulatoire respons, wat leidt tot de productie van extra eiwitten, wat vervolgens weer leidt tot andere cellulaire responsen, et cetera. 
%
In andere woorden, fluctuaties kunnen via regulatoire interacties wellicht consequenties hebben voor alle processen in de cel.
Dit zou ook een effect kunnen hebben op bijvoorbeeld de groeisnelheid van de cel.
%
Op zijn beurt doet dat vermoeden dat cellen niet in een constante staat (chemische samenstelling) verkeren die lijkt op de gemiddelde cel, 
maar dat zij in plaats daarvan een constant veranderende staat hebben.
%
Naar aanleiding van hoofdstuk \ref{chapter:CRP} kan worden getwijfeld aan de relevantie van de vraag in hoe verre stochastische fluctuaties regulatienetwerken verstoren.
Wellicht is een relevantere vraag in hoe verre stochastische fluctuaties een integraal onderdeel zijn van die regulatie \cite{Wollman2018}.


\section{De herkomst van cellulaire individualiteit}

In het laatste hoofdstuk van deze thesis, hoofdstuk \ref{chapter:ribosomes}, proberen we cellulaire individualiteit beter te begrijpen.
%
Onze focus ligt hierbij op het ribosoom.
%
In het algemeen, op een paar uitzonderingen na, zijn alle onderdelen van de cel eiwitten, of geproduceerd door reacties die gekatalyseerd worden door eiwitten. 
Eiwitten zelf worden ook gemaakt door complexen (samengestelde structuren) van vele eiwitten. Deze worden ribosomen genoemd.
%
Ribosomen zijn ook de uitzondering op eerder genoemde regel, aangezien zij ook bestaan uit RNA.
%
Ribosomen zouden een grote bron van cellulaire heterogeniteit kunnen zijn, omdat ze vaak worden genoemd als een cellulaire onderdelen waarvan concentratie fluctuaties effecten in de gehele cel zouden moeten hebben \cite{Davidson2008, Raj2008, Chalancon2012, Bruggeman2018}.
%
Het idee hierachter is dat als de concentratie ribosomen fluctueert, de productiesnelheden van alle eiwitten die worden geproduceerd in de cel mee fluctueren. 
%
Deze cel-brede fluctuaties kunnen in potentie consequenties hebben voor alle cellulaire processen, 
inclusief groei en gedrag.
%
We hebben deze hypothese onderzocht, maar konden deze noch valideren noch weerleggen.
%
Dit zou kunnen komen doordat de ribosomen zo een ingewikkeld complex van eiwitten en RNA zijn, waarvan de onderdelen waarvan we de concentratie experimenteel kunnen volgen elk een eigen dynamiek hebben.
%
Dit laatste hoofdstuk laat dus zien dat het onderwerp cellulaire individualiteit nog vele open vragen kent.




% SWITCH LANGUAGE BACK TO ENGLISH!
\selectlanguage{english}

































