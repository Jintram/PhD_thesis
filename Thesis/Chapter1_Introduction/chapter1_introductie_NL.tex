
% Texstudio Spellchecker language
% !TeX spellcheck = nl_NL

\chapter*{Introductie (NL)}
\addcontentsline{toc}{chapter}{Introductie (NL)}
\setheader{Introductie (NL)}

% SWITCH LANGUAGE TO DUTCH FOR HYPHENATION
\selectlanguage{dutch}


% Who should understand this? Rob, Karin, my mother to some extend...

\section*{De kracht van getallen}

De wereld om ons heen beschrijven met behulp van getallen is niet slechts een hobby voor getallenfetisjisten.
Kwantitatieve beschrijven van fenomenen kunnen inzichten geven waar men anders niet toe gekomen was.
%
Ook in de biologie zijn hier voorbeelden van.
%
Zoals het Lotka-Volterra model uit de vroege 20e eeuw \cite{Lotka1920,Volterra1928}, dat fluctuaties in de hoeveelheden roofdieren en prooien verklaart met behulp van differentiaalvergelijkingen.
%
Of het wiskundige reactie-diffusiemodel van Alan Turing,
dat verklaart hoe ruimtelijke heterogeniteit spontaan kan ontstaan uit moleculen die 
interageren volgens simpele regels \cite{Turing1952}.
%
Deze principes zijn bijvoorbeeld extreem belangrijk tijdens de embryogenese, wanneer cellen moeten bepalen tot welk deel van het lichaam zij zich zullen ontwikkelen.
%
Echter,
wellicht deels vanwege een terughoudendheid bij biologen  \cite{Lazebnik2003} en deels door een gebrek aan de juiste onderzoeksmiddelen \cite{Kitano2002, Wollman2018}, 
zijn dit soort kwantitatieve methoden nooit breed gebruikt in de biologie.
%
Maar sinds het nieuwe millennium heeft kwantitatieve biologie
een enorme vlucht genomen.
%
Ik hoop dat het werk dat beschreven wordt in deze thesis ook spannende voorbeelden geeft van nieuwe inzichten die verkregen zijn met deze nieuwe golf van kwantitatieve onderzoeken.




\section{Prying in the lives of single cells}

%
%In the Tans lab, we use a microscope and a computer to make time lapse movies of living and growing bacteria.
%
%Until recently, bacterial researchers were not able perform experiments on a specimen of their organism of interest.
%Until recently
Before the onset of techniques that allowed high throughput quantitative measurements, bacterial researchers usually did not perform studies on individual specimens of their organism of interest.
%
Experiments 
%to learn about bacterial behaviour 
were conducted in test tubes on millions of individual cells at once.
%
While this gives great information about the average behaviour of bacteria, 
which filled text-books with the detailed workings of bacteria,
it does not allow one to learn everything about how a bacterium works.
%
% An example of this is given in chapter \ref{chapter:filarecovery}.
%
New techniques now allow us to better probe the life of single cells.
%
Where a century ago people had to observe tiny bacterial colonies by eye through the microscope, painstakingly draw them, and quantify division times manually \cite{Kelly1932},
we can now do the same using a computer and 
% our computerized analysis of the data allows us to 
thoroughly track the lives of thousands of individual cells in a few days work, as described in chapter \cite{chapter:methods} of this thesis.
%
Figure \ref{fig:intro:bacs}.A shows an example of a snapshot of a microcolony of the rod-shaped \textit{Escherichia coli} bacterium, acquired by one of our microscopes.
%
This computerized approach gives new insights, as illustrated by chapter \ref{chapter:filarecovery}.
%
During single cell experiments in the Tans lab that probed the effect of antibiotics on \ecoli bacteria \cite{RozendaalVerslagXXX} we noticed a peculiarity.
%
Bacteria stop dividing in this condition, but continue elongating, leading to so-called filamentous phenotypes with an elongated morphology, see also figure \ref{fig:intro:bacs}.B.
%
When antibiotic stress was removed, and \ecoli with elongated morphologies started dividing again, 
we noticed that 
they always divided at very defined relative positions.
%
Before these observations, it was thought that filamentous bacteria behave as if they were a string of multiple bacteria merged together.
%
But our measurement showed that instead of following rules according to that principle, 
bacteria followed another set of rules.
%
This required an unexpected and not previously observed mobility 
of bacterial division site structures.
% in the placement of their potential division sites.
It also revealed that 
one of the bacterial regulation systems for division site placement
 % division regulation system has 
has a previously unrecognised functionality when cells are in their filamentous state.
%
Aside from these two novel observations, 
the results implied that bacterial cells carefully regulate their size,
even in this elongated morphology.
%far beyond often observed bacillary form (the short rod-like shape).
%
%Conditions that induce t
This morphology is not often studied in the lab, 
but is very relevant in daily life where bacteria might for example survive antibiotic treatments or cold conditions (fridges) by adopting this filamentous form.
%for example proliferation of undesired food-borne or pathogenic bacterial is combated exactly by exposing to bacteria to harsh conditions like heat, cold, or antibiotics.
%
%Concluding, chapter \ref{chapter:filarecovery} illustrates both the importance of investigating individual bacteria and the importance of investigating other conditions other than standard lab growth conditions. 

\begin{figure}
    \begin{minipage}[c]{0.5\textwidth}
        %\centering    
        %\includegraphics[width=1.0\textwidth]{pdf_2016-02-17_pos2_L31-mCerulean_clouds.pdf}
        \includegraphics[width=0.99\textwidth]{figure_bacterial_colonies.pdf}
    \end{minipage}\hfill
    \begin{minipage}[c]{0.5\textwidth}
        \caption{ 
            \textbf{Bacterial colonies.}
            These pictures were taken with a microscope in the Tans lab, and show a growing microcolony of bacteria in favorable conditions (A) and (part of) a colony of bacteria that are exposed to a sub-lethal dosis of the antibiotic tetracycline (B). This colony in panel B shows a so-called filamentous morphology, because under the influence of certain stresses bacteria can stop dividing but continue elongating. 
            % 1uM TET, note that information on this dataset can be found in the script 
            % ershovwehrensallfigures
            %
            %        
        }
        \label{fig:intro:bacs}
    \end{minipage}
\end{figure}


\section{Single cells are different} 

Single cell experiments also allow us to probe the differences between individual bacteria.
%Another interesting question when probing single bacteria is asking to what extent individuals are different.
%But observing bacteria does not only give insights about behaviour that is (to an extent) identical in all bacteria.
%
In a pioneering study in 1976, 
%
% As observed in a pioneering study by 
Spudich and Koshland \cite{Spudich1976} found that genetically identical individual bacteria can still show different food-searching behaviour. 
% despite being genetically identical.
In brief, some bacterial individuals favoured swimming straight for long stretches in search of food, 
whilst others preferred to re-orient their direction more often to create a more winding search pattern for food. 
%some bacteria favoured swimming longer distances, whilst others favoured more careful sensing of gradients.
%
This was attributed to \lq{chance occurrences [in their] internal processes}\rq.
%
It is now clear that in general,
%In other words, 
the biochemical reactions that finally lead to bacterial decision making do not always have the same outcome.
%
It is thought that the origin of this stochasticity can be traced back to reactions that involve a small number of molecules in a relatively large volume.
Since reactants need to find each other by diffusion, this introduces a component of chance.
%
This stochasticity can also be observed in gene expression.
Even with precisely the same amount of activation by regulatory molecules, gene expression will fluctuate over time and differ between individuals \cite{Elowitz2002}.
%
This implies that protein concentrations in bacterial cells fluctuate constantly,
which in turn must affect all cellular processes.
%
Indeed, it has been suggested that even fluctuations in single enzymes can result in fluctuations in single cell elongation rates (i.e. growth rates) \cite{Kiviet2014}.
%
We review literature about the consequences of stochasticity on the bacterial metabolism, and subsequent effects on single cell growth and population dynamics in chapter \ref{chapter:literaturereview}.

\section{Regulation and stochasticity}

One could ask to what extent these stochastic fluctuations disturb the cellular regulatory networks.
%
In chapter \ref{chapter:CRP} we show that an important metabolic regulatory protein not only responds to the cellular environment,
but can also respond to stochastic fluctuations that occur inside the cell itself.
%
This suggests that a stochastic protein concentration fluctuation occurring somewhere in the cell (e.g. in single a metabolic enzyme),
might be followed by a fluctuation in the concentration of a specific metabolite, in turn leading to a regulatory response, 
leading for example to the production of additional proteins, which consequently triggers again different responses, etcetera.
%
Fluctuations might thus have cell-wide consequences, for example on cellular growth rate, mediated by regulatory interactions.
%
This implies that cells might not exist in a well-defined average state,
but instead by nature have an ever-changing state.
%
%Indeed, we show in chapter \ref{chapter:CRP} that a model in which the expression proteins and cellular growth rate 
%are continuously subject to fluctuations and also can interact continuously,
%
%In chapter \ref{chapter:CRP} we expand a previous model \cite{Kiviet2014} in which the expression proteins and cellular growth rate 
%are continuously subject to fluctuations and also can interact continuously.
%%
%This model offers an interpretation to the relationships between single cell growth and metabolic regulatory activity that we experimentally measure.
%%
%To show that 
%
This chapter % perhaps 
indicates that asking to what extent stochastic fluctuations disturb regulatory networks might in fact be the wrong question,
and instead one should be asking to what extent fluctuations are an integral component of regulatory networks \cite{Wollman2018}.

\section{Sources of cellular individuality}

In the final chapter of this thesis, chapter \ref{chapter:ribosomes}, we try to further understand cellular individuality.
%
We focus on the ribosome.
%
As a general rule, 
with a few exceptions, all components of the cell either are proteins, or are produced by reactions that are catalyzed by proteins.
%
Proteins themselves are also produced by complexes (i.e. superstructures) of many proteins, which are called ribosomes.
%
Ribosomes are also an exception to aforementioned general rule, 
as they also contain RNA.
%an exceptional cellular structure, 
%%to the rule 
%as also RNA is a key compoonent. % contain RNA.
%%
Ribosomes could be major contributors to cellular individuality,
as they are often cited as a cellular component that can result in cell-wide protein fluctuations \cite{Davidson2008, Raj2008, Chalancon2012, Bruggeman2018}.
%
The idea is that when the concentration of ribosomes fluctuates in a cell, 
the production rates of all proteins that are being produced in that cell also fluctuate simultaneously.
%
These cell-wide fluctuations could potentially have implications on all cellular processes,
including behaviour and growth. 
%
We investigated this hypothesis, but could neither validate nor disprove it.
%
This might be because the ribosome is such a complex structure,
of which the different components that we are able to track experimentally might show different dynamics.
%
This final chapter therefore illustrates that there are still many open questions 
in our understanding of cellular individuality.

% SWITCH LANGUAGE BACK TO ENGLISH!
\selectlanguage{english}

































