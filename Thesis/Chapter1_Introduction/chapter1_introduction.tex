\chapter*{Introduction}
\label{chapter:introduction}


%****************************************************************************
%\begin{abstract}
%no abstract for introductionchapter
%\end{abstract}

%% Start the actual chapter on a new page.
%\newpage % no newpage for introductionchapter


%****************************************************************************
%****************************************************************************
\section{Martijn's great introduction, first section}

%%%%%%%%%%%% Figure Intro Cell-to-cell Variability
%\begin{figure}[h]
%	\includegraphics[width=\textwidth]{XXX}
%	\caption{\label{fig:ch1examplesvariability} \textbf{Examples of cell-to-cell variability.} \figA Genetically identical \ecoli bacteria produce different amounts of proteins (upper panel) and the protein concentration in single cells fluctuates over time (lower graph). \figB Clonal \textit{B. subtilis} bacteria can differentiate into different cell types. The differentiation is driven by stochastic protein production. Figure \figA was taken from \cite{Taniguchi2010}, \figB was taken from \cite{Suel2006}.}
%\end{figure}
%%%%%%%%%%%%

Even in a single species of enzymes, the catalytic rate can vary from one enzyme to the next \cite{Lu1998}. (Interestingly, the average waiting time for a reaction has the same form as the Michaelis Menten equation \cite{Xie2013}.)

Transcription is a bursty process \cite{Golding2005}. 

Global trends in noisy protein expression have been investigated. 
Noise scales with the amount of protein that is expressed \cite{Bar-Even2006}.

\section{Random notes}

In eukaryotes, it is feasible to detect transcript from the RNA \cite{Levsky2002}.


%****************************************************************************
%****************************************************************************
% avoid floats to appear after the footnote.
\FloatBarrier

%\blfootnote{write a footnote?}

