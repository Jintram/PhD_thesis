
HELPS FURTHER UNDERSTANDING OF BIOCHEMICAL NETWORKS 
EVOLUTION DOES NOT ONLY CARE ABOUT STEADY STATE BUT INSTEAD EVERYTHING IS EVOLVABLE

We use engineered bacterial strains with and without 

Experimentally measured differences in the dynamics between an engineered bacterial strains with and without a regulatory feedback interaction,

\clearpage
\section{bla}

explains 

Indeed, we show in chapter \ref{chapter:CRP} that relationships between protein expression and growth can be understood in terms of the 





This suggests that fluctuations occur in activity and concentrations of regulatory circuits, 
which are partially due to 

Indeed, relationships between two signals CAN BE UNDERSTOOD IN DYNAMICAL SENSE






THIS SHOWS THAT THIS IS PERHAPS THE WRONG QUESTION TO ASK


How much do all these fluctuations disturb the cellular regulatory networks?
%


It is clear that cells sense their environment and adjust their protein levels to best thrive in the environment.
%
But how important is it for the cell to keep protein levels very close to target values?
%
As just discussed, all protein levels fluctuate to a certain extent.
%
In chapter \ref{chapter:CRP} we 




But does that mean that all protein levels are set to certain target values, or that 



A prevailing idea is that bacterial cells sense their environments and aim to set protein levels to certain values in order to cope with the environment.
%



Cells sense their environment and adjust their protein levels to best cope with the environment.
%


we discover hypothesis that fluctuations are integral part of regulatory circuits

\section{bla}

TWO CONCEPTS COME TOGETHER

the filamentous bacteria re-arranged their 
%
We showed that this is due to a novel functionality of the well-known MinC, MinD and MinE proteins.
These three proteins form a reaction diffusion system that can create patterns employing the aforementioned Turing mechanism, 
which was known to help non-filamentous cells divide at half their length. 
%

%
Since this kind of single cell division data was not available before, the specific division pattern was not measured before.
OPENS UP NEW INSIGHTS / LARGE IMPLICATIONS
%
We further used 



The division of single bacterial cells could not be tracked before 



\section{blabla}

Experiments were conducted by looking at test tubes with hundreds of millions of cells\footnote{} 

now
what used to take ages
can be done in a few days


Specifically, we are interested in the lives of single cells.
%

%
%


In the Tans lab, we work with single cells, one of the recent advances 
%
and noticed something peculiar when observing .

\section{Individuality in single cells}



\section*{stuff}
But recently, also more and more research into biology takes advantage of quantitative 


The advance of new techniques in biology has 


There are some canonical examples in the field of biology.
%
Like the Lotka-Volterra model from the early 1900s, which explains fluctuations in predator and pray numbers using differential equations \cite{Lotka1920,Volterra1928}.
Or the reaction-diffusion models described by Turing, which explain how large spatial inhomogeneities can spontaneously arise from molecules with simple interaction rules \cite{Turing1952}.
%
Nevertheless, the 

Lazebnik2003


This is also true in the domain of biology.
%
Phenomena that include the 


For example, during the development of an embryo, the multiplying clump of cells need to decide which cells eventually become the head, which cells the end of a toe, etcetera.
%
The precise concentration of certain regulatory factors (morphogens) helps to decide the eventual fate of cells.
%
Mathematical descriptions are extremely useful in this context.
%
For example, it was recently shown that 

%\begin{figure}
%    \begin{minipage}[c]{0.7\textwidth}
%        %\centering    
%        %\includegraphics[width=1.0\textwidth]{pdf_2016-02-17_pos2_L31-mCerulean_clouds.pdf}
%        \includegraphics[width=0.99\textwidth]{lizard_Manukyan2017.png}
%    \end{minipage}\hfill
%    \begin{minipage}[c]{0.3\textwidth}
%        \caption{ 
%            \textbf{Top view of an ocellated lizard.}
%            The pattern on the skin of the lizard is an example of a process that can be understood by writing down a mathematical description of the process.
%            Image taken from \cite{Manukyan2017}.
%            %
%            %        
%        }
%        \label{fig:ribo:switch1}
%    \end{minipage}
%\end{figure}




\subsection*{stuff}

 the development of an embryo is governed by interaction between molecules where 


Classically, biology has always been a very qualitative science.
%
Processes were described in terms 
Focus lay on identifying which factors (which protein or gene, for instance) were involved in certain phenotypes, and whether interaction existed between factors.
%
DNA replication can be 




%Biology has long been a crude science \cite{Lazebnik2003}.
%%
%To understand cellular decision networks, it was deemed sufficient to find out which protein interacted with which other protein or DNA sequence, or what the effect of a protein knockout was.
%%
%In recent years, the emerging field of systems biology, or quantitative biology, has introduced a new way of thinking.
%%In recent years, the emerging field of systems biology has provided us with many insights by \red{try thinking of some examples}
%%
%By careful quantitative analysis of the cell, and mathematical analyses that might only describe the interactions between a few cellular components, albeit in a minute way, 
%one could fundamentally understand and predict decision networks in the cell \cite{Alon2006}.
%
%In this thesis, I will present examples of such ....
%
%In the first chapter of this thesis however, we discuss ...
%
%A good example of a system is the Min system.




\section*{Notes}

Mention Kelly et al 1932 \cite{Kelly1932}?!

\section*{more stuff}


Wollman2018 (review)
Karlebach2008,Ideker2012,DiVentura2006 (reviews) \cite{Karlebach2008,Ideker2012,DiVentura2006}.
Davidson2008 (individuality in bacteria) \cite{Davidson2008}.



%\section{Martijn's great introduction, first section}
%
%%%%%%%%%%%%% Figure Intro Cell-to-cell Variability
%%\begin{figure}[h]
%%	\includegraphics[width=\textwidth]{XXX}
%%	\caption{\label{fig:ch1examplesvariability} \textbf{Examples of cell-to-cell variability.} \figA Genetically identical \ecoli bacteria produce different amounts of proteins (upper panel) and the protein concentration in single cells fluctuates over time (lower graph). \figB Clonal \textit{B. subtilis} bacteria can differentiate into different cell types. The differentiation is driven by stochastic protein production. Figure \figA was taken from \cite{Taniguchi2010}, \figB was taken from \cite{Suel2006}.}
%%\end{figure}
%%%%%%%%%%%%%
%
%Even in a single species of enzymes, the catalytic rate can vary from one enzyme to the next \cite{Lu1998}. (Interestingly, the average waiting time for a reaction has the same form as the Michaelis Menten equation \cite{Xie2013}.)
%
%Transcription is a bursty process \cite{Golding2005}. 
%
%Global trends in noisy protein expression have been investigated. 
%Noise scales with the amount of protein that is expressed \cite{Bar-Even2006}.
%
%\section{Random notes}
%
%In eukaryotes, it is feasible to detect transcript from the RNA \cite{Levsky2002}.


%****************************************************************************
%****************************************************************************
% avoid floats to appear after the footnote.
\FloatBarrier

%\blfootnote{write a footnote?}