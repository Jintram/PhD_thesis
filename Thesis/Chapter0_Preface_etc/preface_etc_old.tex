

%\begin{figure}
%	\centering
%	\includegraphics[width=1.5cm]{eglyphM.jpg}
%	\clearpage % insert a page break
%	% https://www.penn.museum/cgi/hieroglyphsreal.php
%\end{figure}

    

\chapter*{Propositions}

\textit{These propositions are regarded as opposable and defendable, and have
    been approved as such by the promotor prof.dr.ir. Sander J. Tans}

% GUIDELINES THESIS
% At least six DO NOT relate topic thesis --> 4 do
% guideline: total <11
% maximum of two can be playful in nature
% should demonstrate "broad base scientific knowledge"
    % new perspectives
    % incidental results not included in thesis
    % criticism literature
    % comments on related disciplines/methods
    % speculation about future developments

\begin{enumerate}[nosep]%[itemsep=0mm]
    % Thesis
    % ...
%1 ----------------------------------------
    \item One cannot have a meaningful discussion without accepting a non-zero probability that one might not be right.
%2 ----------------------------------------
    \item 
    It is commonplace in scientific community that 
    one junior scientist works on one project. 
    Instead, 
    teams of 2-3 scientists should work on projects. 
%    Research groups should be subdivided in 
%    (overlapping) teams, each responsible for a project,
%    instead of projects being the responsibility of single persons.
    %    Science is too individualistic - only one person working on a project occurs too often.
%3 ----------------------------------------
    %    \item Junior scientists should be paid in an honest manners. Overtime should be paid, even if this means the total salary goes down. (The same applies to other professions
    \item 
    Junior scientists often make long, unpaid hours, which is counter-productive\footnote{
        See for example the article \textit{Proof that you should get a life} by The Economist (2014): \\
        %        \textit{Proof that you should get a life.} The Economist (2014). \\ %Available at:
        https://www.economist.com/free-exchange/2014/12/09/proof-that-you-should-get-a-life.\\ 
        %         [Accessed September 12, 2018].
    }.
    To remove the incentive for management to allow this, overtime should be paid.
    %    %
    %    % The culture of
    %    Making long hours (in science) is counter-productive\footnote{
    %        See for example the article \textit{Proof that you should get a life} by The Economist (2014): \\
    %    %        \textit{Proof that you should get a life.} The Economist (2014). \\ %Available at:
    %         https://www.economist.com/free-exchange/2014/12/09/proof-that-you-should-get-a-life.\\ 
    %    %         [Accessed September 12, 2018].
    %     }. % \cite{economist2014}.
    %    %    \item The culture of admiration for making long hours in science is both counter-productive and unethical.
    %    %, both because this drives smart people out of science and because longer hours don't necessarily increase total output (the marginal product decreases).
    %    To remove the incentive for management to allow this, 
    %%    Therefore compensation should be honest: 
    %    overtime should be paid, even if this means a decrease in salary.
    %    %    Junior scientists should be paid in an honestly: overtime should be paid, even if this means the total salary goes down.
%4 ----------------------------------------
    \item
    %    In addition to last names being inherited through the paternal line,
    %    children should get a second last name inherited through the maternal line.
    %
    Children should be given two last names: 
    one inherited through the paternal line and 
    one through the maternal line.
    %
    Dutch law should allow for two last names.  
%    To accommodate this, Dutch law should allow children to have two last names.
%5 ----------------------------------------
    \item Any study program should include a course that discusses scientific insights on how humans learn most effectively\footnote{An excellent argument for this is made in the book \textit{Make it Stick - The Science of Successful Learning} by Peter C Brown (Harvard University Press, 2014).}.
    %    *about studying how to learn best -- ie. the learning book*
%6 ----------------------------------------
    \item When decisions by 
    %    	(government) 
    institutions about citizens are made by algorithms,
    citizens should be enabled to talk to a person with the power to overrule those decisions.
    %    \item something that there should always be a human that can overrule the decision of an algorithm
%7 ----------------------------------------
    \item 
    %    PhD students 
    Scientists
    should write fully fledged manuscripts
    about their projects twice a year % yearly % every half year
    and discuss these with their research leader,  
    independent of publication plans.   
%    whether anything is published or not. %it is publishable or not.
    % you should write a fully fledged manuscript about your results (and discuss with your supervisor) every half year, whether its publishable or not
%8 ----------------------------------------
    \item Thirty years in the future, a full model of a cell can be made based on an automated biochemical analysis protocol and a computer analysis.
%X ----------------------------------------    
    % More propositions
%    \item singularity AI?    
%    \item The requirement for international experience in academic positions is based on culture, not on merit.
%    \item Creating interesting stories out of scientific data leaves a load of stories untold, and labor wasted.    
%    \item Scientific papers should be written with a more pedagogical goal.
%    \item Printing a thesis is an outdated concept.
%    \item It is better to be lazy than to fail at doing three experiments simultaneously.
    % \item The ability to evaluate one's behavior and [zelfkritiek is belangrijk / niet vastroesten in patronen]
%    \item *Science can learn more from companies*
    %
    % PROPOSITIONS THAT PERTAIN TO THE THESIS
    % ----------------------------------------
    \item The Min system acts as a ruler which allows \textit{Escherichia coli} bacteria to measure absolute distances.
%    The Min system controls where cells divide in filamentous cells.\\
    \textit{This proposition pertains to this dissertation (chapter \ref{chapter:filarecovery}).}
    %
    % ----------------------------------------
    \item 
    The filamentous \ecoli bacterial shape is as important as the rod shape, 
    which is reflected by 
    the careful regulation of the division process in both states.
%    a carefully regulated division process in both states.
    %
    %    The filamentous cell state is as well-regulated as the bacillary cell-state, 
    %    and these two states are of equal importance a natural environment.
    %
    %    Filamentation is not a fringe phenomenon: cells in the real world are often exposed to stressful situations that require growth but no division.\\
    \textit{This proposition pertains to this dissertation (chapter \ref{chapter:filarecovery}).}
    %
    % ----------------------------------------
    \item 
    Even in a constant environment  
    the cellular state is dynamic,
    and traverses through the cellular phase space propelled by fluctuations and regulatory interactions.
    %
    %The cellular state is not constant when cells grow in a constant environment.
    %Fluctuations in individual cells make that the cellular state changes constantly over a wide range of concentrations and growth rates.
    \textit{This proposition pertains to this dissertation (chapter \ref{chapter:CRP}).}
    %
    % ----------------------------------------
    \item
    The growth rate of individual cells is not correlated to their individual ribosomal concentrations.
    %\item Ribosomal proteins do not simply form a the ribosome, instead their differential expression can address precise protein expression needs.\\
    \textit{This proposition pertains to this dissertation (chapter \ref{chapter:ribosomes}).}
    % ----------------------------------------
    \item
    %  For the ribosome to function, not all ribosomal proteins need to be expressed in a one to one stoichiometry. 
    To function, a ribosome does not need all ribosomal proteins.  
    Separate regulation on ribosomal proteins might modulate the function of the ribosome. 
    %A regulated 
    %deviation from that one to one stoichiometry of a ribosomal protein 
    %might modulate the function of the ribosome. 
    \textit{This proposition pertains to this dissertation (chapter \ref{chapter:ribosomes}).}
    %
    % ----------------------------------------
\end{enumerate}


