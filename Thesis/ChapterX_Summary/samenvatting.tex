\chapter*{Samenvatting}
\addcontentsline{toc}{chapter}{Samenvatting}
\setheader{Samenvatting}

% SWITCH LANGUAGE TO DUTCH FOR HYPHENATION
\selectlanguage{dutch}


In dit proefschrift onderzoeken we individuele bacteriecellen.
We proberen hiermee meer inzicht te krijgen in het proces van celdeling tijdens ongunstige condities en
het fenomeen van heterogeniteit in populaties beter te begrijpen.


In \textbf{hoofdstuk \ref{chapter:introduction}} geven we een overzicht van de onderwerpen in dit proefschrift,
alsook context voor de leek.
%
In \textbf{hoofdstuk \ref{chapter:methods}} beschrijven we vervolgens hoe we huidige methoden hebben uitgebreid om het gedrag van 
individuele \textit{Escherichia coli} cellen te kunnen bestuderen.
%
We bespreken een PDMS \textit{microfluidics} opstelling die we gebruiken om microcolonies van bacteriecellen bloot te stellen aan veranderingen in groeimedium en te observeren, 
software uitbreidingen die onze analyse van individuele cellen faciliteren,
en hoe we cross-correlaties kunnen toepassen op onze datastructuur (cross-correlaties zijn relevant in onze studie naar heterogeniteit).

In \textbf{hoofdstuk \ref{chapter:filarecovery}} beschrijven we nieuwe inzichten aangaande celdeling in ongunstige condities.
%
In plaats van cellen te laten groeien in gunstige condities, wat meestal gedaan wordt in het lab, hebben we cellen laten groeien in ongunstige condities.
%
Deze condities, namelijk blootstelling aan antibiotica en hoge temperaturen, kunnen cellen ook in de echte wereld tegenkomen.
%
De cellen reageren op deze ongunstige condities door te stoppen met delen terwijl ze wel blijven groeien. 
%
Dit proces wordt ook wel filamentatie genoemd, 
en leidt tot zeer lange cellen.
%
Wanneer we overschakelden naar gunstige groeicondities, 
herstelden de cellen zich door weer te gaan delen.
%
Tijdens dit proces verplaatsten de cellen hun potentiele delingsplekken (Fts ringen) continu.
%
De plaatsing van de Fts ringen hing af van de lengte van de bacterie en werd bepaald door regulatie door Min eiwitten.
%
De Min eiwitten bleken te functioneren als waren zij een dynamische lineaal.
%
Dit resulteerde
in zeer precieze regulatie van waar
daadwerkelijke celdelingen plaatsvonden.
%
We laten in dit hoofdstuk ook zien dat de timing van de delingen plaatsvond volgens het zogenaamde \textit{adder} principe (letterlijk vertaald: toevoeger principe). 
Dit dicteert dat cellen gemiddeld delen wanneer zij met een specifieke volume gegroeid zijn.
%
Deze observaties aangaande het Min systeem en het \textit{adder} principe zijn opzienbarend, 
aangezien deze systemen tot nu toe vooral beschouwd waren in de context van cellen met een normale morfologie.
%
De resultaten in dit hoofdstuk laten zien dat \ecoli cellen continu hun lengte meten om hun grootte te controleren,
en duiden op een bredere relevantie van het 
\textit{adder} principe en werpen een nieuw licht  
op de functies van het Fts en Min systeem.

\red{Wellicht handig eerst even feedback af te wachten alvorens vertaling af te maken.}










% SWITCH LANGUAGE BACK TO ENGLISH!
\selectlanguage{english}