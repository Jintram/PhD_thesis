

\chapter*{Summary}
\addcontentsline{toc}{chapter}{Summary}

\setheader{Summary}

In this thesis, we probe single cells to 
further understand both
the regulation of cell divisions during adverse conditions
and the phenomenon of cellular heterogeneity.
%
Firstly, in introductory \textbf{chapter \ref{chapter:introduction}} we 
provide an overview of the topics in this thesis and provide some context for the layman.
%
Then, in \textbf{chapter \ref{chapter:methods}} we describe how we expanded on current methods to investigate the behaviour of single \textit{Escherichia coli} cells: 
%
we discuss the PDMS device that we used to subject microcolonies of cells to changes in growth medium and observe them,
% observe single cells that grow in microcolonies, 
the software expansions that simplified and enabled analysis of our single cell time lapse data, and 
the application of cross-correlation analysis to branched lineage data that is acquired by observing growing microcolonies of cells (which is relevant for studying heterogeneity).
%We discuss the PDMS device that we used, the software expansions that simplified and enabled analysis of our time lapse data, and 
%the application of cross-correlation analysis to branched lineage data that is acquired from growing microcolonies of cells.

In \textbf{chapter \ref{chapter:filarecovery}} we discuss 
novel findings regarding 
the regulation of cell divisions during adverse conditions.
%
Instead of growing \ecoli in favourable conditions, as is often done in the lab, we subjected the bacteria to 
real-world conditions like antibiotic exposure and high temperatures.
%
The cells responded to these adverse conditions by halting the cell division process, whilst they continued to grow.
This process is called filamentation and it leads to typical elongated cell morphologies.
%
%Importantly, when we switched growth conditions back to favourable conditions,
%the cells started dividing again to eventually recover their normal bacillary form according to a very specific set of rules.
%%
%The cells continuously re-arrange potential division sites (Fts rings), to place them at relative locations that depend on the total length of the bacterium.
%
Importantly, when we switched growth conditions back to favourable conditions,
the cells started dividing again to eventually recover their normal bacillary form.% according to a very specific set of rules.
%
During this process,
the cells continuously re-arrange potential division sites (Fts rings), 
to place them at locations along the cellular axis depending on a very specific set of rules. % that depend on the total length of the bacterium.
%
Where Fts rings formed depended on the length of the individual bacteria,
and we showed that this placement was regulated by Min proteins.
%Min proteins were hitherto only known for regulating division site placement in bacteria with the normal bacillary morphology. 
%
We also showed that the timing of divisions was set by the so-called adder mechanism.
This means bacteria on average divide each time they have grown by a specific volume.
%This is remarkable, since the adder mechanism is usually considered 
%with regard to 
%size regulation of bacillary shaped bacteria,
%but during divisions of filamentous bacteria the size of offspring is determined by the location of divisions.
%
The observations on the Min regulation and adder mechanism were remarkable,
since these systems were hitherto only known for regulation of division in bacillary shaped bacteria.
%
Taken together, the results in chapter \ref{chapter:filarecovery} indicate that E. coli cells continuously keep track of absolute length to control size, 
%and suggested a wider relevance for the adder principle beyond the control of normally sized cells,
suggest a wider relevance for the adder principle 
and provide a new perspective on the function of the Fts and Min systems.



In chapters \ref{chapter:literaturereview}-\ref{chapter:ribosomes} we investigate the origins of cellular heterogeneity.
%
As described in chapter \ref{chapter:introduction}, 
%it was recognized decades ago that a 
even individual bacteria in an isogenic population in a constant environment can show different behaviour,
which ultimately stems from the stochastic nature of 
chemical reactions that go on inside the cells.
%
Research into this heterogeneity often either focuses on processes where stochasticity in a process can be directly linked to a phenotypic effect
or focuses on how signalling can be robust despite noise.
%
How noise transmits through large biochemical networks, and what effects this has on the cellular state, is researched less often.
%
In
\textbf{chapter \ref{chapter:literaturereview}} we review literature that focuses on this question,
with a specific focus on the metabolism, 
%which 
since this 
is an important large biochemical network in the cell.
%
In \textbf{chapter \ref{chapter:CRP}} we investigate the role of regulatory networks in heterogeneity by looking at single cells growing in a constant environment. 
%
Specifically, we look at the 
cAMP receptor protein (CRP), 
which is a master regulator of metabolic enzyme expression.
%
These enzymes convert carbon molecules taken up by the cell to smaller metabolites and generate energy during this process.
%
The concentration of these smaller metabolites also inhibits CRP activity.
This negative feedback loop was previously reported to be the regulatory interaction that is responsible for 
setting metabolic enzyme expression to the optimal level.
%
Given the hypothesis described in chapter \ref{chapter:literaturereview} that metabolite levels continuously fluctuate in a single cell, even in steady state, 
we speculated that the CRP regulatory system must also continuously receive different inputs, to which it might respond.
%
To test this hypothesis we used a previously designed strain that lacked the metabolic feedback.
% strain that was unable to produce cAMP, which thus lacked the metabolic feedback.
By subjecting this strain to alternating high and low input signals, 
we showed that the regulatory systems can respond on the fast timescales that are associated with stochastic fluctuations.
%
We then investigated whether the dynamic behaviour of the regulatory interaction was different 
between a wild type strain (that still had the feedback) and the strain lacking the feedback.
%
Using cross-correlation analysis of expression-growth dynamics and mathematical modelling of the dynamics,
we revealed noteworthy differences between the case with feedback and the case without feedback.
%
This suggests that regulatory interactions indeed respond to stochastic fluctuations that occur within the cell.
%
This calls for speculation that even regulation networks in cells in a constant environment 
%the regulation networks in cell 
continuously interact and adjust, 
resulting in a perpetually changing cellular state fuelled by random events.




Finally, in \textbf{chapter \ref{chapter:ribosomes}} we focus on the role ribosomal concentration fluctuations might have in cellular heterogeneity.
%
%Almost two decades ago, the terms extrinsic and intrinsic noise were coined.
%These terms refer to noise originating from the biochemical processes of gene expression itself (intrinsic noise), resulting in independent expression fluctuations,
%and fluctuations in cellular components that transmit to the expression of all genes in a cell at once (extrinsic noise), resulting in concerted expression fluctuations \cite{Elowitz2002}.
%
It is often suggested that stochastic fluctuations in the ribosomal concentration might 
contribute to concerted fluctuations in gene expression. 
%act as a source of extrinsic noise.
This suggests that there might be 
%If this is the case, then we expect 
positive correlations between ribosomal expression and the expression of arbitrary other proteins at some time later.
%
Additionally, given the pivotal role that the production of proteins has in cellular growth, 
we hypothesized that single cell fluctuations in ribosomal concentration 
might result in single cell growth rate fluctuations.
This would suggest that there should also be positive correlations between ribosomal concentration fluctuations and single cell growth rate.
%Additionally, we hypothesized that given the central role of the ribosomes in the production of proteins there might be 
%
In chapter \ref{chapter:ribosomes} we investigate these hypotheses,
and labeled ribosomal proteins and used ribosomal RNA reporters to study the ribosomal dynamics.
%
We additionally introduced 
%Additionally we introduced 
a constitutively expressed fluorescent reporter, which allowed us to 
correlate ribosomal concentration with protein production.
%
We measured the expression of these reporters and growth rates in single cells in different media, 
and also tested conditions where cells were exposed to sub-lethal concentrations of antibiotics that inhibited translation.
%
Cross-correlation analysis of these experiments 
provided insufficient evidence to clearly support either 
transmission of ribosomal fluctuations to
protein expression
nor transmission to growth rates.
%
We end the chapter by reflecting on the question whether a single labeled ribosomal proteins is good proxy for the concentration of 
completely assembled active ribosomes,
perhaps each of the 58 ribosomal proteins have their own unique dynamics. 
%
We also suggest a future experiment that might cause transmission of ribomal RNA fluctuations to growth rates, and make this apparent.






\chapter*{Samenvatting}
\addcontentsline{toc}{chapter}{Samenvatting}

\setheader{Samenvatting}

