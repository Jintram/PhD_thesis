

\chapter*{Summary}

In this thesis, we probe single cells to 
further understand both
the regulation of cell divisions during adverse conditions
and the phenomenon of cellular heterogeneity.
%
Firstly, in introductory \textbf{chapter \ref{chapter:introduction}} we 
provide an overview of the topics in this thesis and provide some context for the layman.
%
Then, in \textbf{chapter \ref{chapter:methods}} we describe how we expanded on current methods to investigate the behaviour of single \textit{Escherichia coli} cells: 
%
we discuss the PDMS device that we used to subject microcolonies of cells to changes in growth medium and observe them,
% observe single cells that grow in microcolonies, 
the software expansions that simplified and enabled analysis of our single cell time lapse data, and 
the application of cross-correlation analysis to branched lineage data that is acquired by observing growing microcolonies of cells (which is relevant for studying heterogeneity).
%We discuss the PDMS device that we used, the software expansions that simplified and enabled analysis of our time lapse data, and 
%the application of cross-correlation analysis to branched lineage data that is acquired from growing microcolonies of cells.

In \textbf{chapter \ref{chapter:filarecovery}} we discuss 
novel findings regarding 
the regulation of cell divisions during adverse conditions.
%
Instead of growing \ecoli in favourable conditions, as is often done in the lab, we subjected the bacteria to 
real-world conditions like antibiotic exposure and high temperatures.
%
The cells responded to these adverse conditions by halting the cell division process, whilst they continued to grow.
This process is called filamentation and it leads to typical elongated cell morphologies.
%
%Importantly, when we switched growth conditions back to favourable conditions,
%the cells started dividing again to eventually recover their normal bacillary form according to a very specific set of rules.
%%
%The cells continuously re-arrange potential division sites (Fts rings), to place them at relative locations that depend on the total length of the bacterium.
%
Importantly, when we switched growth conditions back to favourable conditions,
the cells started dividing again to eventually recover their normal bacillary form.% according to a very specific set of rules.
%
During this process,
the cells continuously re-arrange potential division sites (Fts rings), 
to place them at locations along the cellular axis depending on a very specific set of rules. % that depend on the total length of the bacterium.
%
Where Fts rings formed depended on the length of the individual bacteria,
and we showed that this placement was regulated by Min proteins.
%Min proteins were hitherto only known for regulating division site placement in bacteria with the normal bacillary morphology. 
%
We also showed that the timing of divisions was set by the so-called adder mechanism.
This means bacteria on average divide each time they have grown by a specific volume.
%This is remarkable, since the adder mechanism is usually considered 
%with regard to 
%size regulation of bacillary shaped bacteria,
%but during divisions of filamentous bacteria the size of offspring is determined by the location of divisions.
%
The observations on the Min regulation and adder mechanism were remarkable,
since these systems were hitherto only known for regulation of division in bacillary shaped bacteria.
%
Taken together, the results in chapter \ref{chapter:filarecovery} indicate that E. coli cells continuously keep track of absolute length to control size, 
%and suggested a wider relevance for the adder principle beyond the control of normally sized cells,
suggest a wider relevance for the adder principle 
and provide a new perspective on the function of the Fts and Min systems.



In chapters \ref{chapter:literaturereview}-\ref{chapter:ribosomes} we investigate the origins of cellular heterogeneity.
%
As described in chapter \ref{chapter:introduction}, 
%it was recognized decades ago that a 
individual bacteria in an isogenic population can show different behaviour,
which ultimately stems from the stochastic nature of 
chemical reactions that go on inside the cells.
%
Research into this heterogeneity often either focuses on processes where stochasticity in a process can be directly linked to a phenotypic effect
or focuses on how signalling can be robust despite noise.
%
How noise transmits through large biochemical networks, and what effects this has on the cellular state, is researched less often.
%
In
\textbf{chapter \ref{chapter:literaturereview}} we review literature that focuses on this question,
with a specific focus on the metabolism, which is an important large biochemical network in the cell.
%
In \textbf{chapter \ref{chapter:CRP}} we focus 


\chapter*{Bla}


The role of stochasticity in metabolic processes and networks 

In \textbf{chapter \ref{chapter:literaturereview}}, recent studies that focus on 
\textbf{Chapter \ref{chapter:literaturereview}} reviews literature 




\textbf{Chapter \ref{chapter:literaturereview}} is a literature study on the role of stochasticity in metabolic 

 





\chapter*{Samenvatting}