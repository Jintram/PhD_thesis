

%In this thesis, we try to unravel mysteries at the single cell level.
%%
%We cover three topics, 

%In this thesis, we cover three topics that all relate to the observation and behaviour of single cells.
In this thesis we observe and investigate single cell behaviour.  
%
We cover three topics that relate to division, metabolism and translation.
%These topics relate to three important cellular processes respectively: division, metabolism and translation.
%
We first discover how cells strictly regulate their division process during recovery from stress.
%
Then, we 

\section{bla}

We first find out that cells 



In this thesis we seek a better understanding of what goes on at the single cell level.


THIS WORK IS STRUCTURED IN X CHAPTERS


%a more extended version:
% BEGIN
\section{extended description of methods chapter}
We first describe in \textbf{chapter \ref{chapter:methods}} how we expanded on current methods to investigate the behavior of single cells.
%
Specifically, we discuss how a new PDMS microfluidic device (developed by Daan Kiviet) to obtain single cell lineage data from growing microcolonies of \textit{Escherichia coli}, and what modifications were required to apply existing analysis algorithms (designed for the analysis of microcolonies grown on gel pads) on the data from this device.
%
We also introduced two simple new algorithms to process the data gathered in this thesis:
a new tracker algorithm and an algorithm to straighten cells and calculate fluorescence intensity along their axes. 
This latter algorithm was later applied to track the location of labeled division rings and nucleoids in chapter \ref{chapter:filarecovery}.
%
Lastly, this methods briefly discusses the topic of cross correlations,
and describes how to create composite cross-correlations from a branched tree structure such as lineage data acquired from microcolonies of growing cells.
%
This is relevant for chapters \ref{chapter:CRP} and \ref{chapter:ribosomes}, where cross-correlations are used to understand the dynamics between 
single cell gene expression and growth.
% END OF METHODS




%In \textbf{chapter \ref{chapter:filarecovery}} we used single cell experiments 
%to reveal an intricate regulation of cellular division locations in filamentous cells.
%%
%Filamentation, the interuption of cellular divisions without stopping cell growth, leading to elongated morphologies,
%is thought to be 





% PERHAPS A TOO LONG VERSION OF THE CRP CHAPTER
Specifically, we look at the 
cAMP receptor protein (CRP), 
which is a master regulator of metabolic enzyme expression.
Metabolic enzymes are involved in breaking down carbon food sources such as sugars, which results in metabolites that can be used as cellular building blocks, 
and the cellular energy carrier molecule ATP.
CRP is activitated by the regulatory molecule cyclic adenosine monophosphate (cAMP). 
cAMP synthesis (and thus, indirectly, CRP activation) is controlled by negative feedback from carbon metabolite concentrations.
These regulatory interactions constitute a metabolic negative feedback loop that was previously reported to result in optimal metabolic enzyme expression in most cases.
%
Given the hypothesis described in chapter \ref{chapter:literaturereview} that metabolite levels continuously fluctuate in a single cell, even in steady state, 
we speculated that the CRP regulatory system must also continuously receive different inputs, to which it might respond.
%
We employed a fluorescent reporter to be able to measure CRP activity.
In our first experiments, we subjected cells unable to produce cAMP --- which thus lacked the metabolic feedback --- 
to pulses of low and high concentrations of cAMP in the growth medium.
%
These experiments showed that the CRP system can respond to input signals on fast timescales. 
This is consistent with the hypothesis that the CRP system might respond to stochastic fluctuations, 
which are also typically associated with fast timescales.
%Fast timescales are typically associated with stochastic fluctuations.
%
We then further tested this hypothesis by 
again 
employing a fluorescent reporter for CRP activity, 
in order to compare the dynamics of single cell CRP activity and growth in 
wild type cells versus cells from which the metabolic feedback was removed.
%
We grew the cells without metabolic feedback on appropriate cAMP concentrations to ensure wild type like growth rates.
%
We used cross-correlations to assess the interplay between single cell CRP activity and growth rate,
which showed that cells without feedback showed markedly different dynamics than the wild type cells.
%
This suggested that the feedback indeed responds to stochastic fluctuations that occur within the cell.
%
We use a model to further interpret the dynamics revealed by the cross-correlations.
%
Taken together, these experiments show that regulation networks might be continuously responding to stochastic fluctuations in the cell.
%
One cou


% MORE OTHER STUFF


\section{blae}

from which we removed the metabolic feedback 
%
We then tested 
whether the CRP system responds to stochastic input by 
%this hypothesis by 
employing a fluorescent reporter for CRP activity, 
and comparing the dynamics of CRP activity and growth in 
wild type cells versus cells from which the metabolic feedback was removed.
%


%We tested this hypothesis by 
%comparing the dynamics of single cell CRP activity and growth in
%wild type cells 


\chapter*{Bla}


The role of stochasticity in metabolic processes and networks 

In \textbf{chapter \ref{chapter:literaturereview}}, recent studies that focus on 
\textbf{Chapter \ref{chapter:literaturereview}} reviews literature 




\textbf{Chapter \ref{chapter:literaturereview}} is a literature study on the role of stochasticity in metabolic 

% STUFF


Intrinsic noise refers to noise that originates in the gene expression itself, resulting in independ fluctuations,
whilst extrinsic noise refers to 

affects all gene expression at once, resulting in concerted concentration fluctuations

Both refer to noise sources that result in stochastic fluctuations of gene expression,
intrinsic noise refers to fluctuations that arise in a single gene,

% ELOWITZ DEFINITION FROM ABSTRACT
Both stochasticity inherent in the
biochemical process of gene expression (intrinsic noise)and fluctuations in
other cellular components (extrinsic noise)contribute substantially to overall
variation.

intrinsic noise refers to fluctuations caused by noise in the gene itself

extrinsic noise affects all gene expression at once, resulting in concerted concentration fluctuations,
intrinsic noise originates 
intrinsic noise refers to noise that independently affects 

whilst noise that acts 
whilst noise that results in independent fluctuations of the expression of intrinsic noise 

affects 
refers to concerted concentration fluctuations that affect all cellular proteins at once, 
Extrinsic noise referred to stochastic 



\section{bla}


and measured correlations between fluctuations of ribosomal expression and the expression of other proteins.
%
When fluctuations in ribosomal concentrations would affect the expression of all proteins, we 

We expected that if fluctuations in ribosomal affect the expression of 
If ribosomal 
%
We also 


It was observed almost two decades ago 
INTRINSIC/EXTRINSIC NOISE
this suggested 
THAT THERE MUST BE COMPONENTS ..
WE HERE CHECK WHETHER THESE COMPONENTS ARE RIBOSOMES
%
%
It is speculated that there are cellular components that might transmit and/or amplify stochastic fluctuations in cells, 
causing the concentrations of all cellular proteins to fluctuate in concert.
%
%
%
Specifically, it has often been suggested in literature that fluctuations in the ribosomal concentration of cells might contribute to 
concerted fluctuations of the concentration of all proteins in the cell. 
%



%In most conditions,
%we found positive correlations between the ribosomal protein concentration and constitutive reporter concentration,
%but correlations were not more pronounced between ribosomal concentration and constitutive concentration at later times.
%%
%Since extrinsic noise would result in correlations between any two reporters, we concluded that there was insufficient evidence to conclude ribosomal
%fluctuations have an effect on protein expression.
%%
%Also 
%
%Also correlations that we calculated between single cell ribosomal concentrations and growth rate 
%
%The experiments in this chapter yielded only limited data, which made it difficult to draw definitive conclusions.
%
%
%We end this chapter 
%by asking whether a single labeled ribosomal proteins is good proxy for the concentration of 
%completely assembled active ribosomes, 
%%active ribosomal complexes, 
%%We end this chapter with reflections on how well 
%and a suggestion for a future experiment to better assess transmission of ribomal RNA fluctuations to growth rates.





% EXTRINSIC INTR. NOISE
Almost two decades ago, the terms extrinsic and intrinsic noise were coined.
%
%This allows the distinction between noise originating in the gene expression itself (intrinsic noise)
%and fluctuations in cellular components resulting in concerted fluctuations in the expression of all genes at once (extrinsic noise).
These terms refer to noise originating from the biochemical processes of gene expression itself (intrinsic noise), resulting in independent expression fluctuations,
and fluctuations in cellular components that transmit to the expression of all genes in a cell at once (extrinsic noise), resulting in concerted expression fluctuations \cite{Elowitz2002}.
%in concerted fluctuations in the expression of all genes at once (extrinsic noise) .





