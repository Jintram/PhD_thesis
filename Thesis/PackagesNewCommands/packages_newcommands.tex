% Define new convenient shortcut commands and include packages

% *******************************************
% include extrapackages
% *******************************************
\usepackage{soul}				% yellow highlighting \hl{}
\usepackage{textgreek}			%greek upright letters in text mode
\usepackage{xifthen}			%if statements, optional arguments in newcommand
\usepackage[export]{adjustbox}  %align images withing figure environment (left,right)
%\usepackage{endnotes}			%doesn't really work
\usepackage{placeins}			% prevent floats from floating behind the chapter end 	
%\usepackage{booktabs}			% nicer tables
\usepackage{tabu}				% thick lines in tables
\usepackage{colortbl}			% colored cells in tables
\usepackage{bm}					% bold math symbols
\usepackage{mdwlist}			% compact (itemize) lists: itemize*
\usepackage{float}				% enforce float-figure position, e.g. [H]
\usepackage{xfrac}				% small (inline) fractions (\sfrac)

% *******************************************
% Layout
% *******************************************
\renewcommand{\footnotesize}{\small}	% change footnotesize (increase to "small")
\newcommand{\defaultlinesep}{\tabulinesep = 0.8mm} % define a height separation for tables (tabu package)
\newcommand{\cellgray}{\cellcolor[rgb]{0.85 0.85 0.85}} % background color in tables
\newcommand{\cellbright}{\cellcolor[rgb]{0.92 0.92 0.92}} % background color in tables



% *******************************************
% Correction
% *******************************************
% red text
\newcommand{\red}[1]{\textcolor{red}{#1}}
% red text in brackets
\newcommand{\bred}[1]{\textcolor{red}{(#1)}}
% red text with "ToDo:"
\newcommand{\todo}[1]{\textcolor{red}{ToDo: #1}}
% red text
\newcommand{\blue}[1]{\textcolor{blue}{#1}}
% missing reference. set argument to empty if ref. not known
\newcommand{\missref}[1]{\ifthenelse{\equal{#1}{}}{\hl{[ref?]}}{\hl{[ref: #1 ]}}}

% Alternative: \missref and \missref[text]
%\newcommand{\missref}[1][]{%
%	\ifthenelse{\equal{#1}{}}{\hl{[ref?]}}{\hl{[ref: #1 ]}}%
%	}



% *******************************************
% Frequently used expression
% *******************************************
% Ecoli
\newcommand{\ecoli}{\textit{E. coli }}
% P_lac
\newcommand{\plac}{\ensuremath{\text{P}_{\text{lac}}\text{ }}}
% P_N25
\newcommand{\pntwofive}{\ensuremath{\text{P}_{\text{N25}}\text{ }}}
% P_lambda
\newcommand{\plambda}{\ensuremath{\text{P}_{\lambda}\text{ }}}
% P_rrn
\newcommand{\prrn}{\ensuremath{\text{P}_{\text{rrn}}\text{ }}}
% P_rrn-GFP
\newcommand{\prrnGFP}{\ensuremath{\text{P}_{\text{rrn}}\text{-GFP }}}
% lac in italics
\newcommand{\lac}{\textit{lac }}
% crosscorr: R_pmu(tau) 
\newcommand{\Rpmut}{\ensuremath{R_{p\mu}(\tau)} }
% crosscorr: R_pmu
\newcommand{\Rpmu}{\ensuremath{R_{p\mu}} }
% crosscorr: R_Emu(tau) 
\newcommand{\REmut}{\ensuremath{R_{E\mu}(\tau)} }
% crosscorr: R_Emu
\newcommand{\REmu}{\ensuremath{R_{E\mu}} }
% . [point] with hspace before (for equations)
\newcommand{\spacepoint}{\hspace{5px} .}
% , [comma] with hspace before (for equations)
\newcommand{\spacecomma}{\hspace{5px} ,}
 

% (A) etc for subfigures
\newcommand{\figA}{(\textbf{A}) }
\newcommand{\figB}{(\textbf{B}) }
\newcommand{\figC}{(\textbf{C}) }
\newcommand{\figD}{(\textbf{D}) }
\newcommand{\figE}{(\textbf{E}) }
\newcommand{\figF}{(\textbf{F}) }
\newcommand{\figG}{(\textbf{G}) }
\newcommand{\figH}{(\textbf{H}) }
\newcommand{\figI}{(\textbf{I}) }
\newcommand{\figJ}{(\textbf{J}) }

% --------
% Chapter6: Growth Noise
% --------
% short cut for "Extended Data" -> leaves option to change this section name/reference later on
\newcommand{\ED}{Extended Data }
\renewcommand{\NG}{\ensuremath{N_G} } % noise sources in model [this specific command already existsed for some symbol]
\newcommand{\NE}{\ensuremath{N_E} }
\newcommand{\Nmu}{\ensuremath{N_{\mu}} }
\newcommand{\NX}{\ensuremath{N_X} }
\newcommand{\TEE}{\ensuremath{T_{EE}} }	% transmission coefficients in model
\newcommand{\TmuE}{\ensuremath{T_{\mu E}} }
\newcommand{\TEmu}{\ensuremath{T_{E\mu}} }
\newcommand{\TEG}{\ensuremath{T_{EG}} }
\newcommand{\TmuG}{\ensuremath{T_{\mu G}} }


% --------
% Chapter4: Extrinsic Noise
% --------
% corr: R(Y,mu)
\newcommand{\corrYmu}{\ensuremath{R(Y,\mu)} }
% corr: R(C,mu)
\newcommand{\corrCmu}{\ensuremath{R(C,\mu)} }
% corr: R(Y,C)
\newcommand{\corrYC}{\ensuremath{R(Y,C)} }
% noise: n(Y)^2 (maybe modify to underscript Y or brackets, or even P)
\newcommand{\noiseY}{\ensuremath{\eta_Y^2} }
% noise: n(C)^2 (maybe modify to underscript C or brackets, or even P)
\newcommand{\noiseC}{\ensuremath{\eta_C^2} }
% noise: n(mu)^2 (maybe modify to underscript mu or brackets)
\newcommand{\noisemu}{\ensuremath{\eta_{\mu}^2} }
% <mu>
\newcommand{\mumean}{\ensuremath{\langle \mu \rangle} }
% standard deviation: sigma_Y (could still be modified to sigma(Y))
\newcommand{\sigmaY}{\ensuremath{\sigma_Y} }
% standard deviation: sigma_C (could still be modified to sigma(C))
\newcommand{\sigmaC}{\ensuremath{\sigma_C} }
% standard deviation: sigma_mu (could still be modified to sigma(mu))
\newcommand{\sigmamu}{\ensuremath{\sigma_{\mu}} }
% extrinsic noise (squared)
\newcommand{\noiseextr}{\ensuremath{\eta_{extr}^{2}} }
% intrinsic noise (squared)
\newcommand{\noiseintr}{\ensuremath{\eta_{intr}^{2}} }
% noise n(E)^2 (general expression noise -> meybe use Y directly)
\newcommand{\noiseE}{\ensuremath{\eta_{E}^{2}} }
% global noise intensity <N_G^2>
\newcommand{\VarNG}{\ensuremath{\langle N_G^2\rangle} } 
% protein specific noise intensity <N_P^2>
\newcommand{\VarNP}{\ensuremath{\langle N_P^2\rangle} } 
% mu in covariance decomposition (allows to switch between mu and mu^H)
\newcommand{\mucond}{\ensuremath{\mu}}%{\ensuremath{\mu^H}}
% F_extr,E (extrinsic explained fraction) (the E could be removed)
\newcommand{\Fextr}{\ensuremath{F_{extr,E}} }
% Expression noise - vs - Growth
\newcommand{\noiseEvsMu}{\ensuremath{\eta_{E}^2 \text{ vs. }\mu} }