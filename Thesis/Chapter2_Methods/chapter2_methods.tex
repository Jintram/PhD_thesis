



% Some convenient references:
% 
% (Dunlop2008) elowitz lab on cross-corrs to measure regulatory interactions
% (Young2012) single cell tracking from the elowitz lab

\hyphenation{mi-cro-col-o-nies}

\chapter{Methods}
\label{chapter:methods}

\section{Introduction}

A substantial part of the methods used for experiments performed in this thesis were already discussed in detail, either in publications from other labs \cite{Dunlop2008,Young2012} or in previous work from the Tans lab at AMOLF \cite{Kiviet2010, Walker2016t, Gude2016}. An additional independent method section can be found in chapter \ref{chapter:filarecovery} of this thesis as that chapter has been published in a scientific journal. % fingers crossed

In this chapter, I will focus on how I further developed these existing experimental protocols and analyses. %, often building on what has previously been developed. 
Importantly, we have recently employed an adapted "mother machine", a microfluidic device that allows growing and observing bacterial microcolonies for very long time as superfluous cells are washed away, and also allows for quick exchange of growth media.

\section{Single cell experiments}

Observing growing single cells under the microscope provides a novel perspective compared to measuring bacterial behavior in bulk.
It can quantify single cell deviations from the mean behavior (see e.g. \cite{Elowitz2002, Kiviet2014} and chapters \ref{chapter:CRP}, \ref{chapter:ribosomes}), and it provides insights in single cell morphologic changes over time (of which the filamentation and division processes are good examples, see chapter \ref{chapter:filarecovery}).
%
One straightforward way to visualize growing microcolonies of cells under the microscope is by using gel pads \cite{Elowitz2002, Dunlop2008, Dong2010, Kiviet2010, Young2012, Kiviet2014} (Figure \ref{fig:mm:devices}.A).
Pads can be produced by adding agarose to the desired growth medium, solidifying the medium into a gel.
Since it was found cells can also consume agarose, we employ polyacrylamide gel pads in our experiments \cite{Kiviet2010}. The gel pads are soaked in the desired medium, and then transfered to a glass slide. An airtight chamber is created by using a second glass slide with a hole in the middle, a cover slip, silicon grease and a metal scaffold (Figure \ref{fig:mm:devices}.A); for a detailed protocol see \cite{Walker2016t} and \cite{Young2012}.

The disadvantage of growing cells on gel pads is that only a limited number of generations of cells can be recorded.
At some point either (a) cells grow to such high densities that they form multilayers and single cells cannot be distinguished or (b) nutrients provided by the pad run out and cell growth stops.

\subsection{Microfluidic devices}

This advantage can be overcome by using a microfluidic device.
In a microfluidic device, fresh medium is typically pumped through the sample, such that bacteria have continuous access to fresh medium. Additionally, most designs allow for superfluous cells to be washed out of the sample, such that microcolonies can be observed for very long times (hundreds of generations, see e.g. \cite{Wang2010}).
%
Many microfluidic devices have been developed and employed recently \cite{Locke2009, Hammar2014, Hansen2015, Bennett2010, Nanatani2015, Hashimoto2016, Lambert2014, Ullman2013, Young2013, Ferry2011, Wang2010}.

Some experiments described in this thesis (see e.g. chapter \ref{chapter:filarecovery}) were performed with a microfluidic device as depicted in Figure \ref{fig:mm:devices}.B.
This device has a flow channel on its bottom. Bacteria are pipetted on a microscope cover slide, and covered by a very thin membrane. The microfluidic device is placed on top, such that fresh medium flowing through the channel can reach the bacteria under the thin membrane.
The advantage of this device is that all bacteria have equal and close access to the fresh medium. Disadvantages of this device are that continuous growth eventually leads to multilayered colonies that cannot be analyzed (i.e. superfluous cells are not removed).
Additionally, in practice, this device is prone to leakage.
A more detailed description of this device can be found in \cite{Boulineau2013}.

\subsection{The microfluidic device used in this work}

% More info on the microfluidic device by Daan:
% ===
% 
% I've created a pair of new wafers, of which one seems to be working well for E coli in our lab. It's a bit lower than 0.9um (~0.77um) but we are having better experiences with that (a little harder to get cells in, but then no flush-outs). I am in the process of making an epoxy mold of this wafer (essentially a copy of this silicon wafer in epoxy, which works the same as the silicon wafer). I'll make a test flowcell from it early next week, and if that's successful will send both of them immediately to AMOLF. 
% !!!!!!!!!!!!!
% The design on the wafer contains 4 replicate flow channels. Each of them contains a 200um wide main flow channel, splitting into 2 100um flow channels. Perpendicular on these are 5 times repeated blocks with width: 1x 80um, 1x 60um, 2x 40um, 3x 20um, 3x 10um + 3x 5um, and depths of 60um, 30um, 50um and 40um.
% !!!!!!!!!!!!!
%
% ===
%
%I looked up the measurements of the wafer: main flowchannel was 23.5 um and growth channels 0.75um (the 0.77um I sent earlier was from a different part of the wafer). I believe that dimensions can change slightly during transfer to epoxy.
%
%I made the epoxy using the attached protocol (with epoxy R614 and R123).


Most work with microfluidics in this thesis was however done with a more recent microfluidic design.
This device was developed by Daan J. Kiviet (unpublished), it is similar to the "mother machine" described in ref. \cite{Taheri-Araghi2014} but it has wider microcolony wells. As shown in Figure \ref{fig:mm:devices}.C, this design contains 4 replicate flow channels.
Specifically, each flow channel contains a 200 $\upmu$m wide main flow channel, splitting into two 100 $\upmu$m flow channels. 
Perpendicular on these are chambers with the following widths: 
1x 80 $\upmu$m, 
1x 60 $\upmu$m, 
2x 40 $\upmu$m, 
3x 20 $\upmu$m, 
3x 10 $\upmu$m and 
3x 5 $\upmu$m; and with depths of 60 $\upmu$m, 30 $\upmu$m, 50 $\upmu$m and 40 $\upmu$m (Daan Kiviet, personal communications).
These blocks forms chambers that are .75 $\upmu$m high, also referred to as "wells", in which bacterial microcolonies can grow. The main channel is 23.5 $\upmu$m high (heights reported here refer to the original wafer).

\subsubsection{Epoxy mold fabrication}

The PDMS devices were made by casting them into an epoxy mold, which was a gift from Daan J. Kiviet and the Ackermann lab. 
The production of the epoxy mold was performed following an online protocol \cite{EstevezTorres2009}. 
Briefly, this protocol involved creating a copy of the original mold using R123 epoxy resin and R614 hardener. 
The epoxy mixture was prepared as described by the manufacturer, and bubbles were removed by vacuum pumping (ultrasonification is an alternative technique).
The original PDMS cast from the wafer mold was placed in a container, and the epoxy mixture was poured on top (approximately 5mm thick).
The cast was left overnight at room temperature, the PDMS was then removed (using scalpel and tweezers), after which the mould was baked at 70 C for one day, and it was then left for another day at room temperature to harden.
It was then cleaned with plasma and silanized using a trichloromethylsilane-saturated atmosphere for 5 min.

\subsubsection{PDMS device fabrication}

To produce the polydimethylsiloxane (PDMS) device, polymer and curing agent (Sylgard 184 elastomer, Dow Corning Corp.) were mixed in a 50mL Falcon tube using a polystyrene dinner fork (product nr. 888223, Bright Packaging) and a vortex mixer, using 1 mL of curing agent for each 7.7 g of polymer (i.e. not the recommended 1:10 ratio).
Then, this mixture was cast into an epoxy mold (made by Daan J. Kiviet) that is a positive copy of the original wafer mold (sometimes dust was removed from the mold by pressured Nitrogen gas).
Air bubbles were removed from the mixture either by putting the mold and casting in a dessicator for 30 minutes, or by leaving the mixture for several hours before casting.
The mold and casting were then put in a 80 C oven for 1-12 hours. 
Subsequently, the casting was removed from the device, and holes were punched for the liquid in- and outlets.
The device was cut into a smaller size using a scalpel to remove rough, raised or uneven edges.
Then the PDMS device was covalently bound to a clean glass cover slip by treating the PDMS and glass surface with a portable corona device \cite{Haubert2006} (5-10 sweeps of approx. 5 seconds for each surface from approx. 5-10mm distance).
The device was gently tapped using a gloved finger to improve contact between the PDMS and glass surfaces.
(During this procedure the PDMS was only handled with clean metal pincers, a scalpel, or gloved hands.)
Consecutively, the device was baked for another 1-12 hrs. We noticed bonding continued to improve during storage at RT for 1-2 weeks after completion of this protocol .

\subsubsection{Inoculation of bacteria into the device}
To inoculate bacteria into the device, 2ml culture of \textit{Escherichia coli} is grown to high OD ($>$1) in a 10 ml Falcon culture tube on a rotator, either at 37C or 30C (O/N). 
The concentration of bacteria is further increased by spinning down 1ml of the sample in an eppendorf tube at 2300-16100 RCF, and removing supernatent such that the concentration is increased by a factor of approximately 30. The sample is then resuspended.
%
First, 1 $\upmu$L of sterile 0.01\% Tween ($\text{H}_2\text{O}$) solution is slowly pipetted (the plunger is pressed down in approximately 5-10 seconds) into the PDMS device trough one of the holes.
Similarly, 1 $\upmu$L of the concentrated culture is then introduced in the device.
%
Alternatively, a syringe attached to a piece of polyethylene tubing (Fine Bore Polyethyline Tubing, 0.55mm inner diameter, 0.96mm outer diameter, Smiths Medical), attached to a small metal tube (outer diameter approx. 0.65mm), which is then inserted into the device at one of the holes can be used to very slowly introduce first Tween solution and then the condensed bacterial culture into the device.

\subsubsection{Setting up the experiment}
Once the device is inoculated with bacteria, it can be placed under the microscope (usually placed in the same metal scaffold as used for the gel pad experiments, Figure \ref{fig:mm:devices}.A). 
Two small metal tubes (outer diameter approx. 0.65mm) are then inserted in the holes of the PDMS device, and used as connectors to connect polyethylene tubing (Fine Bore Polyethyline Tubing, 0.55mm inner diameter, 0.96mm outer diameter, Smiths medical International Ltd.).
One of the tubes is in turn connected to a syringe (either 10ml or 50ml) placed in a microfluidic pumps (ProSense NE-1000 and NE-300) containing the desired growth medium.
The other end of the other tube is placed in a waste collection erlenmeyer flask.
When desired, instead of one pump with a medium syringe, two pumps each containing a syringe with different medium and an automated valve (Modular valve positioner, RS232, Hamilton) can be used to quickly switch between two media.
Typical flow rates are between 0.5 and 1.0 mL/hr.

%gel pad
%===
%Dunlop et al. 2008; 
%Elowitz, Levine and Siggia 2002; 
%Dong et al. 2010; 
%Young et al. 2012; 
%Kiviet et al. 2014; 
%>>> Elowitz2002, Dunlop2008, Dong2010, Young2012, Kiviet2014
%
%
%microfluidic
%===
%Locke and Elowitz 2009; 
%Hammar et al. 2014; 
%Hansen et al. 2015; 
%Bennett and Hasty 2010; 
%Nanatani et al. 2015; 
%Hashimoto et al. 2016; 
%Lambert and Kussel 2014;
%Ullman et al. 2013; 
%Young et al. 2013; 
%Ferry et al. 2011; 
%Wang et al. 2010;
% >>> Locke2009, Hammar2014, Hansen2015, Bennet2010, Nanatani2015, Hashimoto20126, Lambert2014, Ullman2013, Young2013, Ferry2011, Wang2010

% note that this was developed by Daan Kiviet
\begin{figure}
	\centering
	\includegraphics[width=0.9\textwidth]{thesis_devices-01.png}
	\caption{ 
		(A) Schematic view of a sample that employs a gel pad to grow cells. Two metal parts (grey) hold the glass slides together using screws. Bacteria are pipetted onto a gel pad (red arrow) which is placed in a chamber build from a glass microscope slide, a glass slide with a hole, and a cover slip. The image is placed upside down in the microscope, the eye indicate the direction from which the microscope will image the sample. Image by Noreen Walker.
		(B) Cartoon (not on scale) of a microfluidic device used in some experiments in this thesis. 
		The asterisk indicates bacteria are grown in between a cover slip (shown at the bottom) and a thin polyacrylamide membrane (shown in the middle).
		On top a PDMS slab is placed, which has a flow chamber on its bottom (blue square and zoom), through which medium is flown using two holes (red arrows) that are connected to tubing. 
		This fresh medium can access the bacteria by diffusion through the thin membrane. 
		The 30 by 3 mm chamber contains pillars placed on a 0.6mm grid (here only a few are shown for illustration purposes) to keep the thin polyacrylamide membrane in place. 
		(C) A more advanced microfluidic device employed in this work was designed by Daan J. Kiviet. 
		The mask shown here at the top is used to produce PDMS slabs with four identical growth medium channels, which each also fork into two main channels. 
		These channels (which are 23.5 $\upmu$m high) have protrusions on the side that are less high (0.75 $\upmu$m), which we refer to as wells.
		The PDMS slab is covalently bound to a glass cover slip, and after the inoculation procedure (see main text) single layered microcolonies can grow in the wells.
		Medium is flown through the main channels such that cells receive fresh medium. 
		When the colonies grows and divides, superfluous cells are washed away at the side of the well that faces the main channel.		
		(D) A picture of a microfluidic PDMS device as described in panel C.
	}
	\label{fig:mm:devices}
\end{figure}


\begin{figure}
	\centering
	\includegraphics[width=0.9\textwidth]{thesis_microscope_setup-01.png}
	\caption{ 
	(A) Microscopy setup. One syringe pump (see label 1) pumps culture medium into the sample on the microscope stage in the temperature chamber and then to a waste collection Erlenmeyer flask (2), the other pump pumps medium directly to a waste Erlenmeyer. The two pumps in combination with an automated valve (3) that controls which medium goes to the sample and which directly goes to the waste flask allow for quick switching between the two media.
	(B) The microfluidic device under the microscope. 
	Medium is supplied to the sample (1) through a polyethylene tube (2) and a metal connecting tube (3). Waste medium is disposed through another polyethylene tube (4).
	}
	\label{fig:mm:microscope_setup}
\end{figure}

\section{Data collection}
As also described in chapter \ref{chapter:filarecovery} and previously \cite{Boulineau2013, Kiviet2014}, 
% verbatim copied
cells were imaged with an inverted microscope (Nikon, TE2000), equipped with 100X oil objective (Nikon, Plan Fluor NA 1.3), 
cooled CMOS camera (Hamamatsu, Orca Flash4.0), xenon lamp with liquid light guide (Sutter, Lambda LS), GFP, mCherry, CFP and YFP filter set (Chroma, 41017, 49008, 49001 and 49003), 
computer controlled shutters (Sutter, Lambda 10-3 with SmartShutter), automated stage (Marzhauzer, SCAN IM 120 x 100) and an incubation chamber (Solent) allowing precise 37 C temperature control. 
% end verbatim copied
An additional 1.5X lens was used, resulting in images with pixel size of 0.0438 $\upmu$m. The microscope was controlled by MetaMorph software (Molecular Devices). 
% Note:
% The microns per pixel are hard coded in the DJK_initschnitz function. The
% values are also listed below for reference:
% hamamatsu camera, setup 1 (new camera, installed in 2014-11)
% p.micronsPerPixel = 0.0438;
% CoolSnap camera, setup 1 (old camera)
% p.micronsPerPixel = 0.04065;
% camera hamamatsu, setup 2 (always same camera)
% p.micronsPerPixel = 0.04312;

\section{Updates to the analysis}

To extract quantitative data from the time lapse movies, further analyses are performed.
Briefly: Cells are first segmented to identify areas that constitute cells in the microscopy images. Then, after individual cells have been identified, the lineages of the cells are tracked over the frames of the time lapse. In subsequent analyses cellular parameters are characterized. Examples of important parameters are length, growth rate and the concentration of fluorescent reporter.

Many algorithms have been developed that can segment microcolony data, track individual cells from one frame to the next, and quantify cellular parameters to a greater or lesser extent. Examples include Schnitzcells \cite{Young2012, Kiviet2014}, Supersegger \cite{Stylianidou2016}, Oufti \cite{Paintdakhi2016}, MoMA \cite{Kaiser2016}, Sachs et al. \cite{Sachs2016}, Nobs et al. \cite{Nobs2014}, MicrobeTracker \cite{Sliusarenko2011, Ullman2013}, CBA \cite{Sadanandan2016} and MAMLE \cite{Chowdhury2013}. 
The employed segmentation procedures range from simple image thresholding to more advanced combinations of image transformations. 
A recent development is the usage of machine learning to effectively segment microcolony data \cite{VanValen2016}.

Throughout this work, we use a custom set of scripts based on the schnitzcells framework, which was developed in the Elowitz lab \cite{Young2012}. 
These custom scripts are a mostly written by Daan J. Kiviet \cite{Kiviet2010}, Philippe Nghe and Noreen Walker \cite{Walker2016t}.
Updates and novel functions are also introduced in this work.
These improvements are described in the sections below.
Additionally, we introduced a script called
\begin{verbatim}
Schnitzcells_masterscript.m,
\end{verbatim}
from which the whole analysis can be run. 
This script interacts with a custom made Microsoft Excel configuration file that holds important analysis parameters (directories, segmentation parameters, fluorescence parameters, etc).
This script also provides a graphical user interface to increase the efficiency of the analysis (Figure \ref{fig:mm:GUI}).
More detailed information about how this "master script" can be run can be found in the comments of the script.
The scripts are available in a git repository at 
\begin{verbatim}
https://bitbucket.org/microscopeguerrillas/schnitzcells_tans.git,
\end{verbatim}
which holds the main files, 
and a git repository at
\begin{verbatim}
https://bitbucket.org/microscopeguerrillas/
schnitzcells_tans_extensions.git
\end{verbatim}
that holds additional files required to run the full analysis.


\begin{figure}
	\centering
	\includegraphics[width=0.9\textwidth]{thesis_guiSchnitzcells-01.png}
	\caption{ 
		(A) Screenshot of the GUI introduced to increase the efficiency of the analysis.
		(B) Screenshot of the Matlab figure with which the user can interact to manually correct the analysis. The segmentation is shown by the colors, and the circles show (corrected and uncorrected) centers of cells from the previous frame.
		% G:\EXPERIMENTAL_DATA_2016\2016-12-08_CRP_asc990_lac\Schnitzcells_Analysis_Config_2016_12_08_pos2.xlsx
	}
	\label{fig:mm:GUI}
\end{figure}

\subsection{Segmentation}

Throughout this work, we use a segmentation algorithm developed by Philippe Nghe \cite{Walker2016t}. Additionally, all frames are manually checked for segmentation mistakes.
However, the Nghe script was developed for data from gel pad time lapse experiments.
Thus, some minor modifications were introduced to handle data from the microfluidic device were cells also disappear from the experimental observations.
%
Importantly, cells are flushed away once they reach the main channel, and thus cells that touch the edge of the image (see Figure \ref{fig:mm:segmentation} for an example of the segmentation output) need to be ignored from the analysis.
The original algorithm could not deal with cells touching the edge of the image. 
As a simple fix, we introduced a gradient at the edge of the image (Figure \ref{fig:mm:segmentation}.A) that goes from transparent to white, see the script 
\begin{verbatim}
MW_preprocessimagefadeedge, 
\end{verbatim}
%\url{},
such that cells touching that edge are considered to have their cell edge at the edge of the image with some margin by the algorithm (Figure \ref{fig:mm:segmentation}.C-D).
This procedure allows processing steps of the algorithm to function as if the data came from a gel pad experiment.
Consecutively, cells to which this was applied are removed from the analysis.

More adjustments were made during the tracking of cells, see next section.

% Original text from literature review manuscript (perhaps an earlier version)
% The fact that single cells can show different behaviour despite being genetically identical and experiencing the same environment has been recognized in studies done long ago (Spudich and Jr. 1976; Powell 1956). Since the days of these early studies, the tools to track single cells over time have vastly improved, it is now possible to trace thousands of cells through hundreds of generations using microfluidic chemostats (Locke and Elowitz 2009; Dunlop et al. 2008; Elowitz, Levine and Siggia 2002; Hammar et al. 2014; Hansen et al. 2015; Bennett and Hasty 2010; Nanatani et al. 2015; Dong et al. 2010; Kiviet et al. 2014; Hashimoto et al. 2016; Lambert and Kussel 2014; Ullman et al. 2013; Young et al. 2013; Ferry et al. 2011; Young et al. 2012; Wang et al. 2010). The modern devices generally consist of PDMS, which is cast into shapes using a silicon wafer and covalently bound to glass cover slip allowing observation of cells under the microscope. Devices can also be cloned cheaply using epoxy molds (Kamande et al. 2015). Most chips designs have a main channel where growth medium flows through, from which chambers with low ceilings extrude. Bacteria are trapped in the chambers, and as they grow, the ones near the exit of the chambers are washed away by the main flow into waste. Different well designs exists where cell are growing either as a single file in thin chambers or as a monolayer in wide chambers (Fig 1). Temperature control, a constant supply of media and  the removal of cells ensure a constant environment in which growing single cells can be observed for up to multiple days. Different designs allow for fast switching of medium and medium gradients. Using time lapse microscopy, single cells can be tracked and analysed.
% Different scripts were developed that recognize cells in the microscopy images (segmentation), and  track individuals from one frame to the next, for example Schnitzcells (Young et al. 2012; Kiviet et al. 2014), Supersegger (Stylianidou et al. 2016), Oufti (Paintdakhi et al. 2016), MoMA (Kaiser et al. 2016), Sachs et al. (Sachs et al. 2016)., Nobs et al. (Nobs and Maerkl 2014), MicrobeTracker (Sliusarenko et al. 2011; Ullman et al. 2013), CBA (Sadanandan et al. 2016) and MAMLE (Chowdhury et al. 2013). Segmentation procedures range from simple image thresholding to more advanced combinations of image transformations. It was recently shown that also machine learning can be used very effectively to segment this type of data (Van Valen et al. 2016). Also sophistication of tracking and data analysis varies from one analysis package to the next. Concluding, it transpires that culturing and computer-aided analysis of single cells is an active field of research, where "golden standards" have yet to crystalize out but many tools are currently available to obtain large quantities of single cell data.

\begin{figure}
	\centering
	\includegraphics[width=0.9\textwidth]{thesis_analyses1-01.png}
	\caption{ 
		 (A) The result of the segmentation algorithm, outlines of determined cell areas are shown in color.  (A frame the nucleoid labeled cells time lapse data from chapter \ref{chapter:filarecovery} is shown.) Images are cropped to increase computational efficiency, and also to remove the main channel structure. Note that cells are generally flushed out when they reach the main channel. Growth results in cells moving towards the well exit at the main channel, or towards the edge of the picture labeled top here. Cells that are partially outside the cropped image are ignored in the analysis.
		 (B) The image in panel A is cropped, this panel shows the full image taken by the microscope. The asterisk indicates the well which is also displayed in other panels.
		 (C-D) Image at the top of the cropped image are artificially considered to have a cell edge that is aligned with the image edge. These cells are later removed from the analysis (here indicated with asterisks). Shown are processing steps early (C) and late (D) in the algorithm. (Note that the artifact shown in red in panel D will also be removed later in the analysis, either automatically or manually.)
		% H:\EXPERIMENTAL_DATA_2017\2017-11-08_FilaRecovery_asc1106_hupA-mCherry\Schnitzcells_Analysis_Config_2017-10-12_hupA-mRuby2_pos1d.xlsx
	}
	\label{fig:mm:segmentation}
\end{figure}

\subsection{Tracking}

Calculating the lineage of cells using the segmented images is an essential part of the desired analysis.
Two tracking functions already existed, and we have introduced a third simple tracker matlab script to these.
%
This was done because both the other trackers relied to methods that represented the location of cells as a point in one way or another (Figure \ref{fig:mm:tracking}.A-B).
In one method, cells from both frames to be connected are represented by three points along their skeleton.
In the second method, the centers of cells from one frame are compared with the cell areas from the other frame \cite{Walker2016t}.
%
In specific cases this leads to issues. For example, if cells have atypical shapes (e.g. due to filamentation), the centers characterize their placement less well.
Therefor the introduced method, see
\begin{verbatim}
MW_tracker.m, MW_linkframes, 
\end{verbatim}
tracks cells by looking at overlap between cell areas.
The cell from frame n+1 is considered to be lineage-connected to that cell from the previous frame n that has most overlap with it (Figure \ref{fig:mm:tracking}.C).

Additionally, during image acquisition the sample can shift tiny amounts (usually less than a micrometer), which is corrected by the Nghe script by aligning the centers of the microcolonies for consecutive frames.
This procedure leads to issues when cells disappear from the analysis, as is the case with the microfluidic device.
(When cells disappear, the centroids do not represent the same set of cells any more, and the procedure can misalign the microcolonies.)
Thus, an additional adjustment was introduced in the tracking script, which aims to shift the joined areas that are recognized as cells from the two frames on top of each other, and maximizes the overlap between these two areas (Figure \ref{fig:mm:tracking}.D).
%RESIZING OF THE FRAMES TO FIT THEM IS ONLY DONE FOR NON-MOTHERMACHINE DATA
Specifically, a binary representation of the image (cells detected or no cells detected for that pixel) is integrated along one axis, such that both the n-th frame and the n+1 frame are represented by a line along the other axis. Then, these lines are shifted until the difference between the lines is minimized along that axis. This is done both for the x and y directions (i.e. both axes).

\begin{figure}
	\centering
	\includegraphics[width=0.9\textwidth]{tracking1.pdf}
	\caption{ 
		(A) Default Schnitzcells tracking algorithm, which characterizes cells by three points along its skeleton. Cells are connected between frames by minimizing the distance between these characteristic points. See \cite{Walker2016t} for more information.
		(B) The algorithm introduced by Noreen Walker, which links cells between frames by minimizing the distances of centroids in the n+1 frame and areas of cells in the n-th frame.
		See \cite{Walker2016t} for more information.
		(C) Algorithm introduced in this work, which links every cell in the n+1 frame to that cell in the n-th frame that has most overlap with the projection of the n+1 frame cell onto the n-th frame.
		In this example, the area of the orange cell in the n+1 frame overlaps with the blue, dark blue, orange, green and purple cells in the n-th frame, but shows most overlap with the orange cell. Thus, the two orange cells in these two consecutive frames are connected.
		(D) An example of the overlapping colony areas from two consecutive frames from a microfluidic device dataset. Overlap between the colonies is indicated in white, non-overlap in gray. Alignment using centroids would fail in this case, since the microcolony centers (shown as circles) do not represent the same subset of cells. For this reason the overlap of these two areas is maximized to align subsequent frames in data from microfluidic devices. (The microcolony shown is the same as shown in Figure \ref{fig:mm:segmentation}.)
		% fr 709, 710 from
		%	H:\EXPERIMENTAL_DATA_2017\2017-11-08_FilaRecovery_asc1106_hupA-mCherry\pos1cropd\
		Images (A) and (B) are made by Noreen Walker.
	}
	\label{fig:mm:tracking}
\end{figure}

\subsection{Skeleton length and straightening of skeleton}

Previously \cite{Kiviet2010}, the length of the cells during the analysis was determined using a third order polynomial fit through the cell area. 
In some cases, e.g. in involving filamentation of cells, a 7-th order polynomial was used.
An alternative way of determining length is using the skeleton of the cell area.
The cellular skeletons are determined in the scripts
\begin{verbatim}
NDL_addToSchnitzes_skeletonLengthMW, NDL_lengthforfillamentedcellsMW,
\end{verbatim}
which were developed by Nick de Lange.
The skeleton is determined using Matlab's 
\begin{verbatim}
bwmorph
\end{verbatim}
function, and after removal of branches, it is extrapolated to the edge of the cell's areas using 0.95 $\upmu$m long windows at both ends of the skeleton (Figure \ref{fig:mm:skeletonstraightening}.A-B). 
Lengths of the skeleton and the extrapolated parts are determined using Matlab's 
\begin{verbatim}
bwdistgeodesic
\end{verbatim}
function (using the quasi-euclidean method).
Additionally, an algorithm was developed to produce straightened representations of the cells using the cellular skeletons. This is useful when for example the localization or intensity of the fluorescence signal along the cellular axis needs to be quantified (this is for example done in chapter \ref{chapter:filarecovery} for fluorescently labeled division rings and nucleoids).
This is done in the script 
\begin{verbatim}
MW_straightenbacteria
\end{verbatim}
by placing lines of equal length tangential to the skeleton, and subsequently placing the image pixels corresponding to each line into the columns of a rectangular matrix (Figure \ref{fig:mm:skeletonstraightening}).

\begin{figure}
	\centering
	\includegraphics[width=0.9\textwidth]{thesis_skeleton_and_straightening-01.png}
	\caption{ 
		(A) Skeleton of an example cell area (gray) calculated by Matlab's \texttt{bwmorph} function, after removal of side branches. 
		(B) The skeleton is extrapolated towards cell poles (black circles) using the ends of the skeleton (purple and yellow dots). The total cell length is calculated by summation of the lengths of the extrapolated and original skeleton parts.
		(C-G) Example of cellular straightening. Based on segmentation (see main text and Figure \ref{fig:mm:segmentation}) both for the phase image (C) and the fluorescence image (D) cells can be straightened to produce a fluorescence intensity profile along the cellular axis. This is done by using a series of lines that are placed tangential to the skeleton (E), as shown here for the cell indicated by the blue arrow in panel (C). The pixels closest to this line (represented by colored dots here) are used to generate the straightened bacteria, both for the phase image (F) and the fluorescence image (G).
		The images shown here are part of the dataset with nucleoid labeled  cells described in chapter \ref{chapter:filarecovery}.
	}
	\label{fig:mm:skeletonstraightening}
\end{figure}

\subsection{Correlation functions, scatter plots, weighing and controls}

\subsubsection{The cross-correlation function}

An important tool to gain insights into the dynamics and interactions between different cellular parameters (such as expression of different proteins and growth rate) is the correlation function.
%
As this function is also central in this thesis, 
We'll discuss some basic definitions regarding this quantity here.
%
Simply put, the aim here is to gain insight in to what extent the value of one signal expected to affect the value of another signal, considering there might be a delay in the effect.
%
The cross-correlation function $R(\tau)$ quantifies to what extent the deviation from the mean value in one parameter $f(t)$ at time $t$ is correlated with the deviation from the mean value in another parameter $g(t+\tau)$, at a delay $\tau$ later (or earlier).
%
Mathematically, this is expressed as \cite[see lemma "Cross-Correlation"]{Weisstein2018}
\begin{align}
	\label{eq:CC}
		S_{f,g}(\tau) = f \star g = \int_{\tau=\infty}^{\infty} {\bar {f}(t) \bar{g}(t+\tau) \delta \tau}
\end{align}
or for a discrete signal the cross-correlation can be defined as \cite{Dunlop2008}:
\begin{align}
	\label{eq:CCdiscrete}
	S_{f,g}(\tau) = \frac{1}{N-|\tau|} 
		\sum_{n=0}^{N-|\tau|-1} {\tilde{f}(n) \tilde{g}(n+\tau)},
\end{align}
where $f$ and $g$ are either continuous and mean-subtracted (indicated by the bar) or discrete and mean-subtracted (indicated by the tilde). 
$N$ is the number of time points in the data series.
%
When this is normalized by 
\begin{align*}
\sqrt{S_{f,f}(0)S_{g,g}(0)}
\end{align*}
the result is often also referred to as the cross-correlation \cite{Munsky2012}:
%
\begin{align}
	\label{eq:R}
R_{f,g}(\tau) 
 	& = \frac{1}{\sqrt{S_{f,f}(0)S_{g,g}(0)}} \frac{1}{N-|\tau|} 
	\sum_{n=0}^{N-|\tau|-1} {\tilde{f}(n) \tilde{g}(n+\tau)} 
	\nonumber \\
	& =	\frac{S_{f,g}(\tau)}{\sqrt{S_{f,f}(0)S_{g,g}(0)}} 
	= \frac{S_{f,g}(\tau)}{\sqrt{\sigma^2_f\cdot\sigma^2_f}}
\end{align}
%
$R_{f,g}(\tau)$ defined in equation \ref{eq:R} is also the function that we call the cross-correlation and use throughout chapters {\ref{chapter:CRP} and \ref{chapter:ribosomes} to quantify relationships between different biological quantities.
%
Note that $S_{f,f}(0)$ also equals the variance in f, $\sigma^2_f$. 
%
When $\tau=0$, the cross-correlation in Equation \ref{eq:R} simply becomes the correlation coefficient $\rho_{f,g}$ (also known as the Pearson's correlation) between two the two parameters $f$ and $g$:
%
%
\begin{align}
	\label{correlationCoefficient}
	\rho_{f,g}=		
	\frac{1}{\sigma_f \sigma_g \cdot N} 
	\sum_{n=0}^{N-1} {\tilde{f}(n) \tilde{g}(n)}
	=
	\frac{\sigma^2_{f,g}}{\sigma_f \sigma_g}
	=
	\frac{cov(f,g)}{\sqrt{var(f)var(g)}},
\end{align}
where $\sigma^2_f$ again indicates the variance of $f$, which can be equivalently written as $var(f)$. ($\sigma_f$ simply indicates the standard deviation, the square root of the variance.) The covariance is indicated by $\sigma_{f,g}$ or $cov(f,g)$.
%
% Mathworld says ss^2_xy/ss_xx*ss_yy
% Note that  
%
% The correlation coefficient $\rho_{x,y}$ is also known as the Pearson's correlation.
%WHAT ABOUT NORMALIZATION
%
These quantities are tightly related to the least squares fitting of b in $g=a+bf$, which is given by \cite[see lemma "Correlation Coefficient"]{Weisstein2018}:
\begin{align}
b = \frac{cov(f,g)}{var(f)};
\end{align}
this also shows that $\rho_{f,g} = \sqrt { b \cdot b' }$ (defining $f=a'+b'g$ additionally to $g=a+bf$).

Note that there is some ambiguity between the terms cross-covariance and cross-correlations when talking about time series data. 
While Munsky et al. \cite{Munsky2012} also define the cross-correlation $R(f,g)$ as we do in equation \ref{eq:R}, Dunlop et al. \cite{Dunlop2008} call the quantity $R(f,g)$ defined in equation \ref{eq:R} the cross-covariance, and call $S(f,g)$ in equations \ref{eq:CC} and \ref{eq:CCdiscrete} the cross-correlation (as also defined in \cite[see lemma "Cross-Correlation"]{Weisstein2018}). To be consistent with the definitions of the Pearson's correlation coefficient and the covariance, we follow the definition as given by Munsky et al. \cite{Munsky2012}, as also employed by Kiviet et al. \cite{Kiviet2010, Kiviet2014}.


% convenient:
% http://mathworld.wolfram.com/CorrelationCoefficient.html
% https://en.wikipedia.org/wiki/Covariance
% https://en.wikipedia.org/wiki/Correlation_and_dependence
% script: general_playingwithcorrelations.m
% furthermore
% http://stats.stackexchange.com/questions/22718/what-is-the-difference-between-linear-regression-on-y-with-x-and-x-with-y/22721#22721
% http://stats.stackexchange.com/questions/32464/how-does-the-correlation-coefficient-differ-from-regression-slope


\subsubsection{Delayed Scatter plots}

Cross-correlations are particularly good at detecting linear relationships.
As an addition to the analysis using cross-correlations, 
it can therefor be instructive to analyze time traces of measured cellular parameters by using scatter plots (Figure \ref{fig:mm:scatter}), 
which might show non-linear relationships between the parameters of interest. 
%
However, cellular parameters are shown to sometimes correlate with delays; it has for example been shown that the production rates of some enzymes have an effect on growth rate at some later point in time \cite{Kiviet2014}.
Thus, simply plotting $f(n)$ and $g(n)$ (which could for example be enzyme production and growth rate, respectively) against each other for all $n$ values might not give the maximum insight in the relationships between the two quantities $x$ and $y$.
Therefor, in the scripts
\begin{verbatim}
MW_delayedScatter, MW_getdelayedscatter
\end{verbatim}
$f(n)$ is plotted against $g(n+\tau)$ in multiple plots, each plot corresponding to one of $(N-1)$ values of $\tau$.
This procedure is illustrated in Figure \ref{fig:mm:delayedscatter}, which shows that the delayed relationship between two parameters becomes clearer when plotting a scatter plot for that specific delay.
% 
Note that we can also determine the correlation coefficient $R(\tau)$ for each of the scatter plots corresponding to a particular $\tau$.
Thus, the scatter plots also provide a way to generate a cross-correlation function.

\begin{figure}
	\centering
	\includegraphics[width=0.9\textwidth]{scatter.pdf}
	\caption{ 
		(A) Example of a linear relationship, and a cloud of points (each point might represent an experimental measurement at time $n$) generated by randomly drawing from that relation (shown in black) and adding noise (normal distribution with $\sigma=1$). The Pearson correlation coefficient $R$ is able to detect the relationship despite the noise.
		(B-C) Example of a non-linear relationship ($g=-7*(f).^2+4$, shown in black), and a cloud of points generated by randomly drawing from that relation and adding noise (normal distribution with respectively $\sigma=.25$ and $\sigma=1$ in panels B and C). Since the relationship is non-linear the Pearson correlation coefficient $R$ is not able to detect the relationship. The gray lines are isolines that represent Kernel Density Estimation of the distribution of points.
		In all panels, $I$ gives the mutual information $I(F;G)$ (based on a Kernel Density Estimation of the probability distribution) of which the non-zero value for panel B shows it is able to detect a dependence between $f$ and $g$ for non-linear relationships (see text). Panel C illustrates that non-linear relationships also quickly get harder to detect, even by eye.
	}
	\label{fig:mm:scatter}
	% D:\Local_Software\Martijn_extensions\Martijn_custom\general_playingwithcorrelations.m
\end{figure}


\begin{figure}
	\centering
	\includegraphics[width=0.9\textwidth]{thesis_delayedscatter-01.png}
	\caption{ 
		(A) Two example noisy signals, $f(t)$ and $g(t)$. $f(t)$ shown in blue is a random walk process, while $g(t)$ shows a random walk plus a contribution by $f(t-50)$, i.e. $g(t)=.8\cdot w(t)+.2\cdot f(t-50)^2$, where $w(t)$ represents a random walk like $f(t)$.
		(B-C) As expected, this imaginary scenario shows that plotting $f(t)$ against $g(t)$ reveals the correlation between $f$ and $g$ poorer compared to plotting $f(t)$ against $g(t-50)$.
	}
	\label{fig:mm:delayedscatter}
	% D:\Local_Software\Martijn_extensions\Martijn_custom\general_playingwithcorrelations.m
\end{figure}


\subsubsection{Mutual information}

%The cross-correlation analyses described above quantify the linear dependence of two parameters. 
%
%As relationships between quantities might also be non-linear, it might be interesting to in the future also look into other measures that quantify 

Though not applied in this work, it might be interesting to provide an outlook on
how one could quantify the extent to which parameters are related (in non-linear ways) beyond the visual inspection of scatter plots, or the determination of the correlation coefficient $R(\tau)$.
%whether one parameter affects the other,
%even if that is in a non-linear manner.
%
%Given non-linear relationships, p
One way to investigate this is to look at the independence of two parameters.
If the biological quantity $x$ does not affect another quantity $y$, 
the probability to find a certain value of $y$ should not depend on the value of $x$, and vice versa.
%
In other words, when the two parameters are independent of each other
$p(x,y)=p(x)p(y)$.
%
To what extent values of $x$ are constrained by values of $y$ (either by a relationship between the two, or because of an indirect link) and vice versa, is quantified by the mutual information \cite{Shannon1948, Bishop2006, WikipediaMutual2017}.
% to define distance of point cloud to simple multiplication of the marginal probability distributions of the two separate parameters $f$ and $g$ of interest.
%A measure that quantifies this distance is the mutual information \cite{Shannon1948, Bishop2006, WikipediaMutual2017}. 
For two random variables $X$ and $Y$, this defines a distance measure between the product of the two marginal probability densities $p(x)$ and $p(y)$ and the joint probability $p(x,y)$.
%This is defined as follows:
%
\begin{align}
	I(X;Y) = 
	\sum_{y} { \sum_{x} { p(x,y) \log_2 \left( \frac{p(x,y)}{p(x)p(y)} \right) }},
\end{align}
or equivalently in continuous form:
\begin{align}
	I(X;Y) = 
	\int_{y} { \int_{x} { p(x,y) \log_2 \left( \frac{p(x,y)}{p(x)p(y)} \right) 
			\,dx\,dy}}
	.
\end{align}
%where $p(x)$, $p(y)$ are the marginal probability distributions of two random variables $X$ and $Y$, and $p(x,y)$ is the joint probability distribution. 
(Note that the probability densities are always normalized to one.)
It has also been shown that the mutual information can be found by 
\begin{align}
	I(X;Y)=H(X)+H(Y)-H(X, Y)
	,
\end{align}
with $H(X)$ being the entropy of a random variable $X$ and $H(X,Y)$ the joint entropy\footnote{See also \url{http://mathworld.wolfram.com/MutualInformation.html}.}.
%
% A note on X and x as stochastic variables:
% https://math.stackexchange.com/questions/435846/notation-of-random-variables
% Basically, X and Y refer to the outcome, 
% whereas x and y refer to the probability distribution
%
When $X$ and $Y$ are independent, the mutual information will be zero, since 
\begin{align*}
	p(x,y)=p(x)p(y)
	.
\end{align*}
%
Conversely, the more $X$ and $Y$ depend on each other, the higher the mutual information.
Figure \ref{fig:mm:scatter} illustrates that this offers a way to quantify to what extent two parameters are related.
%
For our (fake) example data, we have used the 
\begin{verbatim}
kde2d
\end{verbatim}
function written by Zdravko Botev \cite{Botev2010}, a kernel density method, 
to estimate the underlying distributions (i.e. the marginal distributions and the joint distribution).
%However, for experimental datasets, one would need to be certain that enough data is available 
However, to apply this analysis to experimental data, one would need to statistically quantify the validity of the estimated probability distributions, 
or use a statistical method
that provides an estimate of the mutual information directly.
%However, to correctly estimate the underlying probability distribution function, and/or to correctly estimate the mutual information for an experimental dataset, one would need an approach that 
%be sure that enough experimental data is available to correctly further develop


\subsection{Obtaining cross-correlations from experimental data}
\label{sec:methods:CCs}

The experimental nature of the data that we gather presents us with specific challenges regarding the cross-correlation analysis.
%
In this section, we will discuss how we obtain cross-correlations and scatter plots from the experimental data.

\subsubsection{Experiments lead to a branched lineage tree}
In a typical experiment, we measure biological quantities over time in growing and dividing bacteria. 
%
Since we are interested in processes that have time scales similar to or longer than one bacterial life cycle, we measure over multiple generations of bacteria (depending on experimental conditions, the generation time of bacteria is typically 20 minutes to 5 hours).
%
For microcolonies growing on gel pads, this leads to a branched data structure, as was previously described \cite{Kiviet2010,Walker2016t}, see also Figure \ref{fig:mm:weighingCCs}.A. 
% 
For microcolonies growing in the microfluidic device as described above (see also Figure \ref{fig:mm:segmentation}.B), cells disappear from the analysis when they exit the well. This means that the datastructure will also contain lineages that end before the end of the experiment (Fig \ref{fig:mm:weighingCCs}.A, branch 4).
%

\subsubsection{Dealing with redundancy in the branched lineage trees}

The first challenge is to use the time series from each branch of the lineage tree
to
create a composite cross-correlation or scatter plot that represents all data.
%
For each branch, cross-correlations and scatter plots can be determined by taking into account all pairs of points with a specific delay in between them.
However, this would result in pairs of points being used more than once, 
since they appear in multiple branches.
For example, the point pair labeled [1] in Figure \ref{fig:mm:weighingCCs}.A appears in all branches, and would thus be overrepresented if no correction is applied.
%
This can be addressed by only taking into account point pairs that are unique \cite{Dunlop2008}.
%
For the scatter plots, 
when we plot point pairs
$f_i(n)$, $g_i(n+\tau)$ for all time points $n$ and all branches $i$, 
we simply omit point pairs that are duplicates.
%
For the cross-correlation, we follow a similar procedure.
To determine the final composite cross-correlation, we 
employ a weighing scheme that grants less weight to point pairs that are redundant in the dataset.
(This was also described earlier \cite{Kiviet2010,Walker2016t,Dunlop2008}).
% 
Specifically, %we compute
 the cross-correlation as defined in Equation \ref{eq:R} is adjusted to obtain a composite cross-correlation with contributions of points from multiple branches $i$:
\begin{align}
	\label{eq:Rcomposite}
	S_\text{composite,f,g}(\tau) = 
	\frac{1}{  W_{\text{total},\tau}  } 
	\sum_i{
		\frac{1}{  N_i-|\tau|  } 
		\sum_{n=0}^{N_i-|\tau|}{ 
			w_{n,i,\tau} \tilde{f}_i(n) \tilde{g}_i(n+\tau)
		}
	},
\end{align}
with $w_{n,i,\tau}$ a weighing factor that corrects for the redundancy of the specific point pair $\tilde{f}_i(n) \tilde{g}_i(n+\tau)$ (see below for a more detailed discussion of the value of $w_{n,i,\tau}$).
Furthermore, 
%
\begin{align*}
W_{\text{total},\tau} = \sum_{n,i} {w_{n,i,\tau}}
\end{align*}
%
is the sum of all weights for a specific $\tau$ value.
(Because any branch $i$ has $N_i-\tau$ pairs of points, the total weight changes per $\tau$ value.)
$R_\text{composite}(\tau)$ is then again determined by normalizing with
\begin{align*}
\sqrt{S_{\text{composite},f,f}(0)S_{\text{composite},g,g}(0)}
.
\end{align*}
% THE FOLLOWING IS FALSE:
%This normalization factor also includes $w_{n,i,\tau}$ terms, and thus also corrects the normalized composite function for the scaling introduced by the weight terms.
% BECAUSE --> every tau value has its own set of weights, so the total needs to be computed.
%Note that Equation \ref{eq:Rcomposite} does not contain a normalization factor to account for the introduction of the weight term $w_{n,i,\tau}$, since this is accounted for by the normalization.
% 
Also, in this formula, $\tilde{f}_i(n)$ is defined as 
\begin{align*}
\tilde{f}_i(n) = f_i(n)-\left<f_i\right>_n
, 
\end{align*}
i.e. the mean of the value of $f(n)$ for all cells in the colony at that point in time ($n$) is determined and subtracted from the value of $f_i(n)$ to obtain the mean-subtracted value $\tilde{f}_i(n)$ for branch $i$.
%
In the rest of this thesis, we refer to $R(\tau)_\text{composite}$ also simply as
the cross-correlation, $R(\tau)$.
%
The most straight-forward weighing simply aims to achieve what was also done for the scatter plots, i.e. make sure that duplicate point pairs are counted only once.
%
Thus, the most straight-forward choice of $w_{n,i,\tau}$ is 
\begin{align}
\label{eq:weightscomposite}
w_{n,i,\tau} = 1/\lambda_{n,i,\tau}
\end{align}
with $\lambda_{n,i,\tau}$ being the number of times this specific point pair was used.
This method still leaves some redundancy in the data, as there might be point pairs that in which one point of the pair was already used, but the other point is unique (this is illustrated at the label [2] in Figure \ref{fig:mm:weighingCCs}).
Note that the uniqueness of a point pair is determined completely by the point in the pair that was measured at the latest time point in the dataset.
This issue is not straightforward to solve, but
other choices of weights $w_{n,i,\tau}$ 
that try to address this issue
are described in \cite{Kiviet2010} and \cite{Walker2016t}. 
Additionally, one can consider additional weighing schemes, e.g. 
\begin{align*}
	w_{n,i,\tau} = 1/ \left( \lambda_{n,i} \cdot \lambda_{n+\tau,i} \right),
\end{align*}
where $\lambda_{n,i}$ and $\lambda_{n+\tau,i}$ are to the number of times the specific points making up the pair are considered for the composite cross-correlation.
%
Nevertheless, it is hard to choose the perfect weighing scheme, since the nature of the cross-correlation 
requires using pairs of points that are separated by a time delay, 
and ergo pairs might consist of two points that have a different degree of uniqueness.
One could choose to throw away any pair that contains a redundant point, but then one would also throw away (valuable) information.
%
In this work, to be consistent with the scatter plots, we use the weights as described in Equation \ref{eq:weightscomposite}, this is also consistent with \cite{Dunlop2008}.

\begin{figure}
	\centering
	\includegraphics[width=0.9\textwidth]{cc_weighing.pdf}
	\caption{ 
		(A) As cells divide, the data from microcolony experiments will show a branched structure. 
		This leads to redundancy in the data. 
		For example, data recorded from the cell on the left will show up in all branches (1-4) of the lineage structure in this cartoon.
		Additionally, in the microfluidic device described in this chapter, cells are washed away from the growth well, and thus from the experimental observations. 
		This is shown in branch 4 in this cartoon.
		(Note that in a growth well, the number of cells will remain approximately constant, 
		and lineages originate from a few cells that remain at the bottom of the well; not shown here).
		(B) Cartoon representation of two signals that might come from branches 1 and 4. A block function would produce a saw-tooth correlation function (see panel C).
		(C) To determine cross-correlation for these cell branches that are not present during the whole experiment, a weighing scheme is used that does not simply average the cross-correlation of the two separate branches, but instead goes over each pair of points in the branch structure, and weighs the contribution of these two points to the composite cross-correlation based on the redundancy. 
		For example, the combination of points labeled [1] in panel (A) will receive less weight than the combination of points indicated by [2]; the aim is to compensate for the fact that points [1] will appear in the time series of all 4 branches and will thus be considered four times.
		This weighing is complicated by the fact that for some pairs of point, one point is more redundant than the other (this is the case for points [2] in panel (A)). See main text for a more detailed description of the weighing procedure.
	}
	\label{fig:mm:weighingCCs}
	% D:\Local_Software\Martijn_extensions\Martijn_custom\general_weighingCCs_setup.m
\end{figure}

\subsubsection{Dealing with impartial time series}

As mentioned, the experimental data of the microfluidic device also contains lineages that end before the end of the experiment (branch 4 in Figure \ref{fig:mm:weighingCCs}.A illustrate this).
%
Such lineages of course also produce time series that end before the experiment ends (Figure \ref{fig:mm:weighingCCs}.B, bottom panel).
%
The fact that branches are of unequal length can be dealt with in a straightforward manner when computing the composite cross-correlation function.
%
Equation \ref{eq:Rcomposite} allows for the fact that branch lengths $N_i$ are different for each branch $i$. 
%
The weighing terms do not need to be adjusted, as 
$w_{n,i,\tau}$ and $W_{\text{total},\tau}$
can be simply calculated for impartial data series.
%
Figure panels \ref{fig:mm:weighingCCs}.B-C illustrate this procedure.

\subsubsection{Controls}

Now that we have a way to determine cross-correlations for experimental data,
an important question that remains is "is the signal that we observe real"?
%
We are aware of two scenarios that might lead to a false positive signals.
%
One scenario is an under-sampled dataset. 
Particularly when timescales of fluctuations are long, e.g. multiple generations of cells, and the size of fluctuations is high, 
a particular real temporal dynamic might dominate the dynamics of the dataset,
and lead to a particular cross-corerelation.
%
This cross-corerelation might then however not 
characterize
the entire dynamics of the parameters of interest.
% 
Another scenario could result from a fluctuating mean signal.
%
In principle, the aim is to measure in steady state, but experimental conditions might not be perfect and lead to a fluctuating mean signal, 
that transpires to the cross-correlation despite the subtraction of the colony mean.
%
(Potentially, there might also be real colony-wide dynamics.)
%
The latter scenario underlines the importance of also visually inspecting the signals of all lineages (Figure \ref{fig:mm:exampleCC}.A-C).
%
Both these scenarios underline the importance of replicate experiments.
%
Nevertheless, also for data from one microcolony, one can quantify parameters that help determine the validity of the calculated cross-correlation.
%
To provide a control 
%for both these scenarios
based on the data of one microcolony, we determine the cross-correlation for time series $f_i$ and $g_j$, where $f_i$ comes from an arbitrary branch $i$ and $g_j$ from an arbitrary branch $j$; 
in other words, we combine time series from two different lineages, which should not correlate, and ergo cross-correlations should be 0.
%
Specifically, for gel pad data all branches are of equal size, and whole time series $f$ are combined with arbitrary time series $g$ from another branch.
%
For microfluidic data from the device described in this chapter, branches are not of equal length, and thus randomization of branch data $f_i(n)$ and $g_j(n)$ is performed for each point in time $n$; this might be slightly less rigorous as this further decorrelates the data.
%
In both cases, we repeat this procedure 50 times, and determine the minimum and the maximum value of $R(\tau)$ resulting from these 50 repeats. 
%
This provides an estimated region of $R(\tau)$ where a non-zero signal might be observed that is likely false-positive.

Additionally, error bars can be estimated by dividing the data from one colony in multiple groups, and analyzing these groups independently.
%
Lastly, as a sanity check, the cross-correlation determined from the scatter plots determined for all different delays $\tau$ can also be determined.
%
The result of these procedures is shown in Figure \ref{fig:mm:exampleCC}.D.


\begin{figure}
	\centering
	\includegraphics[width=0.9\textwidth]{thesis_correlationsExample-01.png}
	\caption{ 
		(A) Example of fluorescence images from a gel pad experiment, where typically a small colony (left) grows into a bigger colony (right).
		% F:\EXPERIMENTAL_DATA_2014-2015_m1\2015-06-12_CRP_asc852-853_plasmids\pos1crop\configFileMadeLater_2015-06-12_pos1.xlsx
		(B) Distribution of growth rates in the microcolony, the black line shows a fitted normal distribution, the dotted line shows values located at 2 standard deviations ($\sigma$) from the mean.
		(C) Fluorescent concentrations over time for all lineages from the microcolony shown in panel (A). Values over the lineages are shown on top of each other (lineages that are plotted behind others are given thicker lines). Note that values overlap at the beginning due to the branched structure of the data (see Figure \ref{fig:mm:weighingCCs}). The dotted lines indicate values that are at a distance of 2 and 5 standard deviations ($\sigma$) from the mean, as determined by fitting a normal distribution.
		(D) The composite cross-correlation as determined between the instantaneous growth rate ($\mu$) and fluorescence concentration ($C$) values over the lineages in this microcolony (black line). The data was subdivided in 4 groups, which allowed for the determination of standard errors. For each delay $\tau$ also a scatter plot was made from which the correlation coefficient was determined, which is shown here in red. (Note that this might deviate slightly depending on the used weighing procedure and the averaging over different sub-groups of data.) Additionally, as a control, growth rate data from the lineages was combined with fluorescence data from random other lineages, and by repeating this procedure 50 times correlation values were determined that should not be interpreted as meaningful deviations from zero.
	}
	\label{fig:mm:exampleCC}
	% D:\Local_Software\Martijn_extensions\Martijn_custom\general_weighingCCs_setup.m
\end{figure}


%%%%



\section*{Acknowledgements}
Thanks to Noreen Walker for helpful discussions regarding the cross-correlation function.

\clearpage


























