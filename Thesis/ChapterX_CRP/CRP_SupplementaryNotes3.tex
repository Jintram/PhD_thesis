

\subsection*{III. The Towbin et al. model equations}

\subsubsection*{The equations}

This supplementary note offers a brief overview of the formulae used in Towbin et al. to describe the CRP regulation system, and its relation to metabolite concentrations and growth.
%
These equations are directly taken from the supplementary notes of Towbin et al. (Equations 7-9), except that some parameter symbols have been changed to match our notation \cite{Towbin2017}.
%
The equations are the following:
%Towbin et al. suggest the following set of equations:

\begin{align}
	\label{eq:TB:dotM}
	\dot{M} & = \mu ( \frac{k_f}{k_f+x} - M) ,\\
%\end{align}
%\begin{align}
	\label{eq:TB:dotx}
	\dot{x} & = \frac{\alpha}{k_2} \left(    \beta f_M(M) \frac{k_1}{x+k_1}  -    \mu    \right) ,\\
%\end{align}
% renamed p_BT to alpha
%\begin{align}
	\label{eq:TB:mu}
	\mu & =\gamma(1-M) \frac{x}{x+k_2}
	.
\end{align}

Here, $M$ represents the metabolic enzyme expression, $x$ the pool of carbon metabolites which give feedback to the CRP system, and $\mu$ the cellular growth rate.
%
Furthermore, $k_1$, $k_2$, and $k_f$ are Michaelis Menten rate constants, where $k_1$ pertains to carbon import (which is self-inhibited), $k_2$ to biomass production and $k_3$ to feedback by metabolites.
%
$\alpha$, $\beta$ and $\gamma$ are also constants, 
$\alpha$ sets the conversion ratio between metabolite consumption and growth,
$\beta$ the maximum import rate of metabolites,
and $\gamma$ the maximum catalytic rate of ribosomes.
%$\gamma$ maximum catalytic rate of ribosomes.
%$\alpha$ is a constants, which set the conversion ratio between metabolite consumption and growth, and $\beta$ is a constant that sets the maximum import rate of metabolites.
% which determine the conversion rate from metabolites to biomass.
%
The left and right terms (within brackets) in Eq. \ref{eq:TB:dotM} relate to production and dilution and in Eq. \ref{eq:TB:dotx} the left and right terms in brackets relate to import and consumption.
%
The growth rate is a function of metabolite consumption and ribosome concentration $R=1-M$.

\subsubsection*{Graphic representations of the model}

The relationships that are set in the differential equations are presented in a diagram in figure \ref{fig:CRP:benjamin_ODEs_diagram}.
%
The contributions of the left and right terms can be plotted separately to understand the dynamic behavior of the system when these terms change independently.

%%%%%%%%%%%%%%%%%%%%%%%%%%%%%%%%%%
\begin{figure}
	\centering
	\includegraphics[width=0.5\textwidth]{CRP-figXsup_BenjaminInteractions.pdf}
	\caption{ 
		\textbf{Diagram of Towbin et al. ODE model.}
		This diagram shows how the parameters in the Towbin et al. model relate to each other. $M$ and $x$ are the metabolic enzyme and metabolite concentration respectively, which are both modeled using differential equations. The ribsome concentration is only implicitly present as $R=1-M$. Metabolic enzyme production $p_m$, growth rate $\mu$, metabolite production/import rate $p_x$ and metabolite consumption are all terms that are part of these differential equations.
	}
	\label{fig:CRP:benjamin_ODEs_diagram}
\end{figure}
%%%%%%%%%%%%%%%%%%%%%%%%%%%%%%%%%%


%%%%%%%%%%%%%%%%%%%%%%%%%%%%%%%%%%
\begin{figure}
	\centering
	\includegraphics[width=0.8\textwidth]{pdf_benjamin_ODE_relations.pdf}
	\caption{ 
		\textbf{Using models to understand the dynamic behavior of stochastic metabolic and growth fluctuations.}
		(Left) Illustrations of the contributions from the positive (production) and negative (dilution) terms to the differential equation that describes the time evolution of the concentration of metabolic enzymes as a function of that concentration $C$, according to the Towbin et al. model \cite{Towbin2017}. The different lines correspond to different values of the metabolite concentration $x$ ($x={0.9, 1.0, 1.1}$). 
		(Right)	Similar as on the left, except that the terms are production and consumption of metabolites as a function of metabolite concentration $x$ and that the different lines correspond to different values of the metabolic enzyme concentration $C$ ($C={0.45,0.50,0.55}$). (Towbin et al. focus on import of metabolites as rate limiting step in the rate at which they are created, but we have used the more general term production here.) 
		These plots are merely to illustrate the systems behavior, so all parameter values were simply set to $1$.
		%, the parameter values used for plotting are: 
		%$k_1=1$,
		%$k_2=1$,
		%$k_f=1$,
		%$\beta=1$,
		%$\gamma=1$ and
		%$\alpha=1$.
		%
		%  mu0 = .25;
		%  C0=.5; C = [C0*.9, C0, C0*1.1] = 0.4500    0.5000    0.5500
		%  x0=1; x = [x0*.9,x0,x0*1.1]
		%
		%
	}
	\label{fig:CRP:benjamin_ODEs}
\end{figure}
%%%%%%%%%%%%%%%%%%%%%%%%%%%%%%%%%%





