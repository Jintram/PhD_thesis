







\section*{Conclusions}

In this chapter, 
we investigated how the CRP regulatory networks responds to environmental inputs versus stochastic inputs.
%
A negative feedback loop, which inhibits CRP activity (through cAMP) and metabolic enzymes expression when metabolite concentrations increase,
is known to be responsible for adjusting metabolic enzyme expression to the growth medium of the cell.
%
We hypothesized that this regulatory interaction should respond equally strong to metabolite fluctuations that occur in the cell due to noise. 

To show that this is a feasible idea, we first subjected the CRP system to a quickly changing cAMP input signal.
To artificially control the input signal, we used $\Delta$cAMP cells that do not respond to internal regulation, but instead responded to the cAMP concentration that we provided in the cellular growth medium.
We also introduced a reporter construct to gauge the metabolic enzyme expression levels.
%\textit{cya} \textit{cpd} null mutants to 
%
The cellular response to an input that we alternated between a high and low signal showed that 
it takes hours before cells reach a new steady state, but much less than an hour before a change in metabolic expression or growth rate can be detected.
%The cellular response to our alternating high/low signal showed that it takes a while before 
%The response to an input that alternated between a high and a low signal in 1 or 5 hour blocks 
%showed 
%When we subjected cells to input that alternated between a high and a low cAMP signal in 1 or 5 hour blocks,
%the metabolic enzyme expression and growth rate 
%
We thus concluded that cellular networks have the potential to quickly react to changing inputs such as stochastic fluctuations in metabolite concentration.

Next, we wanted to 
% expose 
uncover
the regulatory dynamics, if any, of the CRP system to stochastic input signals.
%
%We grew $\Delta$cAMP cells at
%a cAMP concentration that was chosen such that the cells grow at wild growth rates.
We grew  $\Delta$cAMP cells in 800 $\upmu$M cAMP,
a concentration that was chosen to sustain growth rates that compare to wild type growth rates.
%
This resulted in cells that expressed the right amount of metabolic enzymes, 
but whose CRP regulation could not respond to stochastically changing metabolite concentrations.
%
By comparing metabolic expression versus growth scatter plots of these $\Delta$cAMP cells with
scatter plots of wild type cells that could respond to both environmental and stochastic inputs, 
we saw that metabolism-growth dynamics was markedly different between the two conditions.
%
Since removal of stochastic input to the CRP system resulted in a change in metabolism-growth dynamics, 
we interpreted this as evidence that the CRP system responds to stochastic fluctuations in metabolite concentrations.

To further investigate what these changes in dynamics mean, 
we not only compared metabolic expression values with growth rates which were both obtained at the same point in time,
but calculated the correlation between metabolic expression with past, current and future growth rates.
%
This was done by calculating cross-correlations.
%
Using CCs we saw that correlations in the feedback-less $\Delta$cAMP cells and wild type were different over a wide range of delays.
%
This strengthened our idea that stochastic fluctuations have a profound effect on the CRP regulatory system dynamics.

In order to understand the meaning of the difference in dynamics, we used a stochastic differential equation model.
%
This model showed that in experiments without feedback, metabolism-growth dynamics are consistent with modelled dynamics that are influenced by fluctuations in enzyme expression, metabolism itself and volume growth.
% 
However, with feedback, the metabolism-growth dynamics were consistent with modelled dynamics that were dominated by fluctuations in volume growth only.
%
This suggested to us that the negative feedback regulation might actually function to respond to stochastic input, which is 
to prevent fluctuations in metabolism from spreading throughout the cellular biochemical network.



\subsection*{More stuff}
Rosenfeld et al. showed in a synthetic circuit that the gene regulatory function can be altered by noise \cite{Rosenfeld2005}.
%
Point out (when talking about positive CCs) that CV of noise does not change, so effect is subtle.
(Or addition of another independent noise source has little effect on total noise?? - No, addition of two independent noises is 2x sigma [wiki:Sum\_of\_normally\_distributed\_random\_variables], which makes sense since your basically kicking a "particle" twice as hard.)
%
Also take a look at Briat et al. \cite{Briat2016}, which investigates a theoretical regulation system that can handle noise.
% >> Not so interesting since this deals more with a hypothetical (ie theoretical) regulation system that can function in a noisy environment..
% >> Perhaps mention at the end..