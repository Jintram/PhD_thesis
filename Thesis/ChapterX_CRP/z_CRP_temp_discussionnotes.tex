%\section*{Discussion}
\subsection*{Stuff previously noted down}

CONCLUSION: THE RULES ARE THE SAME BUT THE THINGS THAT CHANGE DUE FLUCTUATIONS OR IN CONDITIONS ARE DIFFERENT

\textbf{Talking points.} Evolution doesn't separate dynamic and steady state functionality, and can potentially optimize both.

\begin{itemize}
%
\item Perhaps the feedback ensures that the cells also exhibit "c-line" behavior as observed by you2013.
- Figure \ref{fig:CRP:fig2}.A is also consistent with You2013, i.e. the C-line. However not entirely clear how this would mechanistically work.
\item Why are not all cells growing faster?
\item Also mention Chalancon2012, which talks specifically about interaction between regulation fucntion and noise in a network. (Note that article is in line with what I already knew about literature.)
\item Mention Rosenfeld2005, which actually experimentally establishes that GRF (gene regulation function) changes with noise.
\item Check out review bruggeman2018 (similar to our review).
\item Talk about CV
\item Overproduction metabolic -> too little production generic proteins -> that's why growth decrease.. (this is a bit nuanced difference from simply overproduction costly)
%
\item Note that C+M=T relationship is not observed for fluctuations, as can be seen from scatter plots p-p.
%
\item Additional strains made and conditions measured.
%
\item note that metabolic reporter only captures cell-wide and CRP-controlled fluctuations, but not intrinsic noise or other fluctuations that are not cell-wide or CRP-controlled.
\end{itemize}


\section*{Things to keep in mind}

See notes sent to me by Pieter Rein!

***

\cite{You2013}
You, C., Okano, H., Hui, S., Zhang, Z., Kim, M., Gunderson, C.W., Wang, Y.-P., Lenz, P., Yan, D., and Hwa, T. (2013). Coordination of bacterial proteome with metabolism by cyclic AMP signalling. Nature 500, 301–6. Available at: http://www.ncbi.nlm.nih.gov/pubmed/23925119 [Accessed January 20, 2014].
(This is simply the Hwa article re. proteome partitioning, but should check because they also talk about cAMP)
According to chubukov2014 this is also article that shows akg feedback to CRP.

**

\cite{Somavanshi2016}
Somavanshi, R., Ghosh, B., and Sourjik, V. (2016). Sugar Influx Sensing by the Phosphotransferase System of Escherichia coli. 1–19.
\textit{This paper does not mention CRP (at least at first glance).}
Also check out other papers in (physical) yellow folder in cabinet labeled CRP.

**

\cite{Flamholz2013}
Flamholz, A., Noor, E., Bar-Even, A., Liebermeister, W., and Milo, R. (2013). Glycolytic strategy as a tradeoff between energy yield and protein cost. Proc. Natl. Acad. Sci. 110, 10039–10044.

Fig. S2 is already worth it because it gives nice overview between how glycolysis convert glucose to pyruvate and then either ferments it or aerobically burns it to CO2 (+acetate).

**

\cite{chubukov2014} has a few nice graphs about different "activities" that need to happen in the cell (ie. categories of metabolites and how they are produced.) Might be useful in discussion to point out balancing of fluxes. Also explains how CRP is controlled! Point to YOu2013 for a-kg inhibition feedback loop. 

**

Stewart-Ornstein 2017 \cite{Stewart-Ornstein2017} was sent by Sander, mentions CRP, so quickly check whether might be interesting.

**

Perhaps it is interesting to check out the network topology validation by Michael Stumpf (see Heidelberg Quant conference 2017). 

**


Check also:
Van Heerden et al. \cite{VanHeerden2017}
and
Nordholt et al. \cite{Nordholt2017}.

\subsubsection*{Older piece of text, redundant now}

For example, how cellular composition changes in response to different food sources has been a topic of study for a long time \cite{Schaechter1958}.
%
Regulation of metabolism has recently been shown to fit into a larger picture, in which the cell co-regulates large groups of genes together.
Each of these groups (also called sectors), relate to a major 
%
% Recently, it has been shown that large groups of genes are co-regulated, these groups are also called sectors, each of which relate to a major category of cellular activity such as 
category of cellular activity such as 
metabolism, anabolism, protein synthesis, replication, etc 
%catabolism, metabolism, anabolism, , protein synthesis, replication, etc. 
\cite{Klumpp2009, You2013, Scott2014, Hui2015, Hermsen2015, Erickson2017}.
%
% Note that Erickson is not really growth laws itself, but more about how a switch between two environments occurs
%
An important player in regulating the size of the metabolic sector is the cAMP receptor protein (CRP) 
[cited already \cite{Keseler2017, Gorke2008, Green2014, Grainger2005, Zheng2004}, 
	; not cited: \cite{, Robinson1998, Fic2009, }.
	Ref \cite{Fic2009} is a (somewhat older) review.
%
CRP is activated cyclic adenosine monophosphate (cAMP) and it controls 378 promoters, among which 70 transcription factors \cite{Green2014, Shimada2011}.
%
The CRP.cAMP regulation 
%
Recently, it has been shown that the CRP.cAMP regulation 

%So-called growth laws describe how the ratio between these sectors is controlled by the cells to adapt to different environments.
%
%For some sectors, it is known that their expression level can be controlled by a single molecule or protein.
%For example, guanosine tetraphosphate (ppGpp) controls ribosomal expression \cite{Cashel1969, Potrykus2008, Ross2013, Hui2015}, and the 
%cAMP receptor protein (CRP) controls expression of metabolic enzymes.
% 

