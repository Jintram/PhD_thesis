

\section{More content}




Studies often focus on how biochemical networks either one of these dynamics

An example of biochemical networks responding to extracellular change is the adjustment of enzymatic concentrations to changes in extracellular sugar sources \cite{Towbin2017}.
But even in a constant environment, the stochastic nature of chemical reactions can lead to variations in concentrations of cellular components  \cite{Elowitz2002,Kiviet2014}, 
and a myriad of examples exist of how the biochemical network is designed to both deal with and take advantage of this noise (see also chapter \ref{chapter:literaturereview}).
One example is the suppression of spontaneous fluctuations of concentrations of proteins and metabolites by negative feedback \cite{Brandman2008, Lestas2010, Bowsher2013}.
These two functions --- responding to intracellular on one hand and responding to extracellular changes on the other --- are usually considered separately.
In this chapter, we investigate how these two functions --- responding to intracellular on one hand and responding to extracellular changes on the other --- that are usually considered separately, are related to each other.



\section{Introduction}

% IDEA: FIRST GENERAL QUESTION, THEN LATER GO INTO DETAILS

A straightforward view of cells growing in a constant environment might be 
that of an expanding collection of individual cells that are each perfectly adapted to the environment.
%
Each cell senses the environment, 
conveys this information using intracellular signaling molecules, 
and adjusts its gene expression accordingly to 
% This adaptation occurs by biochemical networks, 
% which make sure that the cell 
express the right amount of proteins that fit the situation \cite{Bray1995, Alon2006, Alon2007, Tyson2010}.
%
This might imply that the cellular behavior solely depends on the environment of the cell.
%
However, as discussed in chapter {chapter:literaturereview}, cellular populations are very heterogeneous \cite{Kiviet2014, Hashimoto2016}.
%
Since intracellular signaling molecules are not isolated from intracellular fluctuations, 
this might also have an effect on gene regulation.
%
Indeed, it has been found that the input-output relationship between a gene's activator and its expression can be different from one cell to the next \cite{Rosenfeld2005, Keegstra2017}.

\section{Articles}

These: 

% SOME CONVENIENT ARTICLES
\cite{Brandman2008}
\cite{Rosenfeld2007}
\cite{Zambrano2015}
\cite{Howell2012}
\cite{Nevozhay2009}
\cite{Hornung2008}
\cite{Dublanche2006}
\cite{Becskei2000}
\cite{Swain2004}
\cite{Bowsher2013}
\cite{Avery2006}
\cite{Maheshri2007}
\cite{Levine2007a}
\cite{Bennett2008a}
\cite{Smits2006}
\cite{Balazsi2011}
\cite{Raj2008}
\cite{Davidson2008}
\cite{Elowitz2002}
\cite{Bruggeman2009}
\cite{Kitano2004a}
.



\section{Introduction}

%Cellular survival critically depends on cellular decision making.
%Biochemical networks allow cells to express the right genes at the right time \cite{Bray1995, Alon2006, Alon2007, Tyson2010}.

Cellular survival strongly depends on expressing the right genes at the right time.
Biochemical networks control when a gene is expressed \cite{Bray1995, Alon2006, Alon2007, Tyson2010}.
%
Often, biochemical networks are studied in the context of a changing environment.
%







In \textit{Escherichia coli}
