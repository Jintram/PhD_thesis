




%%%%%%%%%%%%%%%%%%%%%%%%%%%%%%%%%%%%%%% EDITS TO MAIN TEXT ST FEEDBACK

% MAYBE MENTION AT A LATER POINT?
As the value of $R_{p,\mu}$ at zero delay is reflected in the scatter plots,
which is positive for both the wild type and $\Delta$cAMP strain,
the positive $R_{p,\mu}$ values at zero delay are reflected by positive slopes in the scatter clouds of the metabolic reporter (supplemental figure \ref{fig:CRP:fig2sup}, top panels).

% MAYBE ALSO MENTION AT A LATER POINT:
We furthermore note that the $R_{C_Q,\mu}$ and $R_{p_Q,\mu}$ curves (i.e. constitutive reporter cross-correlations, figure \ref{fig:CRP:fig3}.C-D) for the wild type and $\Delta$cAMP are similar to the 
$R_{C_M,\mu}$ and $R_{p_M,\mu}$ curves (i.e. metabolic reporter cross-correlations) observed for the wild type cells 
(figure \ref{fig:CRP:fig3}.A).


%
%As the scatter plots reflect the correlation at zero delay,
%which is positive for both the wild type and $\Delta$cAMP strain,
%the scatter clouds for the metabolic reporter all have a positive slope (supplemental figure \ref{fig:CRP:fig2sup}, top panels).
%This is also reflected by the slopes of the clouds in the scatter plots (supplemental figure \ref{fig:CRP:fig2sup}).
%
We furthermore note that the relationship between the population averaged production rates and growth rates from different conditions does not seem straight forward (supplemental figure \ref{fig:CRP:fig2sup}, top panels).
%
However, the model previously mentioned in section \ref{CRP:txt:notsoaveragecell} (which was further described in supplementary note I),
predicts that these averages lay on an algebraic curve, which is shown in supplementary figure \ref{fig:CRP:averagerelations2} (bottom left panel).
%
This provides a feasible explanation for the relation between the population average production rates and growth rates across different conditions.




Though we are more interested in the cross-correlations, we also provide the scatter plots in supplementary figure \ref{fig:CRP:fig2sup} (bottom panels).
Also for this data,
the relation between population average production rates and growth rates across different conditions
is consistent with our previously discussed simple mathematical model (see supplementary notes I and bottom right panel of figure \ref{fig:CRP:averagerelations2}).








% THIS TEXT WAS NICE BUT DECIDED TO TELL IT DIFFERENTLY (PREDICTED LINES)
The behavior of cells on the level of population averages is reasonably straightforward to understand.
%
As we show in supplementary note I at the end of this chapter, the steady state relationships between the protein expression and growth 
can be understood using % a mathematical description and 
the following three observations:
%are determined by three ingredients:
(1) the interaction between protein production and dilution sets a protein's concentration (supplementary figure \ref{fig:CRP:averagerelations1}, top panels), 
(2) an increase in metabolic protein expression leads to a mandatory equal decrease in constitutive protein expression and vice versa 
(likely due to limitations on the total protein budget of the cell; see supplementary figure \ref{fig:CRP:averagerelations1}, bottom left panel), 
and 
(3) as also previously observed, there is an optimality relationship between metabolic enzyme expression and growth (supplementary figure \ref{fig:CRP:averagerelations1}, bottom right panel).
%which suggests there is an optimal level of metabolic enzyme expression 
%
Using a simple mathematical model that takes these observations into account,
we predict that the constitutive reporter has a concave form like the optimal curve, though mirrored in the direction of the x-axis.
%
%This model also predicts the algebraic curve that describes the relationship between metabolic reporter production and constitutive reporter production 
%and the algebraic curve that describes the relationships between reporter production growth (supplementary note I).
Supplementary figure \ref{fig:CRP:averagerelations1} shows that our observed steady state values match these predictions quite well. 
% END THIS TEXT WAS NICE BUT DECIDED TO TELL IT DIFFERENTLY (PREDICTED LINES)



























We created scatter plots and  protein production and growth  
both for the metabolic and constitutive reporter (supplemental figure \ref{fig:CRP:fig2sup}).
%
the scatter plots for the relationships between production rate and growth rate are shown in supplemental figure \ref{fig:CRP:fig2sup}
(note that earlier, we determined relationships between the production and growth for the population average values, see section \ref{CRP:txt:notsoaveragecell} and supplementary figure \ref{fig:CRP:averagerelations2}, bottom panels).
%
This model also predicts the algebraic curve that describes the relationship between metabolic reporter production and constitutive reporter production 
and the algebraic curve that describes the relationships between reporter production growth (supplementary note I).
%
%
%
%Note that it is convenient to place these scatter plots in the context of the expected population average relationships (figure \ref{fig:CRP:averagerelations2}), 
%since these reveal that this relationship can be described by an algebraic curve, 
%which can map a production rate to two values.
%
%a specific production rates can map to two growth rate values.
% on the algebraic curves.
%
We calculated the cross-correlation between the metabolic reporter ($p_M$) and the growth rate $\mu$,
i.e. $R_{p_M,\mu}$,
for both the wild type and $\Delta$cAMP strain.
%
%We used the production rates of the metabolic reporter ($p_M$) and the growth rate $\mu$ to 
%When we use the production rates of the metabolic reporter ($p_M$) 
%to calculate $R_{p_M,\mu}$ cross-correlations.
%
Both show positive values, 
but the correlations at positive delays are more pronounced in the $\Delta$cAMP strain.
%
%These 
%also show a difference between the wild type cells and the $\Delta$cAMP cells; 
%the latter shows stronger correlations than the former.
%
%show a similar difference between wild type cells and $\Delta$cAMP cells as in the $R_{C_M,\mu}$ curves:
%the wild type cells show more correlation on the left side of the y-axis (albeit slightly positive correlations), whilst the $\Delta$cAMP cells show more correlation on the right side of the y-axis.
%
This indicates the production of metabolic reporter (and thus metabolic enzymes)
correlates more with future growth rates in the $\Delta$cAMP strain.
%
This again indicates there is a difference in dynamics between the wild type cells with feedback and the feedbackless cells.



















The population average behaviour of the constitutive reporter growth rates is expected to show a concave relationship with reporter concentration (supplementary notes) similar to the metabolic reporter,
but the single cell data does not follow this relationship at all, see figure \ref{fig:CRP:fig2}.B. 










\subsection{PARAGRAPH THAT CAN BE LARGELY REMOVED}
\red{
    If there were a post-translational role for cAMP, then the concentration of cAMP should show a more consistent relation with growth rate than the concentration of metabolic reporter.
    %
    %The concentration of CRP.cAMP should be proportional with the cAMP concentration, and so should the production of our metabolic reporter.
    The production rate of our metabolic reporter should be proportional to the cAMP concentration, 
    so the metabolic reporter production rate could serve as a proxy for cAMP concentration.
    %
    However, this production rate also depends on the growth rate of the cell, and might do so in a non-trivial manner.
    Indeed, it is difficult to find a straightforward explanation for the relationship between production and growth over time, as shown in supplemental figure \ref{fig:CRP:productiontimeevolutions}.
    Only when the metabolic production is normalized against a second reporter with a constitutive promoter (see figure \ref{fig:CRP:fig0sup}.B) as a reference, do we see that most traces overlap.
    %
    (Such a normalization could also be done for the concentration, which appears to make the traces slightly more reproducible, but not substantially different.)
    %
    The larger overlap in the normalized production vs growth time trace (compared to the concentration of metabolic reporter vs. growth) offers some support for a post-translational role of cAMP.
    %
    In \ecoli however, cAMP is usually described as an activator of the translational regulator CRP protein only \cite{gorke2008, keseler2017}.
}








\subsection{UNCLEAR PARAGRAPH THAT SHOULD BE REWRITTEN}
\red{
    The most left green square in this figure indicates a switch to a low cAMP concentration, and the part of the time trace that starts at this square shows that as the metabolic reporter concentration goes down, the growth rate first goes up, and then comes down again. 
    This is consistent with the idea that metabolic enzyme concentration has an optimal value above and below which the cell grows worse \cite{Jensen1993, Dekel2005, Berkhout2013, Ray2016, Towbin2017}.
    %earlier observations which show that the relationship between growth rate and metabolic enzyme concentration shows an optimum concentration above and below which the cell grows worse.
    %
    The idea of an optimum metabolic enzyme concentration is also confirmed when we plot growth rate against the metabolic reporter concentration, as shown in supplemental figure \ref{fig:CRP:scatterspulsing}. 
    %
    Conversely, looking at the part of the trace that starts at the leftmost red square, the reporter concentration appears not to be affected, but the growth rate does go up.
    %
    The rest of the trace is then consistent with traversing on the right part of an optimum curve.
    %
    An explanation for this part of the trace not overlapping with the earlier discussed part of the trace, and the quick growth rate response without concentration change, 
    could be that cAMP also has some post-translational roles in the cell. 
    This would explain the quick responses both in the fast-pulsing regime and the quick response to an upshift of cAMP.
}







% PREVIOUS VERSION AND SCRIBBLES
************************************************************************************
CONTINUE EDITING ABOVE
************************************************************************************

However, the growth rate behaviour showed some puzzling features. 
%
Firstly, in the fast pulse regime (where cAMP concentration was changed every hour), it seems that after a switch to the high cAMP concentration 
a sustained higher growth rate is reached within 20 minutes (the time between two data points).
%Firstly, in the fast regime, within 20 minutes (the time between two data points) after a pulse to a high cAMP concentration there seems to be a sustained faster growth rate.
% Firstly, it appeared that in the fast pulse regime the high cAMP concentration resulted in a sustained faster growth rate within 20 minutes .
%
The sustained higher growth rate is unexpected, as too much metabolic enzyme expression is expected to 
be detrimental (see figure \ref{fig:CRP:ocurvePlatereader} and citations \cite{Towbin2017, Ray2016}).
% come with the cost of superfluous protein production, and might even lead to a detrimental unbalanced ratio of fluxes in the cell \cite{Ray2016}.
%
However, when looking at the switch to high cAMP concentration in the slower pulse regime (with pulses every 5 hours) 
we see that the signal does goes down after approximately two hours. 
%
% MARK
We can further investigate this seemingly erratic behaviour by plotting a time trace of the growth rate against the concentration of metabolic reporter.
%
This trace is shown in figure \ref{fig:CRP:fig1}.C. 
%
We can understand certain features in this curve.
%The different parts of this trace show features we understand.
%
During the fast pulse regime, generally, when CRP reporter concentration goes up, growth rate goes down, and vice versa (figure \ref{fig:CRP:fig1}.C, left panel).
This could indicate that the amount of metabolic enzymes is above the optimum value, 
leading to a further decrease in growth rate when CRP activity goes up,
or a decrease when CRP activity goes down.
%
What is also noticeable, is that the 


At the end of the last 1 hour interval, which had a high cAMP concentration,
the growth rate is 0.74 doublings per hour.
%
Then, right after switching to the first long interval of 5 hours, which has a low cAMP concentration,
the first datapoint immediately shows a sharp decrease in growth rate

, right before switching to the first long interval of 5 hours, the 
At the end of the fast pulsing regime the concentration of cAMP is high and the growth rate is 0.74 mass doublings/hr,




The first datapoint of the slow pulsing regime,

the first five hours at low cAMP concentration in th

Right at the start of the first five hours at low cAMP concentration in the slow pulsing regime

does the concentration 
During the first five hours at low cAMP concentration in the slow pulsing regime both the concentration 
(figure \ref{fig:CRP:fig1}.C, arrow 8 in the right panel).

****************************************
I WAS EDITING HERE, CONTINUE TEXT ABOVE, SEE ALSO NOTES BELOW
****************************************

WHAT DO I WANT TO SAY
> we can now relate the behavior to the optimum curve


REPORTER OTHER TIMESCALE THAN ENZYMES/PROCESSES IN THE CELL THAT RESPOND TO CAMP
























% ANOTHER VERSION
CRP is allosterically activated by the small signaling molecule cyclic adenosine monophosphate (cAMP), 
which is produced from ATP by the enzyme adenylate cyclase (CyaA).
%
The cell can also control cAMP concentration by active degradation, 
which is catalyzed by the 
% An important other way by which the cell controls the cAMP concentrations is active degradation by the 
cAMP phosphodiesterase (CpdA) enzyme\footnote{It is sometimes mentioned in literature that cAMP is also actively exported from the cell by the membrance channel protein TolC, but the primary publication that made this claim has been retracted \cite{Hantke2011}.}.
%
cAMP production by adenylate cyclase is thought to be inhibited by $\upalpha$-ketoacids, such as oxaloacetate (OAA), $\upalpha$-ketoglutarate ($\upalpha$-KG) and pyruvate (PYR) \cite{You2013}.
(Previously, also the phosphorylated enzyme IIA of the phosphotransferase system was thought to activate adenylate cyclase \cite{Keseler2017, Deutscher2008, Gorke2008}, but it has now been suggested that the role of $\upalpha$-ketoacids feedback is bigger \cite{You2013}.)
%
Thus, the CRP-cAMP regulation is wired such that metabolites from the TCA cycle and glycolysis pathway provide negative feedback on metabolic enzyme expression, see also figure \ref{fig:CRP:fig0}.A. 
%
%%%%%%%%%%%
The relationship between the cellular cAMP concentration and growth was quantified for growth on different sugar sources \cite{Towbin2017}.
%
For each sugar, there appeared to be a specific optimal concentration of cAMP, consistent with the idea that
% concentrations above or below this concentration lead to lower growth rates.
%
too low enzyme concentrations curb metabolite fluxes, and too high concentrations draw resources from other cellular processes 
\cite{Jensen1993, Dekel2005, Berkhout2013, Ray2016, Towbin2017, You2013}.
%
This leads to a concentration-growth relationship with a clear optimum, sometimes referred to as the optimum curve \cite{Towbin2017}, as illustrated in figure \ref{fig:CRP:fig0}.B.
%%%%%%%%%%%
% END OF VERSION






An enzyme concentration that is too high or too low leads to lower growth rates.
This is shown by experiments that have quantified 
%as shown by experiments in which 
the relationship between cAMP and cellular growth for different sugar sources. 
For each  relationship is concave with a clear optimum, and hence is also referred to as the optimum curve \cite{Towbin2017}.

%ALSO EDIT THE OPTIMUM CURVE INTO THIS TEXT --- TRY TO BE BRIEF --- SEE ALSO BELOW
%> a sugar corresponds to a specific optimum curve
%> CRP regulation makes sure the respective optimum is reached
%
%\red




% XXXXXXXXXXXXXXXXXXXXXXXXXXXXXXXXXXXXX


The constitution of the proteome can change extensively in response to growth conditions, 

This can involve large scale proteome composition changes, 
as illustrated by experiments where 40\% of the proteome changes in response to growth conditions,
as illustrated by titration experiments where 40\% of the proteome changes in response to growth conditions,
and 


******************************************************
This regulation has large implications for the proteome composition,
which was for example observed to very 
In one set of experiments, proteins related to metabolism were observed to 

This regulation has large implications for the proteome and is vital for the cell,
25\% of the proteome is comprised of metabolic 
as the concentration of metabolic enzymes is 


REGULATION VITAL FOR CELL AS ILLUSTRATED
REGULATION INVOLVES LARGE PROTEOME CHANGES
TOO HIGH OR TOO LOW IS BAD FOR THE CELL


The concentration of genes controlled by CRP can change tenfold, 
******************************************************

% XXXXXXXXXXXXXXXXXXXXXXXXXXXXXXXXXXXXX

% INTRODUCTION
%Studies often focus on how the biochemical network is designed to deal with either one or the other of these two types of dynamics.
%Studies often investigate network architecture in only one of these contexts.
%Studies often investigate network architecture in only an "environmental" context or only in a "stochastic fluctuation" context.
Studies often investigate network architecture only in the context of environmental inputs. %, or only in the context of "stochastic fluctuation" inputs.
%
However, any regulatory interaction in the cell is also faced with stochastic inputs.
%However, any regulatory interaction in the cell is faced with both an environmental and  contexts.
Studies often focus on network architecture in the context of environmental inputs,

%
%It is unclear to what extend regulatory interactions that are known to help the cell deal with a changing environment are affected by stochastic concentration fluctuations
This might have large implications for the optimal network architecture.
%
%In this work, we address the open question that connects these two contexts:
In this work, we investigate the link between environmental and stochastic regulation for the first time in a native cellular control circuit. % by asking
%
%are 
Specifically, we ask:
are regulatory interactions that are known for their role in adaptation to environmental changes also 
%affected 
activated 
by concentration fluctuations in the intracellular environment due to noise?





%More specifically, to maximize growth rate, the cell needs to properly allocate its protein pool among different generic activities such as catabolism, anabolism, protein production, and replication of DNA \cite{Hui2015}.
%The negative feedback loop is thought to enable the cell to estimate what the best concentration of metabolic enzymes is under most conditions
%and set enzyme expression accordingly \cite{Towbin2017}, see also figure \ref{fig:CRP:fig0}.A.
























%%%%%%%%%%%%%%%%%%%%%%%%%%%%%%%%%%%%%% MODEL %%%%%%%%%%%%%%%%%%%%%%%%%%%%%%%%%%%%%%%%%%%%%%%%%%




The interactions between metabolic expression and growth have been described before using differential equations.
%
Kiviet et al. presented a concise model that showed how noise could transmit from enzyme expression to growth.
This model consists of coupled linearized differential equations that describe the interaction between metabolism, protein production, enzymatic concentration and growth,
and Ornstein-Uhlenbeck processes to model noise on these parameters \cite{Kiviet2014}.
Protein production, enzymatic concentration and growth are very concrete parameters, 
but the interpretation the term metabolism remains rather abstract in the Kiviet et al. manuscript. 
Perhaps the interpretation could be refined further as the conglomerate of processes that set the metabolic performance of the cell.
See supplementary note II for more information on the Kiviet model.
%
Towbin et al. solve a system of non-linear differential equations assuming steady state conditions to show that cells use a crude set of rules to find optimal enzyme expression {Towbin2017}.
Interestingly, their metabolic performance is governed completely by what they regard as metabolite concentration $x$.
Metabolites are imported by metabolic enzymes, and consumed to produce proteins and cell growth.
See supplementary note III for more information on the Towbin model. 
%
We here used a set of the more convenient linear equations similar to the Kiviet et al. model as we were only interested in behavior around the steady state parameter values, where a linear description might be applicable.
For the purpose of this model we adopted the view that the metabolite concentration $x$ is proportional to the metabolic performance, 
which allowed for an easy interpretation of how negative feedback from metabolites influences the production of proteins.
%
As we solved our model numerically, it was easy to model all our four parameters, metabolites/metabolic performace $x$, protein production $p$, enzymatic concentration $C$ and growth rate $\mu$, as separate (linear) differential equations.
%
We also added noise sources to these differential equations, and used dampening terms to make sure the system reverted back to equilibrium.
%
Our model is described in more detail in Supplementary notes II, and a graphical diagram is displayed in figure \ref{fig:CRP:fig4}.









%
These models, which are concisely discussed in supplementary notes II and III respectively, give a good starting point for a model for our purposes.
%
Both give functional expressions for growth, enzymatic protein expression, protein production rate. 
%
Additionally, the Kiviet et al. models the metabolic performance of the cell as an Ornstein-Uhlenbeck process,
whereas Towbin et al. explicitly models the metabolite expression $x$ by a differential equation.
%
We combine these approaches by 

Additionally, the Kiviet et al. model describes a fluctuating general metabolic noise source, which reflects the metabolic performance of the cell.
%


In the Kiviet et al. model, the concentration of a metabolite is explicitly modeled with the quantity $x$, 

Towbin et al.



We take a similar approach as the Kiviet et al. model, in that we use a system of linear equations to model the system as a number of Ornstein-Uhlenbeck processes that interact with each other.
%

%
As explained in supplementary note II we extend the Kiviet et al. by describing al of our parameters --- growth, enzymatic protein expression, protein production rate and metabolic performance --- as differential equations (instead of explicit functions).
%
This allowed us to 







To obtain an interpretation of the cross-correlations, 

As also mentioned in Supplementary notes I, the behavior of cells can me modeled by differential equations.
%



\subsection{Scenarios}


\subsubsection*{To write down, maybe}


supplementary note benjamin (to write) shows that excursions from mean concentrations can take arbitrary form
>> so it is convenient to generalize them to a linearized model


















%%%%%%%%%%%%%%%%%%%%%%%%%%%%%%%%%%%%%% RESULTS %%%%%%%%%%%%%%%%%%%%%%%%%%%%%%%%%%%%%%%%%%%%%%%%%%

% 
Because gene expression might fluctuate independent of CRP regulation due to changes in the cellular state,
we 

Because gene expression in general depends on the cellular state and not only on CRP regulation, 
is general is noisy, we also used a second construct to provide a background 


%
As explained later in more detail, we also used a second construct with a promoter that has both the lacI and CRP binding site removed. 
%
























%%%%%%%%%%%%%%%%%%%%%%%%%%%%%%%%%%%%%% INTRODUCTION %%%%%%%%%%%%%%%%%%%%%%%%%%%%%%%%%%%%%%%%%%%%%%%%%%

\section{More content}




Studies often focus on how biochemical networks either one of these dynamics

An example of biochemical networks responding to extracellular change is the adjustment of enzymatic concentrations to changes in extracellular sugar sources \cite{Towbin2017}.
But even in a constant environment, the stochastic nature of chemical reactions can lead to variations in concentrations of cellular components  \cite{Elowitz2002,Kiviet2014}, 
and a myriad of examples exist of how the biochemical network is designed to both deal with and take advantage of this noise (see also chapter \ref{chapter:literaturereview}).
One example is the suppression of spontaneous fluctuations of concentrations of proteins and metabolites by negative feedback \cite{Brandman2008, Lestas2010, Bowsher2013}.
These two functions --- responding to intracellular on one hand and responding to extracellular changes on the other --- are usually considered separately.
In this chapter, we investigate how these two functions --- responding to intracellular on one hand and responding to extracellular changes on the other --- that are usually considered separately, are related to each other.



\section{Introduction}

% IDEA: FIRST GENERAL QUESTION, THEN LATER GO INTO DETAILS

A straightforward view of cells growing in a constant environment might be 
that of an expanding collection of individual cells that are each perfectly adapted to the environment.
%
Each cell senses the environment, 
conveys this information using intracellular signaling molecules, 
and adjusts its gene expression accordingly to 
% This adaptation occurs by biochemical networks, 
% which make sure that the cell 
express the right amount of proteins that fit the situation \cite{Bray1995, Alon2006, Alon2007, Tyson2010}.
%
This might imply that the cellular behavior solely depends on the environment of the cell.
%
However, as discussed in chapter {chapter:literaturereview}, cellular populations are very heterogeneous \cite{Kiviet2014, Hashimoto2016}.
%
Since intracellular signaling molecules are not isolated from intracellular fluctuations, 
this might also have an effect on gene regulation.
%
Indeed, it has been found that the input-output relationship between a gene's activator and its expression can be different from one cell to the next \cite{Rosenfeld2005, Keegstra2017}.

\section{Articles}

These: 

% SOME CONVENIENT ARTICLES
\cite{Brandman2008}
\cite{Rosenfeld2007}
\cite{Zambrano2015}
\cite{Howell2012}
\cite{Nevozhay2009}
\cite{Hornung2008}
\cite{Dublanche2006}
\cite{Becskei2000}
\cite{Swain2004}
\cite{Bowsher2013}
\cite{Avery2006}
\cite{Maheshri2007}
\cite{Levine2007a}
\cite{Bennett2008a}
\cite{Smits2006}
\cite{Balazsi2011}
\cite{Raj2008}
\cite{Davidson2008}
\cite{Elowitz2002}
\cite{Bruggeman2009}
\cite{Kitano2004a}
.



\section{Introduction}

%Cellular survival critically depends on cellular decision making.
%Biochemical networks allow cells to express the right genes at the right time \cite{Bray1995, Alon2006, Alon2007, Tyson2010}.

Cellular survival strongly depends on expressing the right genes at the right time.
Biochemical networks control when a gene is expressed \cite{Bray1995, Alon2006, Alon2007, Tyson2010}.
%
Often, biochemical networks are studied in the context of a changing environment.
%







In \textit{Escherichia coli}





To investigate this question, we employ mutant cells that cannot respond to these stochastic fluctuations.
%



To investigate whether the CRP system responds to these stochastic input signals, 
we employ a mutant that does not have 

we employ a mutant in which we can keep the mean CRP activity at wild type levels,

we employ an adenylate cyclase null mutant that we provide with an external source of cAMP.
%





To investigate whether the CRP system responds to these stochastic input signals, 
% DECOUPLING STORY LINE IS NONSENSE SINCE WE KILL BOTH EXTERNAL AND INTERNAL INPUT AND INSTEAD HAVE cAMP
% --> NOT REALLY SINCE WE MAINTAIN THE EXTERNAL INPUT SETTING, SO CASE IS EXTERNAL+INTERNAL vs. EXTERNAL ONLY
%we need a way to decouple 
%the response to the external fluctuations
%the two types of input provided to the CRP system:
%the external 
%
we need to decouple the input the signal receives due environmental conditions versus the input the cell receives due to internal perturbations (Fig. \ref{fig:CRP:fig0}.D).
%
To achieve this,
we employ an adenylate cyclase null mutant that we provide with an external source of cAMP.
%
In this mutant we can keep the mean CRP activity similar to the wild type CRP activity, 
but leave the cell unable to respond to 


This annihilitates the feedback and allows us to compare a situation where the cell responds to both 
%
In this manuscript, we show that indeed 




Specifically, we focus on the role of the negative feedback in this regulatory architecture.


by the metabolite alpha-ketoglutarate ($\upalpha$-KG) on metabolic enzyme production through the cyclic adenosine monophosphate (cAMP) and CRP 
%\footnote{Previously CRP was also called catabolite gene activator protein (CAP).}.
This negative feedback interaction is responsible for adjusting metabolic enzyme concentrations to different sugar sources in bacterial growth medium \cite{Towbin2017, Doucette2011, You2013}.

% and ask (a) whether this regulatory interaction is affected by stochastic concentration fluctuations in its input and (b) whether CRP 






<Feedback is known to regulate stuff \cite{Goyal2010}>

Both single cell growth rates and protein concentrations can 
which can influence GRF

Maybe put Rosenfeld here?
SOMETHING THAT EXPLAINS WHY THE FOLLOWING IS RELEVANT:
assumes fixed cellular state
[Intracellular messenger molecules might (i) measure fluctuating intracellular concentrations, (ii) fluctuate themselves, 
%
When we pick two cells within a population, their individual growth rates can easily differ by a factor of two, 
as shown by single cell experiments that measured coefficients of variation (CV, standard deviation divided by the mean).
%
Both the CV of instantaneous growth rates (single cell mass doubling rate) and cellular generation time (time between divisions) has been determined to lay between $0.2$-$0.4$ \cite{Kiviet2014, Hashimoto2016}.
%
Also gene expression is known to vary for a long time \cite{Elowitz2002}, 
the CV of gene expression from any promoter lies above 0.1 \cite{Keren2015}.


This has been suggested to influence gene regulation \cite{Rosenfeld2005}.
Groups of genes fluctuate together \cite{Stewart-Ornstein2012}
Johannes' receptor stuff


Also gene expression noise is thought to ELOWITZ
and transmit KIVIET
INTRACELL. ENVIRONMENT / IMPLICATIONS FOR MESSENGERS 

>> FOR EXAMPLE, CAMP

%
This means cellular growth rates within a population can vary a factor of two from one cell to the next.

Also in this work, we find 


% >>> Note: there should not be (too much of) a contrast, because in the end we will find these two views should exist together..

Each of the individual cells tries to maintain homeostasis 


A straightforward view of cells growing in a constant environment is that of 

Cells that grow in a constant environment 
A population of cells in steady state is often though of as an exponential mass of growing cells in homeostasis.
%

% 
However, 
