




%%%%%%%%%%%%%%%%%%%%%%%%%%%%%%%%%%%%%% MODEL %%%%%%%%%%%%%%%%%%%%%%%%%%%%%%%%%%%%%%%%%%%%%%%%%%




The interactions between metabolic expression and growth have been described before using differential equations.
%
Kiviet et al. presented a concise model that showed how noise could transmit from enzyme expression to growth.
This model consists of coupled linearized differential equations that describe the interaction between metabolism, protein production, enzymatic concentration and growth,
and Ornstein-Uhlenbeck processes to model noise on these parameters \cite{Kiviet2014}.
Protein production, enzymatic concentration and growth are very concrete parameters, 
but the interpretation the term metabolism remains rather abstract in the Kiviet et al. manuscript. 
Perhaps the interpretation could be refined further as the conglomerate of processes that set the metabolic performance of the cell.
See supplementary note II for more information on the Kiviet model.
%
Towbin et al. solve a system of non-linear differential equations assuming steady state conditions to show that cells use a crude set of rules to find optimal enzyme expression {Towbin2017}.
Interestingly, their metabolic performance is governed completely by what they regard as metabolite concentration $x$.
Metabolites are imported by metabolic enzymes, and consumed to produce proteins and cell growth.
See supplementary note III for more information on the Towbin model. 
%
We here used a set of the more convenient linear equations similar to the Kiviet et al. model as we were only interested in behavior around the steady state parameter values, where a linear description might be applicable.
For the purpose of this model we adopted the view that the metabolite concentration $x$ is proportional to the metabolic performance, 
which allowed for an easy interpretation of how negative feedback from metabolites influences the production of proteins.
%
As we solved our model numerically, it was easy to model all our four parameters, metabolites/metabolic performace $x$, protein production $p$, enzymatic concentration $C$ and growth rate $\mu$, as separate (linear) differential equations.
%
We also added noise sources to these differential equations, and used dampening terms to make sure the system reverted back to equilibrium.
%
Our model is described in more detail in Supplementary notes II, and a graphical diagram is displayed in figure \ref{fig:CRP:fig4}.









%
These models, which are concisely discussed in supplementary notes II and III respectively, give a good starting point for a model for our purposes.
%
Both give functional expressions for growth, enzymatic protein expression, protein production rate. 
%
Additionally, the Kiviet et al. models the metabolic performance of the cell as an Ornstein-Uhlenbeck process,
whereas Towbin et al. explicitly models the metabolite expression $x$ by a differential equation.
%
We combine these approaches by 

Additionally, the Kiviet et al. model describes a fluctuating general metabolic noise source, which reflects the metabolic performance of the cell.
%


In the Kiviet et al. model, the concentration of a metabolite is explicitly modeled with the quantity $x$, 

Towbin et al.



We take a similar approach as the Kiviet et al. model, in that we use a system of linear equations to model the system as a number of Ornstein-Uhlenbeck processes that interact with each other.
%

%
As explained in supplementary note II we extend the Kiviet et al. by describing al of our parameters --- growth, enzymatic protein expression, protein production rate and metabolic performance --- as differential equations (instead of explicit functions).
%
This allowed us to 







To obtain an interpretation of the cross-correlations, 

As also mentioned in Supplementary notes I, the behavior of cells can me modeled by differential equations.
%



\subsection{Scenarios}


\subsubsection*{To write down, maybe}


supplementary note benjamin (to write) shows that excursions from mean concentrations can take arbitrary form
>> so it is convenient to generalize them to a linearized model


















%%%%%%%%%%%%%%%%%%%%%%%%%%%%%%%%%%%%%% RESULTS %%%%%%%%%%%%%%%%%%%%%%%%%%%%%%%%%%%%%%%%%%%%%%%%%%

% 
Because gene expression might fluctuate independent of CRP regulation due to changes in the cellular state,
we 

Because gene expression in general depends on the cellular state and not only on CRP regulation, 
is general is noisy, we also used a second construct to provide a background 


%
As explained later in more detail, we also used a second construct with a promoter that has both the lacI and CRP binding site removed. 
%
























%%%%%%%%%%%%%%%%%%%%%%%%%%%%%%%%%%%%%% INTRODUCTION %%%%%%%%%%%%%%%%%%%%%%%%%%%%%%%%%%%%%%%%%%%%%%%%%%

\section{More content}




Studies often focus on how biochemical networks either one of these dynamics

An example of biochemical networks responding to extracellular change is the adjustment of enzymatic concentrations to changes in extracellular sugar sources \cite{Towbin2017}.
But even in a constant environment, the stochastic nature of chemical reactions can lead to variations in concentrations of cellular components  \cite{Elowitz2002,Kiviet2014}, 
and a myriad of examples exist of how the biochemical network is designed to both deal with and take advantage of this noise (see also chapter \ref{chapter:literaturereview}).
One example is the suppression of spontaneous fluctuations of concentrations of proteins and metabolites by negative feedback \cite{Brandman2008, Lestas2010, Bowsher2013}.
These two functions --- responding to intracellular on one hand and responding to extracellular changes on the other --- are usually considered separately.
In this chapter, we investigate how these two functions --- responding to intracellular on one hand and responding to extracellular changes on the other --- that are usually considered separately, are related to each other.



\section{Introduction}

% IDEA: FIRST GENERAL QUESTION, THEN LATER GO INTO DETAILS

A straightforward view of cells growing in a constant environment might be 
that of an expanding collection of individual cells that are each perfectly adapted to the environment.
%
Each cell senses the environment, 
conveys this information using intracellular signaling molecules, 
and adjusts its gene expression accordingly to 
% This adaptation occurs by biochemical networks, 
% which make sure that the cell 
express the right amount of proteins that fit the situation \cite{Bray1995, Alon2006, Alon2007, Tyson2010}.
%
This might imply that the cellular behavior solely depends on the environment of the cell.
%
However, as discussed in chapter {chapter:literaturereview}, cellular populations are very heterogeneous \cite{Kiviet2014, Hashimoto2016}.
%
Since intracellular signaling molecules are not isolated from intracellular fluctuations, 
this might also have an effect on gene regulation.
%
Indeed, it has been found that the input-output relationship between a gene's activator and its expression can be different from one cell to the next \cite{Rosenfeld2005, Keegstra2017}.

\section{Articles}

These: 

% SOME CONVENIENT ARTICLES
\cite{Brandman2008}
\cite{Rosenfeld2007}
\cite{Zambrano2015}
\cite{Howell2012}
\cite{Nevozhay2009}
\cite{Hornung2008}
\cite{Dublanche2006}
\cite{Becskei2000}
\cite{Swain2004}
\cite{Bowsher2013}
\cite{Avery2006}
\cite{Maheshri2007}
\cite{Levine2007a}
\cite{Bennett2008a}
\cite{Smits2006}
\cite{Balazsi2011}
\cite{Raj2008}
\cite{Davidson2008}
\cite{Elowitz2002}
\cite{Bruggeman2009}
\cite{Kitano2004a}
.



\section{Introduction}

%Cellular survival critically depends on cellular decision making.
%Biochemical networks allow cells to express the right genes at the right time \cite{Bray1995, Alon2006, Alon2007, Tyson2010}.

Cellular survival strongly depends on expressing the right genes at the right time.
Biochemical networks control when a gene is expressed \cite{Bray1995, Alon2006, Alon2007, Tyson2010}.
%
Often, biochemical networks are studied in the context of a changing environment.
%







In \textit{Escherichia coli}





To investigate this question, we employ mutant cells that cannot respond to these stochastic fluctuations.
%



To investigate whether the CRP system responds to these stochastic input signals, 
we employ a mutant that does not have 

we employ a mutant in which we can keep the mean CRP activity at wild type levels,

we employ an adenylate cyclase null mutant that we provide with an external source of cAMP.
%





To investigate whether the CRP system responds to these stochastic input signals, 
% DECOUPLING STORY LINE IS NONSENSE SINCE WE KILL BOTH EXTERNAL AND INTERNAL INPUT AND INSTEAD HAVE cAMP
% --> NOT REALLY SINCE WE MAINTAIN THE EXTERNAL INPUT SETTING, SO CASE IS EXTERNAL+INTERNAL vs. EXTERNAL ONLY
%we need a way to decouple 
%the response to the external fluctuations
%the two types of input provided to the CRP system:
%the external 
%
we need to decouple the input the signal receives due environmental conditions versus the input the cell receives due to internal perturbations (Fig. \ref{fig:CRP:fig0}.D).
%
To achieve this,
we employ an adenylate cyclase null mutant that we provide with an external source of cAMP.
%
In this mutant we can keep the mean CRP activity similar to the wild type CRP activity, 
but leave the cell unable to respond to 


This annihilitates the feedback and allows us to compare a situation where the cell responds to both 
%
In this manuscript, we show that indeed 




Specifically, we focus on the role of the negative feedback in this regulatory architecture.


by the metabolite alpha-ketoglutarate ($\upalpha$-KG) on metabolic enzyme production through the cyclic adenosine monophosphate (cAMP) and CRP 
%\footnote{Previously CRP was also called catabolite gene activator protein (CAP).}.
This negative feedback interaction is responsible for adjusting metabolic enzyme concentrations to different sugar sources in bacterial growth medium \cite{Towbin2017, Doucette2011, You2013}.

% and ask (a) whether this regulatory interaction is affected by stochastic concentration fluctuations in its input and (b) whether CRP 






<Feedback is known to regulate stuff \cite{Goyal2010}>

Both single cell growth rates and protein concentrations can 
which can influence GRF

Maybe put Rosenfeld here?
SOMETHING THAT EXPLAINS WHY THE FOLLOWING IS RELEVANT:
assumes fixed cellular state
[Intracellular messenger molecules might (i) measure fluctuating intracellular concentrations, (ii) fluctuate themselves, 
%
When we pick two cells within a population, their individual growth rates can easily differ by a factor of two, 
as shown by single cell experiments that measured coefficients of variation (CV, standard deviation divided by the mean).
%
Both the CV of instantaneous growth rates (single cell mass doubling rate) and cellular generation time (time between divisions) has been determined to lay between $0.2$-$0.4$ \cite{Kiviet2014, Hashimoto2016}.
%
Also gene expression is known to vary for a long time \cite{Elowitz2002}, 
the CV of gene expression from any promoter lies above 0.1 \cite{Keren2015}.


This has been suggested to influence gene regulation \cite{Rosenfeld2005}.
Groups of genes fluctuate together \cite{Stewart-Ornstein2012}
Johannes' receptor stuff


Also gene expression noise is thought to ELOWITZ
and transmit KIVIET
INTRACELL. ENVIRONMENT / IMPLICATIONS FOR MESSENGERS 

>> FOR EXAMPLE, CAMP

%
This means cellular growth rates within a population can vary a factor of two from one cell to the next.

Also in this work, we find 


% >>> Note: there should not be (too much of) a contrast, because in the end we will find these two views should exist together..

Each of the individual cells tries to maintain homeostasis 


A straightforward view of cells growing in a constant environment is that of 

Cells that grow in a constant environment 
A population of cells in steady state is often though of as an exponential mass of growing cells in homeostasis.
%

% 
However, 
