



% Notes
% 
% For format of Scientific Reports, see:
% https://www.nature.com/srep/publish/guidelines#format-manuscripts
% - Title length 20 words
% - Main text 4500 words ("guide")
% - Abstract: 200 words (no refs)
% 	> "General introduction to the topic and as a brief, non-technical summary of the main results and their implications"
% - Note that figure captions cannot be more than 350 words
% - Preferably, method section limited to 1500 words
% - results may have subheadings
%
% For format of PLOS ONE, see:
% http://journals.plos.org/plosone/s/submission-guidelines
% - 2 titles, long and short: 250 characters, short title: 100 characters
% - abstract 300 words
% - ..?
% 
% Art of presenting science

% Notes on literature:
% Also check CRP page on ecocyc: https://biocyc.org/gene?orgid=ECOLI&id=CPLX0-226.


%\chapter{Negative metabolic feedback suppresses cell wide fluctuations}
\chapter{CRP responds dynamically to internal noise}
\label{chapter:CRP}

\textit{Martijn Wehrens, Laurens H.J. Krah, Benjamin D. Towbin, Rutger Hermsen, Sander J. Tans}

% Chapter clear first page
\thispagestyle{empty}
\clearpage


\section*{Abstract}

Some regulatory interactions are known to help the cell deal with changes in the environment, but it is unclear how these regulatory interactions respond to concentration changes due to stochastic fluctuations.
%
In this chapter, we address this open question by looking at the cAMP receptor protein (CRP), of which the 

\section{Introduction}


%%% ------------------------
% Let's first define the problem (or knowledge gap)
%%% ------------------------

The world is unpredictable.
%
A bacterial cell observes both its extracellular environment and its intracellular environment,
and adjusts its gene expression to 
optimize its chances of survival in this world.
%
%To survive bacteria need to respond to unforeseen situations and express the right gene at the right time. 
% Expressing the right gene at the right time is vital for bacterial survival. 
%Cells rely on biochemical networks to control gene expression.
%(Biochemical network is a general term that refers to the chemical interactions between proteins and small molecular mechanisms.)
To control gene expression a bacterium --- like any other cell --- relies on chemical interactions between proteins and small molecules \cite{Bray1995, Alon2006, Alon2007, Tyson2010}.
%A bacterium, like any other cell, relies on chemical interactions between proteins and small molecules to control gene expression \cite{Bray1995, Alon2006, Alon2007, Tyson2010}.
%These interactions are also referred to as "biochemical networks".
These interactions, also referred to as "biochemical networks", 
%
%Biochemical networks 
%allow cells to deal with changes that can come both from the extracellular and the intracellular environment.
allow cells to deal with 
% two types of 
unpredictable situations
both in the intracellular and extracellular environment.
%dynamics.
%
%On one hand, the extracellular environment might change, and a different gene expression profile is required.
Changes in the extracellular environment might require the cell to adjust gene expression accordingly.
%
On the other hand,  even without environmental changes, the stochastic nature of chemical reactions can lead to fluctuations of proteins and metabolites over time within the cell \cite{Elowitz2002,Kiviet2014}.
The architecture of the biochemical network might enable the cell to deal with these fluctuations, or even use them to its advantage (see also chapter \ref{chapter:literaturereview}).
%
%Studies often focus on how the biochemical network is designed to deal with either one or the other of these two types of dynamics.
%Studies often investigate network architecture in only one of these contexts.
%Studies often investigate network architecture in only an "environmental" context or only in a "stochastic fluctuation" context.
Studies often investigate network architecture only in the context of "environmental" inputs, or only in the context of "stochastic fluctuation" inputs.
%
However, any regulatory interaction in the cell is faced with both these inputs.
%However, any regulatory interaction in the cell is faced with both an environmental and  contexts.
%
%It is unclear to what extend regulatory interactions that are known to help the cell deal with a changing environment are affected by stochastic concentration fluctuations
This might have implications for the optimal network architecture.
%
In this work, we address the open question that connects these two contexts:
%
are regulatory interactions that are known for their role in adaptation to environmental changes also affected by concentration fluctuations in the intracellular environment due to noise?

%%% ------------------------
% Now illustrate the setting a bit more
%%% ------------------------

To investigate this question, we take a closer look at the dynamics of regulation by the cAMP receptor protein (CRP; previously called catabolite gene activator protein, or CAP). 
%
It is thought that CRP controls the expression level of all catabolic genes in concert \cite{You2013}.
%
Catabolic genes convert large carbohydrate molecules into smaller metabolites, and generate energy for the cell during this process in the form of ATP \cite{Nelson2005}.
%
CRP controls 378 promoters, among which 70 transcription factors \cite{Green2014, Shimada2011}.
% 


MULTIPLE METABOLITES CONTROL THE ACTIVITY OF ADENYLYL CYCLASE, WHICH PRODUCES CAMP
GLUCOSE (EIIA ENZYMES) {SEE REF IN TOWBIN}
ALPHA KETO-ACIDS SUCH AS A-KG AND OAA {SEE YOU2013 ACCORDING TO TOWBIN}
ECOCYC SAYS:
\cite{Hermsen2015}

One of these metabolites is alpha-ketoglutarate ($\upalpha$-KG).
%
$\upalpha$-KG promotes the conversion of ADP to ATP
%
Specifically, we focus on the role of the negative feedback in this regulatory architecture.


by the metabolite alpha-ketoglutarate ($\upalpha$-KG) on metabolic enzyme production through the cyclic adenosine monophosphate (cAMP) and CRP 
%\footnote{Previously CRP was also called catabolite gene activator protein (CAP).}.
This negative feedback interaction is responsible for adjusting metabolic enzyme concentrations to different sugar sources in bacterial growth medium \cite{Towbin2017, Doucette2011, You2013}.

% and ask (a) whether this regulatory interaction is affected by stochastic concentration fluctuations in its input and (b) whether CRP 






<Feedback is known to regulate stuff \cite{Goyal2010}>

Both single cell growth rates and protein concentrations can 
which can influence GRF

Maybe put Rosenfeld here?
SOMETHING THAT EXPLAINS WHY THE FOLLOWING IS RELEVANT:
assumes fixed cellular state
[Intracellular messenger molecules might (i) measure fluctuating intracellular concentrations, (ii) fluctuate themselves, 
%
When we pick two cells within a population, their individual growth rates can easily differ by a factor of two, 
as shown by single cell experiments that measured coefficients of variation (CV, standard deviation divided by the mean).
%
Both the CV of instantaneous growth rates (single cell mass doubling rate) and cellular generation time (time between divisions) has been determined to lay between $0.2$-$0.4$ \cite{Kiviet2014, Hashimoto2016}.
%
Also gene expression is known to vary for a long time \cite{Elowitz2002}, 
the CV of gene expression from any promoter lies above 0.1 \cite{Keren2015}.


This has been suggested to influence gene regulation \cite{Rosenfeld2005}.
Groups of genes fluctuate together \cite{Stewart-Ornstein2012}
Johannes' receptor stuff


Also gene expression noise is thought to ELOWITZ
and transmit KIVIET
INTRACELL. ENVIRONMENT / IMPLICATIONS FOR MESSENGERS 

>> FOR EXAMPLE, CAMP

%
This means cellular growth rates within a population can vary a factor of two from one cell to the next.

Also in this work, we find 


% >>> Note: there should not be (too much of) a contrast, because in the end we will find these two views should exist together..

Each of the individual cells tries to maintain homeostasis 


A straightforward view of cells growing in a constant environment is that of 

Cells that grow in a constant environment 
A population of cells in steady state is often though of as an exponential mass of growing cells in homeostasis.
%

% 
However, 

%%%%%%%%%%%%%%%%%%%%%%%%%%%%%%%%
%
% See file "CRP_literalgraveyard.tex" for more previous versions and ideas.
% 
%%%%%%%%%%%%%%%%%%%%%%%%%%%%%%%%


\section{Results}

%%%%%%%%%%%%%%%%%%%%%%%%%%%%%%%%%%
\begin{figure}
	\centering
	\includegraphics[width=1.0\textwidth]{CRP-fig0.pdf}
	\caption{ 
		(A) 
	}
	\label{fig:CRP:fig0}
\end{figure}
%%%%%%%%%%%%%%%%%%%%%%%%%%%%%%%%%%


%%%%%%%%%%%%%%%%%%%%%%%%%%%%%%%%%%
\begin{figure}
	\centering
	\includegraphics[width=1.0\textwidth]{CRP-fig1_.pdf}
	\clearpage % insert a page break
	\label{fig:CRP:fig1-2}
\end{figure}	

\clearpage

\captionof{figure}{    
	\textbf{Bulk response to metabolic perturbations.}
	A.
	B.
	C.
}
%%%%%%%%%%%%%%%%%%%%%%%%%%%%%%%%%%

%%%%%%%%%%%%%%%%%%%%%%%%%%%%%%%%%%
\begin{figure}
	\centering
	\includegraphics[width=1.0\textwidth]{CRP-fig2.pdf}
	\caption{ 
		(A) 
	}
	\label{fig:CRP:fig2}
\end{figure}
%%%%%%%%%%%%%%%%%%%%%%%%%%%%%%%%%%

%%%%%%%%%%%%%%%%%%%%%%%%%%%%%%%%%%
\begin{figure}
	\centering
	\includegraphics[width=1.0\textwidth]{CRP-fig3.pdf}
	\caption{ 
		(A) 
	}
	\label{fig:CRP:fig3}
\end{figure}
%%%%%%%%%%%%%%%%%%%%%%%%%%%%%%%%%%

%%%%%%%%%%%%%%%%%%%%%%%%%%%%%%%%%%
\begin{figure}
	\centering
	\includegraphics[width=1.0\textwidth]{CRP-fig4.pdf}
	\clearpage % insert a page break
	\label{fig:CRP:fig1-2}
\end{figure}	

\clearpage

\captionof{figure}{    
	\textbf{A simple model explains how feedback filters out noise transmission.}
	A simple model predicts correlation inversion.
	A.
	B.
	C.
}

%%%%%%%%%%%%%%%%%%%%%%%%%%%%%%%%%%

\section{Discussion}

\textbf{Talking points.} Evolution doesn't separate dynamic and steady state functionality, and can potentially optimize both.


For example, how cellular composition changes in response to different food sources has been a topic of study for a long time \cite{Schaechter1958}.
%
Regulation of metabolism has recently been shown to fit into a larger picture, in which the cell co-regulates large groups of genes together.
Each of these groups (also called sectors), relate to a major 
%
% Recently, it has been shown that large groups of genes are co-regulated, these groups are also called sectors, each of which relate to a major category of cellular activity such as 
category of cellular activity such as 
metabolism, anabolism, protein synthesis, replication, etc 
%catabolism, metabolism, anabolism, , protein synthesis, replication, etc. 
\cite{Klumpp2009, You2013, Scott2014, Hui2015, Hermsen2015, Erickson2017}.
%
% Note that Erickson is not really growth laws itself, but more about how a switch between two environments occurs
%
An important player in regulating the size of the metabolic sector is the cAMP receptor protein (CRP) \cite{Keseler2017, Grainger2005, Robinson1998, Zheng2004, Gorke2008, Fic2009, Green2014}.
%
CRP is activated by cyclic adenosine monophosphate (cAMP) and it controls 378 promoters, among which 70 transcription factors \cite{Green2014, Shimada2011}.
%
The CRP.cAMP regulation 
%
Recently, it has been shown that the CRP.cAMP regulation 

%So-called growth laws describe how the ratio between these sectors is controlled by the cells to adapt to different environments.
%
%For some sectors, it is known that their expression level can be controlled by a single molecule or protein.
%For example, guanosine tetraphosphate (ppGpp) controls ribosomal expression \cite{Cashel1969, Potrykus2008, Ross2013, Hui2015}, and the 
%cAMP receptor protein (CRP) controls expression of metabolic enzymes.
% 



\section{Methods}

% For promoter sequences, see: M_2016_06_26_CRP_s70_promoter_sequences.docx

\section*{Acknowledgements}

I thank Pieter Rein ten Wolde and Harmen Wierenga for useful discussions.

%\section{Author contributions}

\section*{Things to keep in mind}

See notes sent to me by Pieter Rein!

***

\cite{You2013}
You, C., Okano, H., Hui, S., Zhang, Z., Kim, M., Gunderson, C.W., Wang, Y.-P., Lenz, P., Yan, D., and Hwa, T. (2013). Coordination of bacterial proteome with metabolism by cyclic AMP signalling. Nature 500, 301–6. Available at: http://www.ncbi.nlm.nih.gov/pubmed/23925119 [Accessed January 20, 2014].
(This is simply the Hwa article re. proteome partitioning, but should check because they also talk about cAMP)
According to chubukov2014 this is also article that shows akg feedback to CRP.

**

\cite{Somavanshi2016}
Somavanshi, R., Ghosh, B., and Sourjik, V. (2016). Sugar Influx Sensing by the Phosphotransferase System of Escherichia coli. 1–19.

Also check out other papers in (physical) yellow folder in cabinet labeled CRP.

**

\cite{Flamholz2013}
Flamholz, A., Noor, E., Bar-Even, A., Liebermeister, W., and Milo, R. (2013). Glycolytic strategy as a tradeoff between energy yield and protein cost. Proc. Natl. Acad. Sci. 110, 10039–10044.

Fig. S2 is already worth it because it gives nice overview between how glycolysis convert glucose to pyruvate and then either ferments it or aerobically burns it to CO2 (+acetate).

**

\cite{chubukov2014} has a few nice graphs about different "activities" that need to happen in the cell (ie. categories of metabolites and how they are produced.) Also explains how CRP is controlled! Point to YOu2013 for a-kg inhibition feedback loop.

**

Stewart-Ornstein 2017 \cite{Stewart-Ornstein2017} was sent by Sander, mentions CRP, so quickly check whether might be interesting.

**

Perhaps it is interesting to check out the network topology validation by Michael Stumpf (see Heidelberg Quant conference 2017). 

**


Check also:
\cite{VanHeerden2017}
and
\cite{Nordholt2017}.

%%%%%%%%%%%%%%%%%%%%%%%%%%%%%%%%%%%%%%%%%%%%%%%%%%%%%%%%%%%%%%%%%%%%%%%%%%%%%%%%%%%%%%%%%%%%%%%%%%%%%%%%%%%%%%%%%%%%%%%%%%%%%%%%%%%%%%%%%%%%%%%%%%%%%%%%%%%%%%%%%%%%%%%%%%%%%%%%%%%%%%%%%%%%%%%%%%%%%%%%%%%%%%%%%
%%%%%%%%%%%%%%%%%%%%%%%%%%%%%%%%%%%%%%%%%%%%%%%%%%%%%%%%%%%%%%%%%%%%%%%%%%%%%%%%%%%%%%%%%%%%%%%%%%%%%%%%%%%%%%%%%%%%%%%%%%%%%%%%%%%%%%%%%%%%%%%%%%%%%%%%%%%%%%%%%%%%%%%%%%%%%%%%%%%%%%%%%%%%%%%%%%%%%%%%%%%%%%%%%

\section*{Supplementary figures}

%%%%%%%%%%%%%%%%%%%%%%%%%%%%%%%%%%%%%%%%%%%%%%%%%%%%%%%%%%%%%%%%%%%%%%%%%%%%%%%%%%%%%%%%%%%%%%%%%%%%%%%%%%%%%%%%%%%%%%%%%%%%%%%%%%%%%%%%%%%%%%%%%%%%%%%%%%%%%%%%%%%%%%%%%%%%%%%%%%%%%%%%%%%%%%%%%%%%%%%%%%%%%%%%%
%% Pulsing dynamics

\begin{figure}%%%%%%%%%%%%%%%%%%%%%%%%%%%%%%%%%%
	\centering
	\includegraphics[width=1.0\textwidth]{pdf_averagetimetraces_4.pdf}
	\includegraphics[width=1.0\textwidth]{pdf_averagetimetraces_2.pdf}
	\includegraphics[width=1.0\textwidth]{pdf_averagetimetraces_5.pdf}
	\clearpage % insert a page break
	\label{fig:XXX:XXX}
\end{figure}	

\clearpage

\captionof{figure}{    
	\textbf{blabla}
}%%%%%%%%%%%%%%%%%%%%%%%%%%%%%%%%%%%%%%%%%%%%%%%

% pdf_scattersoftraces_1

\begin{figure}%%%%%%%%%%%%%%%%%%%%%%%%%%%%%%%%%%%%%%%%%%%%%%%
	\centering
	\includegraphics[width=0.49\textwidth]{pdf_scattersoftraces_1.pdf}
	\includegraphics[width=0.49\textwidth]{pdf_scattersoftraces_2.pdf}	
	\caption{ 
		(A) 
	}
	\label{fig:CRP:XXX}
\end{figure}%%%%%%%%%%%%%%%%%%%%%%%%%%%%%%%%%%%%%%%%%%%%%%%

\begin{figure}%%%%%%%%%%%%%%%%%%%%%%%%%%%%%%%%%%
	\centering
	\includegraphics[width=1.0\textwidth]{pdf_timetracesinglecell13.pdf}
	\includegraphics[width=1.0\textwidth]{pdf_timetracesinglecell49.pdf}
	\includegraphics[width=1.0\textwidth]{pdf_timetracesinglecell144.pdf}
	\clearpage % insert a page break
	\label{fig:XXX:XXX}
\end{figure}	

\clearpage

\captionof{figure}{    
	\textbf{blabla}
}%%%%%%%%%%%%%%%%%%%%%%%%%%%%%%%%%%%%%%%%%%%%%%%



% Dynamic behavior against itself.
\begin{figure}%%%%%%%%%%%%%%%%%%%%%%%%%%%%%%%%%%%%%%%%%%%%%%%
	\centering
	\includegraphics[width=1.00\textwidth]{pdf_highlowplots_case1.pdf}
	\caption{ 
		(A) 
	}
	\label{fig:CRP:XXX}
\end{figure}%%%%%%%%%%%%%%%%%%%%%%%%%%%%%%%%%%%%%%%%%%%%%%%
\begin{figure}%%%%%%%%%%%%%%%%%%%%%%%%%%%%%%%%%%%%%%%%%%%%%%%
	\centering
	\includegraphics[width=1.00\textwidth]{pdf_highlowplots_case2.pdf}
	\caption{ 
		(A) 
	}
	\label{fig:CRP:XXX}
\end{figure}%%%%%%%%%%%%%%%%%%%%%%%%%%%%%%%%%%%%%%%%%%%%%%%
\begin{figure}%%%%%%%%%%%%%%%%%%%%%%%%%%%%%%%%%%%%%%%%%%%%%%%
	\centering
	\includegraphics[width=1.00\textwidth]{pdf_highlowplots_case3.pdf}
	\caption{ 
		(A) 
	}
	\label{fig:CRP:XXX}
\end{figure}%%%%%%%%%%%%%%%%%%%%%%%%%%%%%%%%%%%%%%%%%%%%%%%
\begin{figure}%%%%%%%%%%%%%%%%%%%%%%%%%%%%%%%%%%%%%%%%%%%%%%%
	\centering
	\includegraphics[width=1.00\textwidth]{pdf_highlowplots_case4.pdf}
	\caption{ 
		(A) 
	}
	\label{fig:CRP:XXX}
\end{figure}%%%%%%%%%%%%%%%%%%%%%%%%%%%%%%%%%%%%%%%%%%%%%%%
\begin{figure}%%%%%%%%%%%%%%%%%%%%%%%%%%%%%%%%%%%%%%%%%%%%%%%
	\centering
	\includegraphics[width=1.00\textwidth]{pdf_highlowplots_case5.pdf}
	\caption{ 
		(A) 
	}
	\label{fig:CRP:XXX}
\end{figure}%%%%%%%%%%%%%%%%%%%%%%%%%%%%%%%%%%%%%%%%%%%%%%%
\begin{figure}%%%%%%%%%%%%%%%%%%%%%%%%%%%%%%%%%%%%%%%%%%%%%%%
	\centering
	\includegraphics[width=1.00\textwidth]{pdf_highlowplots_case6.pdf}
	\caption{ 
		(A) 
	}
	\label{fig:CRP:XXX}
\end{figure}%%%%%%%%%%%%%%%%%%%%%%%%%%%%%%%%%%%%%%%%%%%%%%%





%%%%%%%%%%%%%%%%%%%%%%%%%%%%%%%%%%%%%%%%%%%%%%%%%%%%%%%%%%%%%%%%%%%%%%%%%%%%%%%%%%%%%%%%%%%%%%%%%%%%%%%%%%%%%%%%%%%%%%%%%%%%%%%%%%%%%%%%%%%%%%%%%%%%%%%%%%%%%%%%%%%%%%%%%%%%%%%%%%%%%%%%%%%%%%%%%%%%%%%%%%%%%%%%%
%Sup belonging to figure 2 %%%%%%%%%%%%%%%%%%%%%%%%%%%%%%%%%%%%%%%%%%%%%%%%%%%%%%%%%%%%%%%%%%%%%%%%%%%%%%%%%%%%%%%%%%%%%%%%%%%%%%%%%%%%%%%%%%%%%%%%%%%%%%%%%%%%%%%%%%%%%%%%%%%%%%%%%%%%%%%%%%%%%%%%%%%%%%%%%%%%%%%%%%%%%%%%%%%%%%%%%%%%%%%%%

%%%%%%%%%%%%%%%%%%%%%%%%%%%%%%%%%%
\begin{figure}
	\centering
	\includegraphics[width=1.0\textwidth]{CRP-fig2sup.pdf}
	\caption{ 
		(A) 
	}
	\label{fig:CRP:fig2}
\end{figure}
%%%%%%%%%%%%%%%%%%%%%%%%%%%%%%%%%%

%%%%%%%%%%%%%%%%%%%%%%%%%%%%%%%%%%
\begin{figure}
	\centering
	\includegraphics[width=1.0\textwidth]{pdf_chromo1prime_overview_custom_scatter_rateVsRateYC.pdf}
	\includegraphics[width=1.0\textwidth]{pdf_chromo1prime_overview_custom_scatter_concentrationVsConcentrationYC.pdf}
	\caption{ 
		(A) 
	}
	\label{fig:CRP:fig3}
\end{figure}
%%%%%%%%%%%%%%%%%%%%%%%%%%%%%%%%%%

%%%%%%%%%%%%%%%%%%%%%%%%%%%%%%%%%%%%%%%%%%%%%%%%%%%%%%%%%%%%%%%%%%%%%%%%%%%%%%%%%%%%%%%%%%%%%%%%%%%%%%%%%%%%%%%%%%%%%%%%%%%%%%%%%%%%%%%%%%%%%%%%%%%%%%%%%%%%%%%%%%%%%%%%%%%%%%%%%%%%%%%%%%%%%%%%%%%%%%%%%%%%%%%%%
%Sup belonging to figure 3 %%%%%%%%%%%%%%%%%%%%%%%%%%%%%%%%%%%%%%%%%%%%%%%%%%%%%%%%%%%%%%%%%%%%%%%%%%%%%%%%%%%%%%%%%%%%%%%%%%%%%%%%%%%%%%%%%%%%%%%%%%%%%%%%%%%%%%%%%%%%%%%%%%%%%%%%%%%%%%%%%%%%%%%%%%%%%%%%%%%%%%%%%%%%%%%%%%%%%%%%%%%%%%%%%

%%%%%%%%%%%%%%%%%%%%%%%%%%%%%%%%%%
\begin{figure}
	\centering
	\includegraphics[width=1.0\textwidth]{CRP-fig3sup.pdf}
	\caption{ 
		(A) 
	}
	\label{fig:CRP:fig3}
\end{figure}
%%%%%%%%%%%%%%%%%%%%%%%%%%%%%%%%%%

%%%%%%%%%%%%%%%%%%%%%%%%%%%%%%%%%%%%%%%%%%%%%%%%%%%%%%%%%%%%%%%%%%%%%%%%%%%%%%%%%%%%%%%%%%%%%%%%%%%%%%%%%%%%%%%%%%%%%%%%%%%%%%%%%%%%%%%%%%%%%%%%%%%%%%%%%%%%%%%%%%%%%%%%%%%%%%%%%%%%%%%%%%%%%%%%%%%%%%%%%%%%%%%%%
%Sup belonging to figure 4 %%%%%%%%%%%%%%%%%%%%%%%%%%%%%%%%%%%%%%%%%%%%%%%%%%%%%%%%%%%%%%%%%%%%%%%%%%%%%%%%%%%%%%%%%%%%%%%%%%%%%%%%%%%%%%%%%%%%%%%%%%%%%%%%%%%%%%%%%%%%%%%%%%%%%%%%%%%%%%%%%%%%%%%%%%%%%%%%%%%%%%%%%%%%%%%%%%%%%%%%%%%%%%%%%

%%%%%%%%%%%%%%%%%%%%%%%%%%%%%%%%%%%%%%%%%%%%%%%%%%%%%%%%%%%%%%%%%%%%%%%%%%%%%%%%%%%%%%%%%%%%%%%%%%%%%%%%%%%%%%%%%%%%%%%%%%%%%%%%%%%%%%%%%%%%%%%%%%%%%%%%%%%%%%%%%%%%%%%%%%%%%%%%%%%%%%%%%%%%%%%%%%%%%%%%%%%%%%%%%
%Extra sup not belonging but anyways shown %%%%%%%%%%%%%%%%%%%%%%%%%%%%%%%%%%%%%%%%%%%%%%%%%%%%%%%%%%%%%%%%%%%%%%%%%%%%%%%%%%%%%%%%%%%%%%%%%%%%%%%%%%%%%%%%%%%%%%%%%%%%%%%%%%%%%%%%%%%%%%%%%%%%%%%%%%%%%%%%%%%%%%%%%%%%%%%%%%%%%%%%%%%%%%%%%%%%%%%%%%%%%%%%%


%%%%%%%%%%%%%%%%%%%%%%%%%%%%%%
% Figure with floating caption

\begin{figure}
	\centering
	\includegraphics[width=1.0\textwidth]{pdf_chromo1_overview_means.pdf}
	\clearpage % insert a page break
	\label{fig:XXX:XXX}
\end{figure}	

\clearpage

\captionof{figure}{    
	\textbf{blabla}
}
%%%%%%%%%%%%%%%%%%%%%%%%%%%%%%

%%%%%%%%%%%%%%%%%%%%%%%%%%%%%% CRP CCs 1
% Figure with floating caption

\begin{figure}
	\centering
	\includegraphics[width=1.0\textwidth]{pdf_chromo1_CCs_Y6_mean_cycCor_muP9_fitNew_cycCor}
	\clearpage % insert a page break
	\label{fig:XXX:XXX}
\end{figure}	

\clearpage

\captionof{figure}{    
	\textbf{blabla}
}
%%%%%%%%%%%%%%%%%%%%%%%%%%%%%%

%%%%%%%%%%%%%%%%%%%%%%%%%%%%%% 2
% Figure with floating caption

\begin{figure}
	\centering
	\includegraphics[width=1.0\textwidth]{pdf_chromo1_CCs_dY5_cycCor_muP9_fitNew_atdY5_cycCor}
	\clearpage % insert a page break
	\label{fig:XXX:XXX}
\end{figure}	

\clearpage

\captionof{figure}{    
	\textbf{blabla}
}
%%%%%%%%%%%%%%%%%%%%%%%%%%%%%%

%%%%%%%%%%%%%%%%%%%%%%%%%%%%%% Consti CCs 1
% Figure with floating caption

\begin{figure}
	\centering
	\includegraphics[width=1.0\textwidth]{pdf_chromo1_CCs_C6_mean_cycCor_muP9_fitNew_cycCor}
	\clearpage % insert a page break
	\label{fig:XXX:XXX}
\end{figure}	

\clearpage

\captionof{figure}{    
	\textbf{blabla}
}
%%%%%%%%%%%%%%%%%%%%%%%%%%%%%%

%%%%%%%%%%%%%%%%%%%%%%%%%%%%%% 2
% Figure with floating caption

\begin{figure}
	\centering
	\includegraphics[width=1.0\textwidth]{pdf_chromo1_CCs_dC5_cycCor_muP9_fitNew_atdC5_cycCor}
	\clearpage % insert a page break
	\label{fig:XXX:XXX}
\end{figure}	

\clearpage

\captionof{figure}{    
	\textbf{blabla}
}
%%%%%%%%%%%%%%%%%%%%%%%%%%%%%%

%%%%%%%%%%%%%%%%%%%%%%%%%%%%%% Branches mu
% Figure with floating caption

\begin{figure}
	\centering
	\includegraphics[width=1.0\textwidth]{pdf_chromo1_branches_muP9_fitNew_cycCor}
	\clearpage % insert a page break
	\label{fig:XXX:XXX}
\end{figure}	

\clearpage

\captionof{figure}{    
	\textbf{blabla}
}
%%%%%%%%%%%%%%%%%%%%%%%%%%%%%%

%%%%%%%%%%%%%%%%%%%%%%%%%%%%%% Branches Y (CRP)
% Figure with floating caption

\begin{figure}
	\centering
	\includegraphics[width=1.0\textwidth]{pdf_chromo1_branches_Y6_mean_cycCor}
	\clearpage % insert a page break
	\label{fig:XXX:XXX}
\end{figure}	

\clearpage

\captionof{figure}{    
	\textbf{blabla}
}
%%%%%%%%%%%%%%%%%%%%%%%%%%%%%%

%%%%%%%%%%%%%%%%%%%%%%%%%%%%%% Branches dY (CRP)
% Figure with floating caption

\begin{figure}
	\centering
	\includegraphics[width=1.0\textwidth]{pdf_chromo1_branches_dY5_cycCor}
	\clearpage % insert a page break
	\label{fig:XXX:XXX}
\end{figure}	

\clearpage

\captionof{figure}{    
	\textbf{blabla}
}
%%%%%%%%%%%%%%%%%%%%%%%%%%%%%%





%%%%%%%%%%%%%%%%%%%%%%%%%%%%%%%%%%
\begin{figure}
	\centering
	\includegraphics[width=1.0\textwidth]{pdf_plasmids2_overview_correlations_CmuPmu_G.pdf}
	\caption{ 
		\textbf{Without feedback, metabolic fluctuations transmit to growth.}
		The interaction of metabolic and growth fluctuations was measured by plasmid constructs.
		(A) 
	}
	\label{fig:CRP:plasmidCCs}
\end{figure}
%%%%%%%%%%%%%%%%%%%%%%%%%%%%%%%%%%



