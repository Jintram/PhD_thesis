



\chapter{Protocols}

%%%%%%%%%%%%%%%%%%%%%%%%%%%%%%%%%%%%%%%%%%
\section*{Cloning}
%%%%%%%%%%%%%%%%%%%%%%%%%%%%%%%%%%%%%%%%%%

\subsection*{Chemical transformation}
%\begin{inparaenum}[(1)]

\begin{compactitem}
    \item Continuously keep all tubes on ice.
    \item Set water bath to 42C.
    \item Retrieve competent cells from -80C and thaw approximately 10 minutes on ice.
    \item Mix ligation mixture itself with pipette and add to cells (\red{X $\mu L$}).
    \item After addition, mix by gently tapping tube.
    \item Put this on ice for 30 minutes (min. 2 minutes).
    \item Heat shock for 40s at 42C using water bath and floating foam holder.
    \item Put 2 minutes ice.
    \item Add 900 $\mu L$ TY medium.
    \item Put cells at 37C in rotor.
    \item Spin them down, resuspend after removing supernatent, plate everything on antibiotics plate.
    \item Leave plate O/N at 37C (but less than 16hrs).
\end{compactitem}
%\end{inparaenum}

\hfill \textit{Protocol suggested by Vanda Sunderlikova.}


\subsection*{Creating competent cells}
%\begin{inparaenum}[(1)]

\textbf{Day 1}
\begin{compactitem}
    \item Inoculate cells in 3ml TY for O/N saturated culture, put at 30C in shaker.
\end{compactitem}
\textbf{Day 2}
\begin{compactitem}    
    \item Dilute cells 1:200, grow until OD=0.35, around 10ml is needed per transformation.
    \item (Wait.)
    \item Pre-cool centrifuges and liquids.
    \item Incubate cell 30 mins on ice.
    \item Spin at 4C 5 min 5000rpm (1000g).
    \item (Wash 1.) Decant, resuspend in same volume dH2O (pipet up and down).
    \item Spin at 4C 5 min 5000rpm (1000g).
    \item (Wash 2.) Decant, resuspend in dH2O (pipet up and down).
    \item Spin at 4C 5 min 5000rpm (1000g).
    \item (Wash 3.) Decant, resuspend in dH2O (pipet up and down).
    \item Spin at 4C 5 min 5000rpm (1000g).    
    \item Add glycerol such that final concentration glycerol is 15-20\%.
    \item 50µm aliquots.
    \item Either use aliquots or store at -80C.
\end{compactitem}
%\end{inparaenum}

\hfill \textit{Protocol suggested by Milena Lazova, adapted by Vanda Sunderlikova.}

\subsection*{Purification of plasmid from bacterial culture}
To purify plasmids from cell culture, we used the \textit{QIAprep® Spin Miniprep Kit High-Yield Protocol}, 
%('for purification of up to 30 μg plasmid DNA')
which was used with QIAprep Spin Miniprep Kit (cat no. 27104 or 27106). Protocols are supplied with the kits and can also be downloaded at \texttt{www.qiagen.com}.

\newpage

%%%%%%%%%%%%%%%%%%%%%%%%%%%%%%%%%%%%%%%%%%
\section*{Microfluidics}
%%%%%%%%%%%%%%%%%%%%%%%%%%%%%%%%%%%%%%%%%%

\subsection*{Casting PDMS Mother machine flowcell from epoxy mold}

Use a large 50mL Falcon tube to mix components, and a plastic fork (bought at local supermarket) for mixing, the latter can be done (carefully) on vortex with open Falcon and fork. 
\red{[TODO: Add brand etc of PDMS]}

%\begin{inparaenum}[(1)]
\begin{compactitem}
    \item Carefully clean mold with air.   
    \item Mix ... gr + .. gr for 60s and fill mold with low layer (10:1 ratio polymers:curing agent, ~4gr/flowcell; start out 10ml polymer for 2 casts)
    \item Leave in desiccator for approximately 30 minutes (until air bubbles have gone).
    \item Leave in oven for approximately 1-2hrs.
    \item Drill ports with puncher.
    \item (Optional: Check ports and integrity of microstructures under microscope.)
    \item Put in oven 2-24hrs.
\end{compactitem}
%\end{inparaenum}
    
\hfill \textit{Protocol suggested by Daan J. Kiviet.}    
    
%%%%%%%%%%%%%%%%%%%%%%%%%%%%%%%%%%%%%%%%%%

\subsection*{Casting PDMS Mother machine flowcell on wafer}

%\begin{inparaenum}[(1)]
\begin{compactitem}
    \item Cover inside of glass petridish with aluminum foil, in such a way it can be easily removed when it contains the PDMS cast.
    \item Mix 80 gr of polymer solution with 8 gr cross linking solution in plastic cup (1:10 V:V ratio).
    \item Mix 4-5 minutes thoroughly with a spoon.
    \item Tape wafer onto aluminum foil in petridish. 
    \item Poor mixture on wafer. (Make sure solified PDMS can be easily removed from petridish.)
    \item Degass for 1hr in vacuum chamber.
    \item Make sure there are no bubbles. Persistent bubbles can be removed with pipette.
    \item Bake ("cure") for 1-2hrs at 80C. (Or suggested by Victor Caldas up to 1-2 days.)
    \item Remove from mold, be careful with the wafer, it is fragile. (Avoid force perpendicular to it.)
\end{compactitem}
%\end{inparaenum}

\hfill \textit{Protocol adapted from \cite{Boulineau2013} and Dmitry Ershov.}    
    
\newpage    
    
%%%%%%%%%%%%%%%%%%%%%%%%%%%%%%%%%%%%%%%%%%    
    
\subsection*{PDMS to glass binding with plasma (option 1)}    

%\begin{inparaenum}[(1)]
\begin{compactitem}
    \item Bind PDMS to glass using Hi setting for 45 seconds, and leave at hot plate for a few minutes.
\end{compactitem}
%\end{inparaenum}

\hfill \textit{Protocol suggested by Daan J. Kiviet.}

%%%%%%%%%%%%%%%%%%%%%%%%%%%%%%%%%%%%%%%%%%

\subsection*{PDMS to glass binding with Corona (option 2)}    

I found that the procedure works well without cleaning steps of glass and PDMS. I avoid touching the PDMS with bare hands (only gloves or pincers) and start Corona procedure immediately after removing from mold. (Also Daan Kiviet has the experience that need for cleaning is eliminated by immediately executing procedure after removal from mold.) Generally, I work on a clean(ed) surface, and place the PDMS slab on a glass slide (microstructures facing upwards) when not handling it. 

\subsubsection*{Important notes}
%\begin{inparaenum}[(1)]
\begin{compactitem}
    \item Wear gloves. Use blue gloves, the white ones contain sulfur which hinders the reaction.
    \item Work in hood (or ventilated area) because $\text{O}_\text{3}$ is released
    \item Corona produces RF noise, keep 1m away from any digital equipment (phones, smartwatch, et cetera), use line filter for power supply.     
\end{compactitem}
%\end{inparaenum}

\subsubsection*{Protocol}
%\begin{inparaenum}[(1)]
\begin{compactitem}
    \item (If necessary: clean PDMS. 1x scotch tape [the high quality opaque scotch tape]; rinse isopropanol; methanol; H2O)
    \item (Optional. Clean glass: (Haubert et al. wipes w. methanol) rinse acetone, isopropanol, methanol, H2O.)
    \item (Optional: blow-dry PDMS cast.)
    \item Set corona to low stable setting, minimize crackling/sparking
    \item Make 4 passes w. corona over PDMS surface and glass surface, ~.635cm above, (5-20s)
%    \item Pass corona back/forth over surfaces ~.635cm above, 5-20s
    \item (Optional step*: Extremely gently, apply pressure w. finger such that it seals tight everywhere. This can be seen as disappearing air bubble.; Note also that binding starts only after 5 mins.)
%    \item Press surfaces together (binding starts only after 5 mins)
    \item Put 3hrs at 80C (then 1hr-O/N at RT).
\end{compactitem}
%\end{inparaenum}

\hfill \textit{Protocol suggested by Keita Kamino.}    

\hfill

Furthermore: Note surfaces become hydrophilic. References: Haubert et al. \cite{Haubert2006}. See also online:
\begin{verbatim}
https://www.youtube.com/watch?v=Zq8E8gFk490 
https://www.youtube.com/watch?v=vZmhm8HgPys.
\end{verbatim}

%%%%%%%%%%%%%%%%%%%%%%%%%%%%%%%%%%%%%%%%%%

\newpage

\subsection*{Setting up a flowcell}

Once mastered, the procedure described in Boulineau et al. \cite{Boulineau2013} results in a device were cells are kept in place and can be easily exposed to different media by using microfluidic techniques. However, setting up the device in such a way that it remains stable can be challenging. Specifically, there are no seals to keep the fluid in, which easily leads to leaks. 
The description of the procedure here is an attempt to write down as many hands-on tips to get it working. 
% See lab journal p. 

%\begin{inparaenum}[(1)]
\begin{compactitem}
    \item ...
\end{compactitem}
%\end{inparaenum}

I have tried one approach to prevent leakage, which did prevent leakage, but also resulted in cells not being held in place an becoming mobile.
This approach involved adding another microscope glass slide, which had a hole in it which exactly fit the PDMS device. The idea was that the device can now be more easily sealed on the sides by adding grease. This was however not the case, since 

\hfill \textit{Protocol adapted from \cite{Boulineau2013} and Dmitry Ershov.}    








    
