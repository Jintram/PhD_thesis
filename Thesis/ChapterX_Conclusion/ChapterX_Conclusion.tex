



% "A chapter explaining the scientific and tehcnical implications for society of the research findings in considerable detail."

\chapter{Implications for society}

\textit{This chapter discusses scientific and technical implications of the research in this thesis for society.}


\section{An understanding of the fundamentals}
\label{section:society:fundamentals}

All organisms, from bacteria to bird, from jellyfish to jungle tree, have cells. 
%
Some simple organisms, like bacteria, consist of only one cell.
Some organisms consist of many more cells.
Humans for example consist of around ten trillion cells (a one with thirteen zeroes) \cite[BNID 102390]{Milo2010}.
%
%Cells have a membrane that separates their interior environment from the outside.
%They are filled with proteins, other biomolecules like signalling molecules and building blocks, and DNA which stores genetic building plans.
A cell consists of a membrane that % separates their interior environment from the outside,
seperates the outside environment from its inside environment, 
which is filled with proteins, other biomolecules like signalling molecules and building blocks, and DNA which stores genetic building plans.
%Inside the cell, genetic building plans are stored on DNA.
%Ribosomes convert this information to proteins.
%Biomolecules use this information to produce more 
%
Interactions between proteins and other cellular molecules, combined with sensory input from the environment outside the cell, 
determine how the cell behaves, what molecules are produced in the cell and whether it will grow.
%
Like 
% decision making 
in an electronic circuit board,
the properties of the components and their interactions can be identified and quantified,
such that the workings of the cell can be understood and manipulated.
%
Currently, we do not understand all the processes that go on in the biological cell.
%
Quantitative biology approaches like this work allow us to further chart the processes that go on in the cell,
where in this case we have used the bacterium \textit{Escherichia coli} as a model organism. 
%
%Such approaches eventually aid to a better understanding and ability to manipulate these biological processes.
The aim is to obtain a better understanding and ability to manipulate the biological processes in the biological cell. 
%
Since cells are building blocks of all organisms, 
including humans and pathogenic microorganisms,
this eventually might bring us closer to understanding and solving human disease.
%
Additionally, microorganisms play an important role in industry,
for example in food production and sewage treatment.
%
What exactly the direct implications of understanding fundamental cellular processes are is hard to predict,
but such knowledge has the potential to have a wide impact on society.

\section{Killing bacteria}

Nevertheless, we will try to reflect more specifically on the impact of different aspects of this work.
%
In chapter \ref{chapter:filarecovery}, we described how bacteria manage resuming their divisions after they 
paused the division process due to situations they experienced as stressful.
%
This is thought to be a bacterial survival mechanism.
%
Knowledge about bacterial survival mechanisms might be useful in situations 
where from a human perspective it is desired to kill or inhibit bacterial growth.
%
Though bacteria filament in response to many adverse conditions, 
practically relevant conditions include antibiotic exposure and high or low temperatures.
%
These conditions are relevant in clinical and food preservation settings, respectively.
%
Information about the filamentation process might improve predictions about bacterial survival in these settings,
and eventually might help adjusting clinical treatments or preservation processes to increase their effectiveness.
%
Furthermore, targeting the components of the filamentation mechanism specifically might 
decrease bacterial proliferation in % settings where proliferation is not desired.
such settings.

\section{Methods with a wider relevance}

%The two other topics in this thesis, 
Besides 
% Next to 
division of filamentous bacteria, there are two more topics in this thesis.
%Namely 
These are: 
(1) how stochasticity, regulation and individuality of bacteria relate to each other, investigated in the CRP metabolic regulatory system, discussed in chapter \ref{chapter:CRP},
and (2) how stochasticity in ribosomal expression relates to bacterial individuality and growth, discussed in chapter \ref{chapter:ribosomes}.
%
%As described in section \ref{section:society:fundamentals}, it is important to gain deeper insights in how any biological cell works,
%and it is hard to predict direct implications of such endeavours.
%
Firstly, 
it is noteworthy to mention that the methods used in the studies of these two topics (see chapter \ref{chapter:methods})
could have a societal relevance. 
%
Automated image analysis of large quantities of individual cells as used in this thesis has wide applications, % advances in these methods 
and experience from this type of analysis might also prove useful in a more clinical settings.
%
Conceivable examples include analysis of blood cells or other tissue samples for the purpose of diagnostics.
%
Furthermore, growing biological cells in microfluidic chambers in order to obtain single cell data is also a technique that can be applied more broadly.
%
Single cell data can be and is already being acquired from human cell lines, 
which might improve our understanding of human biology and disease.
%
Improvements in single cell microfluidic methods might contribute to the advancement of this research.

\section{The importance of heterogeneity}

One can speculate about the more direct, albeit future, impacts on society of 
the work described in chapters \ref{chapter:CRP} and \ref{chapter:ribosomes}.
%
%the two topics mentioned in the previous paragraph.
% this work.
%
In chapter \ref{chapter:CRP} we saw that regulation systems might be important in transmitting stochastic fluctuations, 
which contributes to heterogeneity in the population.
%
In chapter \ref{chapter:ribosomes} we investigated the role of ribosomes in cellular individuality, which relates immediately to population heterogeneity.
%
Understanding the mechanisms behind heterogeneity gives us fundamental insights in the workings of the biological cell, 
but heterogeneity might also have a more direct relevance.
%
For example, microorganisms are also used in industrial applications.
%
Practical examples include the production of yoghurt and beer.
Perhaps less well known is the effort to 
%and researchers are now also trying to 
genetically modify organisms to produce fuels \cite{Lee2008, Savakis2013}. 
%
In such applications, microorganisms are usually employed to perform a single task.
%
Like the production of a specific beer, or a specific molecule that can be used as fuel. 
%
There likely is an optimal state for an individual microorganism to perform such a task, 
but the phenomenon of heterogeneity prevents cells in the population to uniformly be in that state.
%
%Heterogeneity in a population performing such a singular task might decrease the efficiency and yield of such processes, 
%because not all individuals are functioning 
%
Thus, whilst heterogeneity might be beneficial from an evolutionary perspective,
genetic engineering to alter this phenomenon could potentially improve the yield and/or allow better fine-tuning of industrial processes that involve microorganisms.

Furthermore, as also discussed in chapter \ref{chapter:literaturereview},
heterogeneity might contribute to enhance the survival chances of a population, 
because different individuals can be prepared for different future scenarios (bet hedging).
%
Knowledge about the origins of heterogeneity might thus be relevant in a clinical setting,
%when it is often desired
where the desire often exists 
to prevent the growth of microorganisms like bacteria.










