% Texstudio Spellchecker language
% !TeX spellcheck = nl_NL

\chapter*{Implicaties voor de samenleving (NL)}
\addcontentsline{toc}{chapter}{Implicaties voor de samenleving (NL)}
\setheader{Implicaties voor de samenleving (NL)}

% SWITCH LANGUAGE TO DUTCH FOR HYPHENATION
\selectlanguage{dutch}

\textit{In dit hoofdstuk worden wetenschappelijke en technische implicaties van dit onderzoek voor de samenleving besproken.}

\section*{Het begrijpen van de fundamenten}

Alle organismen, van bacterie tot boomvalk, van kwal tot Koreaanse spar, hebben cellen.
%
Sommige simpele organismen, zoals bacterien, bestaan uit slechts één cel.
Sommige organismen bestaan uit heel veel cellen.
Mensen bijvoorbeeld, bestaan uit ongeveer tien biljoen cellen (een een met dertien nullen) \cite[BNID 102390]{Milo2010}.
%
Een cel bestaat uit een membraan dat 
een scheidingsbarriere vormt tussen wat zich in de cel bevindt en dat wat zich daarbuiten bevindt,
in de cel bevinden zich eiwitten, andere biomoleculen zoals signaaleiwitten en bouwstenen, en DNA waarop genetische informatie opgeslagen is.
%
Interacties tussen eiwitten en andere cellulaire moleculen, in combinatie met sensor \textit{inputs} vanuit wat buiten de cel gebeurt,
bepalen hoe de cel zich gedraagt, welke moleculen er geproduceerd worden in de cel, en of de cel gaat groeien.
%
Zoals in een elektronisch circuit kunnen de eigenschappen van componenten en hun interacties 
geïdentificeerd en gekarakteriseerd worden,
zodat de werking van de cel begrepen en gemanipuleerd kan worden.
%
Momenteel begrijpen we niet alle processen die in de biologische cel plaatsvinden.
%
De aanpak van de kwantitatieve biologie, zoals in deze thesis,
is erop gericht cellulaire processen verder in kaart te brengen.
Hierbij wordt de bacterie \textit{Escherichia coli} als model organisme gebruikt.
%
Het doel is om een beter begrip te krijgen van biologische processen in de cel, 
en die processen ook te kunnen manipuleren.
%
Aangezien alle organismen opgebouwd zijn uit cellen,
inclusief mensen en ziekteverwekkende micro-organismen, 
kan dit uiteindelijk bijdragen aan het begrijpen en genezen van menselijke ziektes.
%
Daarnaast spelen microorganismen een belangrijke rol in de industrie,
bijvoorbeeld in voedselproductie en rioolwaterzuivering.
%
Wat precies de directe implicaties zullen zijn van een beter begrip van fundamentele processen die plaatsvinden in de cel is moeilijk te voorspellen,
maar in potentie kan zulke kennis een grote impact hebben. 

\section*{Bacteriën doden}

Desondanks zullen we meer specifiek proberen te reflecteren op de impact van het werk in deze thesis.
%
In hoofdstuk \ref{chapter:filarecovery} beschrijven we hoe bacteriën verder kunnen delen nadat
het delingsproces gepauzeerd is door situaties die de bacterie als stressvol ervaart. 
%
Er wordt gedacht dat dit proces een overlevingsmechanisme is.
%
Kennis over hoe overlevingsmechanismen werken zijn nuttig in situaties waarin
het vanuit een menselijke oogpunt gewenst is bacteriën te doden of hun groei te remmen.
%
Bacteriën filamenteren (het stoppen met delen maar blijven groeien) in reactie op vele vijandige condities. 
Praktisch relevante condities waarin dit gebeurt zijn bijvoorbeeld blootstelling aan antibiotica en hoge en lage temperaturen.
%
Deze condities zijn respectievelijk relevant in klinische context en de context van conservering van voeding.
%
Kennis over het filamentatieproces kan bijvoorbeeld voorspellingen met betrekking tot overlevingskansen van bacteriën verbeteren,
en zou uiteindelijk kunnen bijdragen aan het aanpassen van klinische behandelingen of conservering van voeding effectiever maken. 
%
Daarnaast kunnen componenten van het filamentatiemechanisme wellicht specifiek worden uitgeschakeld om 
bacteriegroei te remmen in deze contexten.


\section*{Methoden met een bredere relevantie}

Naast de deling van filamenteuze bacteriën zijn er nog twee onderwerpen in deze thesis.
%
Te weten: 
(1) hoe stochasticiteit, regulatie en individualiteit van bacteriën in verhouding staan met elkaar, een vraag die we onderzoeken in een het CRP metabole regulatiesysteem, zoals beschreven in hoofdstuk \ref{chapter:CRP},
en (2) hoe stochasticiteit in ribosomale expressie zich verhoudt met bacteriele individualiteit en groei, zoals beschreven in hoofdstuk \ref{chapter:ribosomes}.


\red{Alinea gaat verder:}
%
%As described in section \ref{section:society:fundamentals}, it is important to gain deeper insights in how any biological cell works,
%and it is hard to predict direct implications of such endeavours.
%
Firstly, 
it is noteworthy to mention that the methods used in the studies of these two topics (see chapter \ref{chapter:methods})
could have a societal relevance. 
%
Automated image analysis of large quantities of individual cells as used in this thesis has wide applications, % advances in these methods 
and experience from this type of analysis might also prove useful in a more clinical settings.
%
Conceivable examples include analysis of blood cells or other tissue samples for the purpose of diagnostics.
%
Furthermore, growing biological cells in microfluidic chambers in order to obtain single cell data is also a technique that can be applied more broadly.
%
Single cell data can be and is already being acquired from human cell lines, 
which might improve our understanding of human biology and disease.
%
Improvements in single cell microfluidic methods might contribute to the advancement of this research.

\section{The importance of heterogeneity}

One can speculate about the more direct, albeit future, impacts on society of 
the work described in chapters \ref{chapter:CRP} and \ref{chapter:ribosomes}.
%
%the two topics mentioned in the previous paragraph.
% this work.
%
In chapter \ref{chapter:CRP} we saw that regulation systems might be important in transmitting stochastic fluctuations, 
which contributes to heterogeneity in the population.
%
In chapter \ref{chapter:ribosomes} we investigated the role of ribosomes in cellular individuality, which relates immediately to population heterogeneity.
%
Understanding the mechanisms behind heterogeneity gives us fundamental insights in the workings of the biological cell, 
but heterogeneity might also have a more direct relevance.
%
For example, microorganisms are also used in industrial applications.
%
Practical examples include the production of yoghurt and beer.
Perhaps less well known is the effort to 
%and researchers are now also trying to 
genetically modify organisms to produce fuels \cite{Lee2008, Savakis2013}. 
%
In such applications, microorganisms are usually employed to perform a single task.
%
Like the production of a specific beer, or a specific molecule that can be used as fuel. 
%
There likely is an optimal state for an individual microorganism to perform such a task, 
but the phenomenon of heterogeneity prevents cells in the population to uniformly be in that state.
%
%Heterogeneity in a population performing such a singular task might decrease the efficiency and yield of such processes, 
%because not all individuals are functioning 
%
Thus, whilst heterogeneity might be beneficial from an evolutionary perspective,
genetic engineering to alter this phenomenon could potentially improve the yield and/or allow better fine-tuning of industrial processes that involve microorganisms.

Furthermore, as also discussed in chapter \ref{chapter:literaturereview},
heterogeneity might contribute to enhance the survival chances of a population, 
because different individuals can be prepared for different future scenarios (bet hedging).
%
Knowledge about the origins of heterogeneity might thus be relevant in a clinical setting,
%when it is often desired
where the desire often exists 
to prevent the growth of microorganisms like bacteria.

% SWITCH LANGUAGE BACK TO ENGLISH!
\selectlanguage{english}








