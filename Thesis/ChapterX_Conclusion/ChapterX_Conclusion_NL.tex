% Texstudio Spellchecker language
% !TeX spellcheck = nl_NL

\chapter*{Implicaties voor de samenleving (NL)}
\addcontentsline{toc}{chapter}{Implicaties voor de samenleving (NL)}
\setheader{Implicaties voor de samenleving (NL)}

% SWITCH LANGUAGE TO DUTCH FOR HYPHENATION
\selectlanguage{dutch}

\textit{In dit hoofdstuk worden wetenschappelijke en technische implicaties van dit onderzoek voor de samenleving besproken.}

\section*{Het begrijpen van de fundamenten}

Alle organismen, van bacterie tot boomvalk, van kwal tot Koreaanse spar, zijn opgebouwd uit cellen.
%
Sommige simpele organismen, zoals bacteriën, bestaan uit slechts één cel.
Sommige organismen bestaan uit heel veel cellen.
Mensen bestaan bijvoorbeeld uit ongeveer tien biljoen cellen (een één met dertien nullen) \cite[BNID 102390]{Milo2010}.
%
Een cel heeft een membraan dat 
een scheidingsbarriere vormt tussen dat wat zich in de cel bevindt en dat wat zich daarbuiten bevindt. 
In de cel bevinden zich eiwitten, andere biomoleculen zoals signaaleiwitten en bouwstenen, en DNA waarop genetische informatie opgeslagen is.
%
Interacties tussen eiwitten en andere cellulaire moleculen, in combinatie met sensorische inputs vanuit wat buiten de cel gebeurt,
bepalen hoe de cel zich gedraagt, welke moleculen er geproduceerd worden in de cel, en of de cel gaat groeien.
%
Zoals in een elektronisch circuit kunnen de eigenschappen van componenten en hun interacties 
geïdentificeerd en gekarakteriseerd worden,
zodat de werking van de cel begrepen en gemanipuleerd kan worden.
%
Momenteel begrijpen we niet alle processen die in de biologische cel plaatsvinden.
%
De aanpak van de kwantitatieve biologie, zoals in deze thesis,
is erop gericht cellulaire processen verder in kaart te brengen.
Hierbij wordt de bacterie \textit{Escherichia coli} als modelorganisme gebruikt.
%
Het doel is om een beter begrip te krijgen van biologische processen in de cel, 
en die processen ook te kunnen manipuleren.
%
Aangezien alle organismen opgebouwd zijn uit cellen,
inclusief mensen en ziekteverwekkende micro-organismen, 
kan dit uiteindelijk bijdragen aan het begrijpen en genezen van menselijke ziektes.
%
Daarnaast spelen micro-organismen een belangrijke rol in de industrie,
bijvoorbeeld in voedselproductie en rioolwaterzuivering.
%
Wat precies de directe implicaties zullen zijn van een beter begrip van fundamentele processen die plaatsvinden in de cel is moeilijk te voorspellen,
maar in potentie kan zulke kennis een grote impact hebben. 

\section*{Bacteriën doden}

Desalniettemin zullen we meer specifiek proberen te reflecteren op de impact van het werk in deze thesis.
%
In hoofdstuk \ref{chapter:filarecovery} beschrijven we hoe bacteriën verder kunnen delen nadat
het delingsproces gepauzeerd is door situaties die de bacterie als stressvol ervaart. 
%
Er wordt gedacht dat dit proces een overlevingsmechanisme is.
%
Kennis over hoe overlevingsmechanismen werken zijn nuttig in situaties waarin
het vanuit een menselijk oogpunt gewenst is bacteriën te doden of hun groei te remmen.
%
Bacteriën filamenteren (stoppen met delen terwijl ze blijven groeien) in reactie op vele vijandige condities. 
Praktisch relevante condities waarin dit gebeurt zijn bijvoorbeeld blootstelling aan antibiotica en hoge en lage temperaturen.
%
Deze condities zijn respectievelijk relevant in klinische context en in de context van conservering van voeding.
%
Kennis over het filamentatieproces kan bijvoorbeeld voorspellingen met betrekking tot overlevingskansen van bacteriën verbeteren,
en zou uiteindelijk kunnen bijdragen aan het aanpassen van klinische behandelingen of conservering van voeding effectiever kunnen maken. 
%
Daarnaast kunnen in deze context componenten van het filamentatiemechanisme wellicht specifiek worden uitgeschakeld om 
bacteriegroei te remmen.


\section*{Methoden met een bredere relevantie}

Naast de deling van filamenteuze bacteriën zijn er nog twee onderwerpen in deze thesis.
%
Te weten: 
(1) hoe stochasticiteit, regulatie en individualiteit van bacteriën in verhouding staan met elkaar, een vraag die we onderzoeken in het CRP metabole regulatiesysteem, zoals beschreven in hoofdstuk \ref{chapter:CRP},
en (2) hoe stochasticiteit in ribosomale expressie zich verhoudt tot bacteriële individualiteit en groei, zoals beschreven in hoofdstuk \ref{chapter:ribosomes}.
%
Ten eerste is het noemenswaardig dat de methoden die in deze twee studies gebruikt zijn (zie hoofdstuk \ref{chapter:methods})
van nut kunnen zijn voor de samenleving.
%
Geautomatiseerde analyse van afbeeldingen met grote hoeveelheden individuele cellen heeft bredere toepassingen, 
en opgedane ervaringen met dit type analyse kan nuttig zijn in een klinische setting. 
%
Dit zou bijvoorbeeld nuttig kunnen zijn bij de analyse van bloedcellen of andere weefselanalyse voor diagnostische toepassingen.
%
Daarnaast kan ook de techniek waarmee individuele cellen in microkamers gekweekt worden (\textit{single cell microfluidics}) breder toegepast worden.
%
Momenteel zijn er al onderzoeken waarbij data wordt verkregen van individuele humane cellijnen, 
wat kan bijdragen aan een beter begrip van humane biologie en ziektes.
%
Vorderingen in \textit{single cell microfluidics} methoden kunnen wellicht bijdragen aan de verdere ontwikkeling van dit onderzoek.


\section*{Het belang van heterogeniteit}

We kunnen slechts speculeren over meer directe, toekomstige impact op de samenleving 
van hoofdstukken \ref{chapter:CRP} en \ref{chapter:ribosomes}.
%
In hoofdstuk \ref{chapter:CRP} zagen we dat regulatiesystemen wellicht een rol spelen in het doorgeven van fluctuaties,
wat bijdraagt aan de heterogeniteit van populaties.
%
In hoofdstuk \ref{chapter:ribosomes} hebben we onderzocht hoe ribosomen bijdragen aan cellulaire individualiteit, 
deze individualiteit draagt ook bij aan de heterogeniteit van populaties.
%waarmee we ook probeerden heterogeniteit in populaties beter te begrijpen.
%
Het begrijpen van heterogeniteit in populaties verschaft ons fundamentele inzichten in hoe de cel werkt, 
maar heterogeniteit heeft ook een directere relevantie.
%
Micro-organismen worden ook gebruikt in de industrie.
%
Bijvoorbeeld bij de productie van yoghurt en bier.
%
Wellicht minder bekend is het idee om micro-organismen 
genetisch te modificeren om zo brandstof te produceren \cite{Lee2008, Savakis2013}. 
%
Bij zulke toepassingen worden de micro-organismen doorgaans gebruikt om één specifieke taak uit te voeren.
%
Bijvoorbeeld de vergisting van een specifiek bier, of de productie van een molecuul dat als brandstof gebruikt kan worden.
%
Waarschijnlijk is er een optimale staat (configuratie) waarin een individueel micro-organisme zo'n taak kan uitvoeren,
maar heterogeniteit in de populatie zorgt ervoor dat niet de gehele populatie diezelfde staat kan aannemen. 
%
Dus hoewel heterogeniteit wellicht voordelig is uit evolutionair oogpunt, 
zou genetische modificatie van deze eigenschap wellicht de opbrengst en/of regulatie van industriële processen waar micro-organismen bij betrokken zijn 
kunnen verbeteren.


Voorts, zoals ook in hoofdstuk \ref{chapter:literaturereview} beschreven, 
kan heterogeniteit bijdragen aan de overlevingskans van populaties, 
omdat verschillende individuen voorbereid kunnen zijn op verschillende toekomstige scenario's (\textit{bet hedging}).
%
Kennis over de oorsprong van die heterogeniteit kan dus relevant zijn in een klinische context, 
waar het vaak juist gewenst is de groei van micro-organismen zoals bacteriën af te remmen. 




% SWITCH LANGUAGE BACK TO ENGLISH!
\selectlanguage{english}








