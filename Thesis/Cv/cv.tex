

\hyphenation{Nij-me-gen}

\chapter*{Curriculum vitae}
\addcontentsline{toc}{chapter}{Curriculum vitae}


Martijn Wehrens (April 3rd, 1986) was born in Nijmegen, where he also attented high school 
at the \textit{Nijmeegse Scholengemeenschap Groenewoud}.
%
He always had an interest in science and technology, 
and at that time for example worked on building and programming moving little machines at an engineering club.
%
To understand the most complicated machine of all, life, he started to study biology at the Radboud University in Nijmegen in 2004.
%
In 2007 he realized that to fully grasp fundamental processes in the biological cell, 
one should delve deeper into the biochemical processes that go on in the cell.
%
%He therefor also joined the chemistry bachelor program.
He therefore also started to study chemistry. 
%
Next to his studies he also participated in the Honours program, with elective courses on free will, holy texts, community, and economics; 
% Gemeenschap en Utopie, Concepten van de vrije wil, Heilige teksten, Cultuur en Economie
worked as editor for student magazine ANS; and was a member of the students council.
%
After finishing both the biology and chemistry bachelor program in 2010, he wanted to further understand the cell at the fundamental level.
%
He therefor joined the MSc program in chemistry at the University of Amsterdam.
%
During this theoretical program he focused on computer simulations of chemical processes.
%
This was also the topic of his MSc thesis, where he elaborated and used the eGFRD reaction-diffusion simulation algorithm to show that 
positive feedback in chemical reactions can contribute to clustering of receptor molecules on the surface of biological cells. 
%
This research was done in the Biochemical Networks research group of Prof. Pieter Rein ten Wolde at the AMOLF research institute in Amsterdam.
%
After completing the MSc program in chemistry in 2013, he further pursued his interest in biochemical networks 
and started to work as a PhD student in the Prof. Sander J. Tans Biophysics group at AMOLF in January of 2014. 
%
The work on dynamical regulation in single cells he performed as a PhD student is described in this thesis.




%\begin{figure}[H]
%    %    \centering
%    \includegraphics[width=3cm]{Wehrens_Martijn_32350_crop2_small.jpg}
%    \caption*{\textbf{Martijn Wehrens}}
%    \label{fig:cv:mw}
%\end{figure}

%\begin{figure}[H]
%    %    \centering
%    \includegraphics[width=3cm,right]{Wehrens_Martijn_32350_crop2_small.jpg}
%    \caption*{\textbf{Martijn Wehrens}}
%    \label{fig:cv:mw}
%\end{figure}

\begin{figure}
    \hfill
    \begin{minipage}[c]{0.25\textwidth}
        %\centering    
        %\includegraphics[width=1.0\textwidth]{pdf_2016-02-17_pos2_L31-mCerulean_clouds.pdf}
        \includegraphics{WehrensMartijn.jpg}
        \caption*{\textbf{Martijn Wehrens}}
    \end{minipage}
\end{figure}

%
%-2013 University of Amsterdam, MSc Chemistry
%
%
%
%
%However, since not all 
%
%Then, in 2004 he started the biology study program at the Radboud University of Nijmegen.
%%
%
%
%To gain deeper insights in what goes on at the biochemical level, 
%
%Fascinated by the biological processes that were studied, 
%
%1998-2004 Nijmeegse Scholengemeenschap Groenewoud (high school)
%engineering club, building and programming moving machines
%2007-2010 Radboud University Nijmegen, BSc Chemistry BSc Biology
%Honours program, student council, editor at student magazine ANS
%2010-2013 University of Amsterdam, MSc Chemistry
%eGFRD
%2014-2019 PhD program at AMOLF

%
%
%\section*{Overveiw}
%
%\begin{description}    
%    \item[2010-2013] University of Amsterdam, MSc Chemistry.
%    \item[2007-2010] Radboud University Nijmegen. \\
%    BSc Chemistry, 
%    BSc Biology \\
%    kjdhfkjhadsfkjhdjhf
%    %\item[2004-2010] Radboud University Nijmegen, BSc Biology.
%    
%    \item[1998-2004] Nijmeegse Scholengemeenschap Groenewoud (high school). \\
%    VWO level. Study programs "nature \& technology" and "nature \& biology".
%\end{description}

